\subsection{Алгоритми розв'язування задач}

\begin{problem*}
	Розглянемо лінійну систему керування 
	\begin{equation} 
	    \label{eq:algo-1-1}
	    \frac{\diff x(t)}{\diff t} = A (t) \cdot x(t) + B(t) \cdot u(t),
	\end{equation} 
	де $x \in \RR^n$ -- вектор фазових координат, $u \in \RR^m$ -- вектор керування, $A(t) \in \RR^{n\times n}$, $B(t) \in \RR^{n \times m}$ -- матриці з неперервними компонентами, $t \in [t_0, T]$. Задана початкова умова $x(t_0) = x_0$, де $x_0 \in \RR^n$.  Керування системою є відомим і розглядається в класі програмних керувань, або керувань з оберненим зв'язком. Необхідно відповісти на такі запитання:
	\begin{enumerate}
		\item визначити клас, до якого належить керування (програмне чи з оберненим зв'язком);
		\item знайти траєкторію системи, що відповідає заданому керуванню;
		\item якщо керування задане у класі керувань з оберненим зв'язком, то знайти керування у класі програмних керувань, яке відповідає заданому керуванню;
		\item визначити, до якого простору функцій належить траєкторія системи (неперервно диференційованих, кусково гладких, абсолютно неперервних);
		\item порівняти задане керування з іншим керуванням відносно критерію якості 
		\begin{equation}
		    \label{eq:algo-1-2}
		    \JJ(u) = \int_{t_0}^T f_0(x(t), u(t), t) \diff t + \Phi(x(T)) \to \inf.   
		\end{equation}
	Тут $f_0(x,u,t)$, $\Phi (x)$ -- неперервні скалярні функції;
	   \item знайти лінійну однорідну систему диференціальних рівнянь, одержану з системи \eqref{eq:algo-1-1} при підстановці керування $u(x,t) = C(t) \cdot x$, де $C(t) \in \RR^{n\times m}$ -- матриця з неперервними компонентами, $t \in [t_0, T]$. Обчислити нормовану за моментом $s$ фундаментальну матрицю знайденої системи, 
	   \item записати спряжену систему до знайденої в попередньому пункті системи. Обчислити  нормовану за моментом $s$ фундаментальну матрицю спряженої системи,  $s \in [t_0, T]$.
	
	\end{enumerate}
\end{problem*}

\begin{algorithm}
    \label{algo-1-1}
	Розглянемо підходи до розв'язування задачі.
	\begin{enumerate}
		\item Якщо керування має вигляд $u = u(t)$ і не залежить від вектора стану $x$, то керування є програмним. Інакше, якщо керування має вигляд $u = u(x, t)$, то  керування  є керуванням з оберненим зв'язком.
		\item Для знаходження траєкторії $x(t)$, яка відповідає заданому керуванню $u$, підставляємо керування $u$ в систему \eqref{eq:algo-1-1} і розв'язуємо її, враховуючи початкову умову.
		\item У керування $u = u(x, t)$ підставляємо знайдену у попередньому пункті траєкторію $x = x(t)$ і одержуємо програмне керування $u(t) = u(x(t), t)$.
		\item Якщо керування неперервне, то траєкторія системи є неперервно диференційованою. Якщо керування є  кусково неперервним, то траєкторія системи стає кусково гладкою. Щоб у цьому переконатись, у всіх точках розриву % негладкості?
		розглядаємо односторонні похідні (зліва і справа) відповідного розв'язку системи і порівнюємо їх. 
		\item Значення критерію якості \eqref{eq:algo-1-2} обчислюється на кожному з керувань і порівнюється.
		\item Підставляємо керування $u(x,t) = C(t) \cdot x$ в систему \eqref{eq:algo-1-1}. Одержуємо лінійну однорідну систему диференціальних рівнянь вигляду 
		\begin{equation}
		    \label{eq:algo-1-2a}
		\frac{\diff x(t)}{\diff t} = D (t) \cdot x(t), \ D(t) = A(t) + B(t) \cdot C(t).
		\end{equation}
		Фундаментальною матрицею системи \eqref{eq:algo-1-2a}, нормованою за моментом $s$, називається  $n \times n$ -матриця $\Theta(t, s)$, яка є розв'язком матричного рівняння \begin{equation}
		    \label{eq:algo-1-3}
		    \frac{\diff \Theta(t, s)}{\diff t} = D(t) \cdot \Theta(t, s), \quad \Theta(s, s) = I.
		\end{equation}
		Тому для знаходження $\Theta(t, s)$ ми спочатку знаходимо загальний розв'язок системи \eqref{eq:algo-1-2a}. Потім, використовуючи загальний розв'язок, знаходимо $n$ частинних розв'язків $x^{(k)}(t)$ системи \eqref{eq:algo-1-2a}, які відповідають умовам Коші $x(s) = e^{(k)}$, $k=1,2,\ldots,n$. Тут $e^{(k)}$ -- $k$-й орт (вектор, який складається з усіх нулів, крім $k$-го елементу, на місці якого стоїть 1). З \eqref{eq:algo-1-3} випливає, що вектори $x^{(k)}(t)$ є стовпчиками матриці $\Theta(t, s)$, $k=1,2,\ldots,n$. 
		\item Спряженою системою до системи \eqref{eq:algo-1-2a} називається система вигляду
		\begin{equation}
		    \label{eq:algo-1-4}
		    \frac{\diff y(t)}{\diff t} = - D^*(t) \cdot y(t),
		\end{equation}
		де $y = (y_1, \ldots, y_n)^*$. Фундаментальна матриця спряженої системи шукається з фундаментальної матриці системи \eqref{eq:algo-1-2a} за правилом $\Psi(t,s) = \Theta^*(s,t)$.
	\end{enumerate}
\end{algorithm}

\vspace*{\baselineskip}

\begin{problem*}
	Звести задачу керування системою \eqref{eq:algo-1-1} з функціоналом Больца вигляду \eqref{eq:algo-1-2}  до задачі з функціоналом, що залежить лише від кінцевого стану системи (функціонал Майєра).
\end{problem*}

\begin{algorithm}
    \label{algo-1-2}
    Перехід між задачами відбувається у кілька кроків:
	\begin{enumerate}
		\item Вводиться змінна \[x_{n+1} (t) \overset{\text{def}}{=} \int_{t_0}^t f_0(x(t), u(t), t) \diff t.\]
		\item Тоді \[ \JJ(u) = x_{n+1} (T) + \Phi(T). \]
		Такий критерій якості є функціоналом типу Майєра.
		\item До системи \eqref{eq:algo-1-2} додається рівняння \[ \frac{\diff x_{n+1}(t)}{\diff t} = f_0(x(t), u(t), t). \]
		Цим самим збільшується порядок системи на одиницю.
	\end{enumerate}
\end{algorithm}
