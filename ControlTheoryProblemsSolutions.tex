% cd ..\..\Users\NikitaSkybytskyi\Desktop\control-theory
% cls && pdflatex ControlTheoryProblemsSolutions.tex && cls && pdflatex ControlTheoryProblemsSolutions.tex && start ControlTheoryProblemsSolutions.pdf

\documentclass[a5paper, 10pt]{article}
\usepackage[T2A,T1]{fontenc}
\usepackage[utf8]{inputenc}
\usepackage[english, ukrainian]{babel}
\usepackage{amsmath, amssymb}
\usepackage[top = 2 cm, left = 1 cm, right = 1 cm, bottom = 2 cm]{geometry} 

\title{{\Huge ТЕОРІЯ КЕРУВАННЯ}}
\date{}

\usepackage{fancyhdr}
\pagestyle{fancy}
\lhead{Семінари з теорії керування, 2018}
\rhead{Нікіта Скибицький, ОМ-3}
\cfoot{\thepage}

\usepackage{amsthm}
\newtheorem{definition}{Означення}
\theoremstyle{definition}
\newtheorem*{problem*}{\normalfont{\textit{Задача}}}
\newtheorem{problem}{\normalfont{\textit{Задача}}}[section]
\newtheorem{algorithm}{Алгоритм}[section]
\newtheorem*{solution}{Розв'язок}

\allowdisplaybreaks
\setlength\parindent{0pt}
\numberwithin{equation}{section}

\usepackage{xcolor}
\usepackage{hyperref}
\hypersetup{unicode=true,colorlinks=true,linktoc=all,linkcolor=red}

\usepackage{graphicx}

\newcommand{\JJ}{\mathcal{J}}
\newcommand{\KK}{\mathcal{K}}
\newcommand{\MM}{\mathcal{M}}
\newcommand{\UU}{\mathcal{U}}
\newcommand{\XX}{\mathcal{X}}
\newcommand{\BB}{\mathcal{B}}
\newcommand{\NN}{\mathcal{N}}
\newcommand{\HH}{\mathcal{H}}
\newcommand{\EE}{\mathcal{E}}
\newcommand{\RR}{\mathbb{R}}
\newcommand{\Max}{\displaystyle\max\limits}

\newcommand*\diff{\mathop{}\!\mathrm{d}}

\renewcommand{\phi}{\varphi}
\renewcommand{\SS}{\mathcal{S}}
\renewcommand{\epsilon}{\varepsilon}

\DeclareMathOperator{\erf}{erf}
\DeclareMathOperator{\erfi}{erfi}
\DeclareMathOperator{\signum}{sgn}
\DeclareMathOperator*{\argmin}{arg\,min}

\begin{document}

\maketitle \thispagestyle{empty} \newpage 

У ваших руках конспект семінарських занять з нормативного курсу ``Теорія керування'', прочитаного доц., д.ф.-м.н. Пічкуром Володимиром Володимировичем на третьому курсі спеціальності прикладна математика факультету комп\-'ю\-тер\-них наук та кібернетики Київського національного університету імені Тараса Шевченка восени 2018-го року. \\

Конспект у компактній формі відображає матеріал курсу, допомагає сформувати загальне уявлення про предмет вивчення, правильно зорієнтуватися в даній галузі знань. Конспект лекцій з названої дисципліни сприятиме більш успішному вивченню дисципліни, причому більшою мірою для студентів заочної форми, екстернату, дистанційного та індивідуального навчання. \\

Упорядник безмежно вдячний Живолович Олександрі та Мельник Катерині а також решті групи ОМ-3, чиї безцінні конспекти лягли в основу цього збірника, та Антиповій Алісі за верстку частини задач. \\

Структура конспекту наступна: задачі розділені за темами (\tt{section}\normalfont), кожна тема містить три частини (\tt{subsection}\normalfont): 
\begin{enumerate}
	\item Алгоритми -- типові задачі теми із загальними алгоритмами розв'язування.
	\item Аудиторне заняття -- задачі, що пропонувалися для роботи на семінарі, абсолютна більшість із розв'язаннями.
	\item Домашнє завдання -- задачі, які пропонувалися (не всі) на домашню роботу, майже всі із розв'язаннями.
\end{enumerate}

Комп\-'ю\-тер\-ний набір та верстка -- Скибицький Нікіта Максимович.

\newpage

\tableofcontents \newpage

\section{Системи керування. Постановка задачі оптимального керування}

\subsection{Алгоритми}

\begin{problem*}
	Розглянемо лінійну систему керування 
	\begin{equation} 
	    \label{eq:algo-1-1}
	    \frac{\diff x(t)}{\diff t} = A (t) \cdot x(t) + B(t) \cdot u(t),
	\end{equation} 
	де $x \in \RR^n$ -- вектор фазових координат, $A(t) \in \RR^{n\times n}$, $u \in \RR^m$ -- вектор керування, $B(t) \in \RR^{n \times m}$, $t \in [t_0, T]$. Задані початкові умови $x(t_0) = x_0$, де $x_0 \in \RR^n$, керування системою є відомим. Необхідно:
	\begin{enumerate}
		\item Визначити клас керування (програмне чи з оберненим зв'язком).
		\item Знайти траєкторію системи, що відповідає заданому керуванню.
		\item Звести задане керування до програмного.
		\item Перевірити траєкторію на неперервну диференційованість.
		\item Порівняти задане керування з іншим керуванням відносно заданого критерію якості 
		\begin{equation}
		    \label{eq:algo-1-2}
		    \JJ(u) = \int_{t_0}^T f(x(t), u(t), t) \diff t + \Phi(x(T)) \to \min    
		\end{equation}
		\item Знайти фундаментальну матрицю системи, нормовану за моментом $s$. %, де $s \in \RR^1$.
		\item Побудувати спряжену систему диференціальних рівнянь.
	\end{enumerate}
\end{problem*}

\begin{algorithm}
    \label{algo-1-1}
	Розглянемо всі пункти задачі описані вище.
	\begin{enumerate}
		\item Якщо $u = u(t)$ не залежить від $x$, то керування програмне, інакше ($u = u(x, t)$) керування з оберненим зв'язком.
		\item Для знаходження траєкторії %просто
		розв'язується система (\ref{eq:algo-1-1}) з підставленим $u$.
		\item У визначенні $u = u(x, t)$ $x$ замінюється на знайдену у попередньому пункті траєкторію $x = x(t)$.
		\item Це питання має сенс якщо керування кусково-неперервне, тоді у всіх точках розриву % негладкості?
		необхідно розглянути односторонні похідні і перевірити їх на рівність. Якщо похідні рівні то траєкторія неперервно диференційовна, інакше ні.
		\item Значення критерію якості (\ref{eq:algo-1-2}) обчислюється на обох керуваннях і порівнюється.
		\item Розв'язується система диференціальних рівнянь 
		\begin{equation}
		    \label{eq:algo-1-3}
		    \frac{\diff \Theta(t, s)}{\diff t} = A(t) \cdot \Theta(t, s), \quad \Theta(s, s) = I.
		\end{equation}
		\item Спряженою системою до системи (\ref{eq:algo-1-1}) називається система вигляду
		\begin{equation}
		    \label{eq:algo-1-4}
		    \frac{\diff y(t)}{\diff t} = - A^*(t) \cdot y(t),
		\end{equation}
		де $y = (y_1, \ldots, y_n)^*$.
	\end{enumerate}
\end{algorithm}

\vspace*{\baselineskip}

\begin{problem*}
	Звести задачу Лагранжа (або Больца) з функціоналом вигляду (\ref{eq:algo-1-2}) за умов (\ref{eq:algo-1-1}) до задачі Майєра.
\end{problem*}

\begin{algorithm}
    \label{algo-1-2}
    Зведення відбувається у кілька кроків:
	\begin{enumerate}
		\item Вводиться змінна \[x_{n+1} (t) \overset{\text{def}}{=} \int_{t_0}^t f(x(t), u(t), t) \diff t.\]
		\item Тоді \[ \JJ(u) = x_{n+1} (T) + \Phi(T) \to \inf. \]
		\item До системи додається умова \[ \frac{\diff x_{n+1}(t)}{\diff t} = f(x(t), u(t), t). \]
	\end{enumerate}
\end{algorithm}
 \newpage
\input{01-WithSolutions.tex} \newpage
\input{01-NoSolutions.tex} \newpage

\section{Елементи багатозначного аналізу. Множина досяжності}

\subsection{Алгоритми}

\begin{problem*}
	Знайти
	\begin{enumerate}
		\item $A+B$;
		\item $\lambda A$;
		\item $\alpha(A,B)$;
		\item $MA$,
	\end{enumerate}
	де множини $A \subset\RR^m$, $B\subset\RR^m$, скаляр $\lambda\in\RR^1$, матриця $M\in\RR^{n\times m}$.
\end{problem*}

\begin{algorithm}
	\label{algo-2-1}
	Розглянемо всі пункти задачі вище.
	\begin{enumerate}
		\item Знаходимо за визначенням, $A+B=\{a+b|a\in A,b\in B\}$.
		\item Знаходимо за визначенням, $\lambda A =\{\lambda a|a\in A\}$. 
		\item \begin{enumerate}
			\item Знаходимо відхилення $\beta(A,B)$ і $\beta(B,A)$ за визначенням, 
			\begin{equation}
				\label{eq:2.1}
				\beta(A,B) = \max_{a\in A}\rho(a,B),
			\end{equation}
			де 
			\begin{equation}
				\label{eq:2.2}
				\rho(a,B) = \min_{b\in B} \rho(a,b).
			\end{equation}
			\item Знаходимо $\alpha(A,B)$ за визначенням, 
			\begin{equation}
				\label{eq:2.3}
				\alpha(A,B)=\max\{\beta(A,B),\beta(B,A).
			\end{equation}
		\end{enumerate} 
		\item Знаходимо за визначенням, $MA=\{Ma|a\in A\}$.
	\end{enumerate}
\end{algorithm}

\vspace*{\baselineskip}

\begin{problem*}
	Знайти опорну функцію множини $A \subset \RR^n$.
\end{problem*}

\begin{algorithm}
	\label{algo-2-2}
	\begin{enumerate}
		\item Намагаємося знайти за визначенням,
		\begin{equation}
		 	\label{eq:2.4}
		 	c(A,\psi) = \Max_{a\in A} \langle a, \psi \rangle.
		\end{equation}
		\item Якщо не вийшло, то намагаємося знайти за геометричною властивістю: $c(A,\psi)$ -- (орієнтована) відстань від початку координат до опорної площини множини $A$, для якої напрямок-вектор $\psi$ є вектором нормалі.
	\end{enumerate}
\end{algorithm}

\vspace*{\baselineskip}

\begin{problem*}
	Знайти інтеграл Аумана $\JJ = \int F(x) \diff x$, де $F(x)\subset\RR^n$.
\end{problem*}

\begin{algorithm}
	\label{algo-2-3}
	\begin{enumerate}
		\item Знаходимо опорну функцію від інтегралу:
		\begin{equation}
		 	\label{eq:2.5}
		 	c(\JJ, \psi) = \int c (F(x), \psi) \diff x.
		\end{equation}
		\item Знаходимо $\JJ$ як опуклий компакт з відомою опорною функцією $c(\JJ, \psi)$.
	\end{enumerate}
\end{algorithm}

\vspace*{\baselineskip}

\begin{problem*}
	Знайти множину досяжності системи $\dot x = A x + B u$, де $x(t_0) \in \mathcal{M}_0$, $u \in \mathcal{U}$.
\end{problem*}

\begin{algorithm}
	\label{algo-2-4}
	\begin{enumerate}
		\item Знаходимо фундаментальну матрицю $\Theta(t,s)$ системи нормовану за моментом $s$.
		\item Знаходимо інтеграл Аумана
		\begin{equation}
			\label{eq:2.6}
			\int_{t_0}^t \Theta(t, s) B(s) \UU(s) \diff s
		\end{equation}
		за алгоритмом \ref{algo-2-3}.
		\item Використовуємо теорему про вигляд множини досяжності лінійної системи керування:
		\begin{equation}
			\label{eq:2.7}
		 	\XX(t, \MM_0) = \Theta(t, t_0) \MM_0 + \int_{t_0}^t \Theta(t, s)B(s)\UU(s) \diff s.
		\end{equation}
	\end{enumerate}
\end{algorithm}

\vspace*{\baselineskip}

\begin{problem*}
	Знайти опорну функцію множини досяжності системи $\dot x = A x + B u$, де $x(t_0) \in \mathcal{M}_0$, $u \in \mathcal{U}$.
\end{problem*}

\begin{algorithm}
	\label{algo-2-5}
	\begin{enumerate}
		\item Знаходимо фундаментальну матрицю $\Theta(t,s)$ системи нормовану за моментом $s$.
		\item Знаходимо опорну функцію $c(\MM_0, \Theta^*(t, t_0) \psi)$ за алгоритмом \ref{algo-2-2}.
		\item Знаходимо опорну функцію $c(\UU(s), B^*(s) \Theta^*(t, s) \psi)$ за алгоритмом \ref{algo-2-2}.
		\item Використовуємо теорему про вигляд опорної функції множини досяжності лінійної системи керування: 
		\begin{equation}
			\label{eq:2.8}
			c(\XX(t, \MM_0), \psi) = c(\MM_0, \Theta^*(t, t_0) \psi) + \int_{t_0}^t c(\UU(s), B^*(s) \Theta^*(t, s) \psi) \diff s.
		\end{equation}
	\end{enumerate}
\end{algorithm}

 \newpage
\subsection{Задачі з розв'язками}

\begin{problem}
	Знайти $A + B$ і $\lambda A$, а також метрику Хаусдорфа $\alpha (A, B)$, якщо:
	
	\begin{enumerate}
	    \item $A = \{-3, 2, -1\},  B = \{-2, 5, 1\},  \lambda = 3$;
	    
	    \item $A = \{4, 2, -4\},  B = [-2, 3],  \lambda = -1$;
	    
	    \item $A = [-1, 2],  B = [3, 7],  \lambda = -2$;	    
	\end{enumerate}
	
\end{problem}

\begin{solution}
	Скористаємося пунктами 1-3 алгоритму \ref{algo-2-1}:
	\begin{enumerate}
	    \item \begin{align*} A + B &= \{-3-2, -3+5, -3+1\} \cup \{2-2, 2+5, 2+1\} \cup \\
	    &\quad\cup \{-1-2, -1+5, -1+1\} =\\&= \{-5,2,-2,0,7,3,-3,4,0\}=\{-5,-3-2,0,2,3,4,7\},\end{align*}
	    \[ \lambda A = \{3 \cdot -3, 3 \cdot 2, 3 \cdot -1 \} = \{-9,6,-3\}=\{-9,-3,6\}.\]
	    У нашому випадку \begin{align*}\beta (A, B) &= \max\left\{\min_{b\in B} \rho(-3,b), \min_{b\in B} \rho(2,b), \min_{b\in B} \rho(-1,b)\right\}=\\&=\max\{1,1,1\}=1, \\ \beta (B, A) &= \max\left\{\min_{a\in A} \rho(-2,a), \min_{a\in A} \rho(5,a), \min_{a\in A} \rho(1,a)\right\}=\\&=\max\{1,3,1\}=3,\end{align*} тому $\alpha (A, B) = \max\{1,3\}=3$.
	    \item \begin{align*} A + B &= [4-2,4+3] \cup [2-2,2+3] \cup [-4-2,-4+3] = \\
	        &= [2,7] \cup [0,5] \cup [-6,-1] = [-6,7].
	    \end{align*} 
	    \[ \lambda A = \{-1 \cdot 4, -1 \cdot 2, -1 \cdot -4 \} = \{-4,-2,4\}.\]
	    У нашому випадку \begin{align*}\beta (A, B) &= \max\left\{\min_{b\in B} \rho(4,b), \min_{b\in B} \rho(2,b), \min_{b\in B} \rho(-4,b)\right\}= \\ & = \max\{1,0,2\} = 2, \\ \beta (B, A) &= \max_{b\in B} \min \left\{\rho(b,4), \rho(b,2), \rho(b,-4)\right\} = 3,\end{align*} тому $\alpha (A, B) = \max\{2,3\}=3$.
	    \item $A + B = [2,9]$, $\lambda A = [-4,2]$, $\beta (A, B) = 4$, $\beta (B, A) = 5$, оскільки відповідні краї відхиляються на $4$ та $5$ відповідно ($-1$ від $3$ та $7$ від $2$), тому $\alpha (A, B) = 5$.
	\end{enumerate}
\end{solution}

\begin{problem}
	Знайти $MA$, якщо
	\[
	M=
  \begin{pmatrix}
    -2 & 4 \\
    3 & 5
  \end{pmatrix}
  , A= 
  \left\{
  \begin{pmatrix}
    -1 \\
    2 
  \end{pmatrix},
    \begin{pmatrix}
    3 \\
    -4 
  \end{pmatrix},
    \begin{pmatrix}
    0 \\
    -2 
  \end{pmatrix}
  \right\}
  .\]
\end{problem}

\begin{solution}
	Скористаємося пунктом 4 алгоритму \ref{algo-2-1}:
	\begin{align*} 
		\begin{pmatrix} -2 & 4 \\ 3 & 5 \end{pmatrix} \cdot \begin{pmatrix} -1 \\ 2 \end{pmatrix} &= \begin{pmatrix} 10 \\ 7 \end{pmatrix}, \\
		\begin{pmatrix} -2 & 4 \\ 3 & 5 \end{pmatrix} \cdot \begin{pmatrix} 3 \\ -4 \end{pmatrix} &= \begin{pmatrix} -22 \\ -11 \end{pmatrix}, \\
		\begin{pmatrix} -2 & 4 \\ 3 & 5 \end{pmatrix} \cdot \begin{pmatrix} 0 \\ -2 \end{pmatrix} &= \begin{pmatrix} -8 \\ -10 \end{pmatrix}.
	\end{align*}
  
  Отже, отримаємо \[MA = \left\{ \begin{pmatrix} 10 \\ 7 \end{pmatrix}, \begin{pmatrix} -22 \\ -11 \end{pmatrix}, \begin{pmatrix} -8 \\ -10 \end{pmatrix} \right\}.\]
\end{solution}

\begin{problem}
	Знайти опорні функції таких множин:

	\begin{enumerate}
		\item $A = [0, r]$;

		\item $A = [-r, r]$;

		\item $A = \{ (x_1, x_2): |x_1| \le 1, |x_2| \le 2 \}$;

		\item $A = \KK_r (0) = \{ x \in \RR^n: \|x\| \le r \}$;

		\item $A = \SS^n = \{ x \in \RR^n: \|x\| = 1 \}$.
	\end{enumerate}
\end{problem}

\begin{solution}
	У пунктах 1-3 скористаємося пунктом 1 алгоритму \ref{algo-2-2},  а у пунктах 4-5 -- пунктом 2 того ж алгоритму.
	\begin{enumerate}
		\item За означенням опорної функції, \[ c(A, \psi) = \max_{a \in A} \langle a, \psi \rangle = \begin{cases} 0, & \psi < 0 \\ r \psi, & 0 \le \psi \end{cases} = \max \{ 0, r \psi \}. \]

		\item За означенням опорної функції, \[ c(A, \psi) = \max_{a \in A} \langle a, \psi \rangle = \begin{cases} - r \psi, & \psi < 0 \\ r \psi, & 0 \le \psi \end{cases} = r |\psi |. \]

		\item За означенням опорної функції, \[ c(A, \psi) = \max_{a \in A} \langle a, \psi \rangle = \max_{a \in A} (\psi_1 x_1 + \psi_2 x_2) = |\psi_1| + 2 |\psi_2|. \]

		\item За властивістю опорної функції, $c(\KK_r (0), \psi) = r \| \psi \|$.

		\item За властивістю опорної функції, $c(\SS^n, \psi) = \| \psi \|$.
	\end{enumerate}
\end{solution}

\begin{problem}
	Знайти інтеграл Аумана $\JJ = \int_0^1 F(x) \diff x$ таких багатозначних відображень:

	\begin{enumerate}
		\item $F(x) = [0, x]$, $x \in [0, 1]$;

		\item $F(x) = \KK_x (0) = \{ y \in \RR^n: \|y\| \le x \}$, $x \in [0, 1]$.
	\end{enumerate}
\end{problem}

\begin{solution}
	Скористаємося алгоритмом \ref{algo-2-3}:
	\begin{enumerate}
		\item \[c(\JJ) = \int_0^1 c([0, x], \psi) \diff x = \begin{cases} \psi / 2, & 0 \le \psi \\ 0, & \psi < 0 \end{cases}. \]

		Далі знання опорних функцій підказують, що $\JJ = [0, 1 / 2]$.

		\item \[c(\JJ) = \int_0^1 c(\KK_x(0), \psi) \diff x = \int_0^1 x \| \psi \| \diff x = \| \psi \| / 2. \]

		Далі знання опорних функцій підказують, що $\JJ = \KK_{1 / 2} (0)$.
	\end{enumerate}
\end{solution}

\begin{problem}
	Знайти множину досяжності такої системи керування: \[ \frac{\diff x}{\diff t} = x + u, \]

	де $x (0) = x_0 \in \MM_0$, $u (t) \in \UU$, $t \ge 0$, \[ \MM_0 = \{ x: |x| \le 1 \}, \] \[ \UU = \{ u: |u| \le 1 \}. \]
\end{problem}

\begin{solution}
	Скористаємося алгоритмом \ref{algo-2-4}. \\

	Перш за все знаходимо $\Theta(t,s)$. Нескладно бачити, що $\Theta(t, s) = e^{t - s}$. \\

	Далі \[ \XX(t, \MM_0) = \Theta(t, t_0) \MM_0 + \int_{t_0}^t \Theta(t, s) B(s) \UU(s) \diff s. \]

	Підставимо вже відомі значення: 
	\begin{align*} 
		\XX(t, [-1, 1]) &= \Theta(t, 0) \cdot [-1, 1] + \int_0^t \left( \Theta(t, s) \cdot 1 \cdot [-1, 1] \right) \diff s = \\
		&= [-e^t, e^t] + \int_0^t [-e^{t - s}, e^{t - s}] \diff s = \\ 
		&= [-e^t, e^t] + [1 - e^t, e^t - 1] = [1 - 2 e^t, 2 e^t - 1]. 
	\end{align*}
\end{solution}

\begin{problem}
	Знайти опорну функцію множини досяжності для системи керування: \[
	\left\{
		\begin{aligned}
			\frac{\diff x_1}{\diff t} &= 2x_1 + x_2 + u_1, \\
			\frac{\diff x_2}{\diff t} &= 3x_1 + 4x_2 + u_2,
		\end{aligned}
	\right.
	\]
	де $x (0) = (x_{01}, x_{02}) \in \MM_0$, $u(t) = (u_1(t), u_2(t)) \in \UU$, $t \ge 0$, \[ \MM_0 = \{ (x_{01}, x_{02}): |x_{01}| \le 1, |x_{02}| \le 1 \}, \] \[ \UU = \{(u_1, u_2): |u_1| \le 1, |u_2| \le 1\}. \]
\end{problem}

\begin{solution}
	Скористаємося алгоритмом \ref{algo-2-5}. \\

	Перш за все знаходимо $\Theta(t,s)$ з системи \[ \frac{\diff \Theta(t, s)}{\diff t} = A(t) \cdot \Theta(t, s) = \begin{pmatrix} 2 & 1 \\ 3 & 4 \end{pmatrix} \Theta(t, s). \]

	Нескладно бачити, що \[ \Theta(t, s) = \frac{1}{4}
	\begin{pmatrix}
		3 e^{t - s} + e^{5 (t - s)} & - e^{t - s} + e^{5 (t - s)} \\ -3 e^{t - s} + 3 e^{5 (t - s)} & e^{t - s} + 3 e^{5 (t - s)}
	\end{pmatrix} 
	\]

	Скористаємося теоремою про вигляд опорної функції множини досяжності лінійної системи керування: \[ c(\XX(t, \MM_0), \psi) = c(\MM_0, \Theta^*(t, t_0) \psi) + \int_{t_0}^t c(\UU(s), B^*(s) \Theta^*(t, s) \psi) \diff s. \]

	Підставимо вже відомі значення: \begin{multline*}  c(\XX(t, [-1,1]^2), \psi) = c([-1,1]^2, \Theta^*(t, 0) \psi) + \\ + \int_0^t c\left([-1,1]^2, \begin{pmatrix} 1 & 0 \\ 0 & 1 \end{pmatrix} \Theta^*(t, s) \psi\right) \diff s, \end{multline*}

	і підставляємо $\Theta$:
	\begin{multline*} 
		c\left(\XX(t, [-1,1]^2), \begin{pmatrix} \psi_1 \\ \psi_2 \end{pmatrix}\right) = \\
		= c\left([-1,1]^2, \frac{1}{4} \begin{pmatrix} 3 e^t + e^{5 t} & -3 e^t + 3 e^{5 t} \\ - e^t + e^{5 t} & e^t + 3 e^{5 t}	\end{pmatrix} \begin{pmatrix} \psi_1 \\ \psi_2 \end{pmatrix} \right) + \\
		+ \int_0^t c\left([-1,1]^2, \frac{1}{4} \begin{pmatrix} 3 e^{t - s} + e^{5 (t - s)} & -3 e^{t - s} + 3 e^{5 (t - s)} \\ - e^{t - s} + e^{5 (t - s)} & e^{t - s} + 3 e^{5 (t - s)} \end{pmatrix} \begin{pmatrix} \psi_1 \\ \psi_2 \end{pmatrix} \right) \diff s = \\
		= c\left([-1,1]^2, \frac{1}{4} \begin{pmatrix} (3 e^t + e^{5 t}) \psi_1 + (-3 e^t + 3 e^{5 t}) \psi_2 \\ (- e^t + e^{5 t}) \psi_1 + (e^t + 3 e^{5 t}) \psi_2 \end{pmatrix}  \right) + \\
		+ \int_0^t c\left([-1,1]^2, \frac{1}{4} \begin{pmatrix} (3 e^{t - s} + e^{5 (t - s)}) \psi_1 + (-3 e^{t - s} + 3 e^{5 (t - s)}) \psi_2 \\ (- e^{t - s} + e^{5 (t - s)}) \psi_1 + (e^{t - s} + 3 e^{5 (t - s)}) \psi_2 \end{pmatrix} \right) \diff s = \\
		= \frac{1}{4} \left( \left|(3 e^t + e^{5 t}) \psi_1 + (-3 e^t + 3 e^{5 t}) \psi_2\right| + \left|(- e^t + e^{5 t}) \psi_1 + (e^t + 3 e^{5 t}) \psi_2\right| \right) + \\
		+ \frac{1}{4} \int_0^t \left|(3 e^{t - s} + e^{5 (t - s)}) \psi_1 + (-3 e^{t - s} + 3 e^{5 (t - s)}) \psi_2\right| \diff s + \\
		+ \frac{1}{4} \int_0^t \left|(- e^{t - s} + e^{5 (t - s)}) \psi_1 + (e^{t - s} + 3 e^{5 (t - s)}) \psi_2\right| \diff s.
	\end{multline*}
\end{solution}

% Тут необхідно доінтегрувати, але, здається, все одно нічого путнього (тобто добре відомої нам опорної функції) не вийде

\begin{problem}
    Знайти $A + B$ і $\lambda A$, а також метрику Хаусдорфа $\alpha(A, B)$, якщо
    \begin{enumerate}
        \item $A = \{4,-2,3\}$, $B = \{7,-1,1\}$, $\lambda=2$;
        \item $A = \{5,-5,2\}$, $B = [1,3]$, $\lambda=-1$;
        \item $A = [-4,-2]$, $B = [-1,5]$, $\lambda=3$;
    \end{enumerate}
\end{problem}

\begin{solution}
    \begin{enumerate}
        \item $A = \{4,-2,3\}$, $B = \{7,-1,1\}$, $\lambda=2$;
        \begin{multline*} 
            A + B = \{4+7,4-1,4+1,-2+7,-2-1,-2+1,3+7,3-1,3+1\}= \\
            = \{11,3,5,5,-3,-1,10,2,4\} = \{-3,-1,2,3,4,5,10,11\}.
        \end{multline*} 
        \[ \lambda A = \{2 \cdot 4, 2 \cdot -2, 2 \cdot 3\} = \{8, -4, 6\}. \]
        \begin{multline*} 
            \alpha(A,B) = \max\{\beta(A,B),\beta(B,A)\} = \\
            = \max\{\max\{3,1,2\},\max\{3,1,2\}\}=\max\{3,3\}=3.
        \end{multline*} 
        \item $A = \{5,-5,2\}$, $B = [1,3]$, $\lambda=-1$;
        \begin{multline*} 
            A + B = (5 + [1,3]) \cup (-5 + [1,3]) \cup (2+[1,3])= \\
            = [6,8] \cup [-4,-1] \cup [3,5] = [-4,-1] \cup [3,5] \cup [6,8].
        \end{multline*} 
        \[ \lambda A = \{-1 \cdot 5, -1 \cdot -5, -1 \cdot 2\} = \{-5, 5, -2\}. \]
        \begin{multline*} 
            \alpha(A,B) = \max\{\beta(A,B),\beta(B,A)\} = \\
            = \max\{\max\{2,6,0\},\max_{b\in[1,3]}\{|b-2|\}\}=\max\{6,1\}=6.
        \end{multline*} 
        \item $A = [-4,-2]$, $B = [-1,5]$, $\lambda=3$;
        \[ A + B = [-4-1,-2+5] = [-5,3]. \]
        \[ \lambda A = [3 \cdot -4, 3 \cdot -2] = [-12, -6]. \]
        \begin{multline*} 
            \alpha(A,B) = \max\{\beta(A,B),\beta(B,A)\} = \\
            = \max\{\max\{|-4+1|,|-2+1|\},\max\{|-1+2|,|5+2|\}\}=\max\{3,7\}=7.
        \end{multline*}
    \end{enumerate}
\end{solution}

\begin{problem}
    Знайти $MA$, якщо \[ M = \begin{pmatrix} 2 & 1 \\ -5 & 3 \end{pmatrix}, A = \left\{ \begin{pmatrix} -1 \\ 1 \end{pmatrix}, \begin{pmatrix} 2 \\ -4 \end{pmatrix}, \begin{pmatrix} -3 \\ -2 \end{pmatrix} \right\}. \]
\end{problem}

\begin{solution}
    \begin{multline*} 
        M A = \left\{ M \begin{pmatrix} -1 \\ 1 \end{pmatrix}, M \begin{pmatrix} 2 \\ -4 \end{pmatrix}, M \begin{pmatrix} -3 \\ -2 \end{pmatrix} \right\} = \\
        = \left\{ \begin{pmatrix} -1 \\ 8 \end{pmatrix}, \begin{pmatrix} 0 \\ -22 \end{pmatrix}, \begin{pmatrix} -8 \\ 9 \end{pmatrix} \right\}.
    \end{multline*}        
\end{solution}

\begin{problem}
    Знайти опорні функції таких множин:
    \begin{enumerate}
        \item $A = \{ -1, 1 \}$;
        \item $A = \{ (x_1, x_2, x_3) : |x_1| \le 2, |x_2| \le  4, |x_3| \le 1 \}$;
        \item $A = \{ a \}$;
        \item $A = \KK_r(a) = \{ x\in \RR^n : \| x - a \| \le r \}$.
    \end{enumerate}
\end{problem}

\begin{solution}
    \begin{enumerate}
        \item За визначенням, $c(A, \psi) = \Max_{x \in \{-1, 1\}} \langle x, \psi\rangle = \max(-\psi,\psi) = |\psi|$.
        \item За визначенням, $c(A, \psi) = \Max_{\substack{ x_1 : |x_1| \le 2 \\ x_2 : |x_2| \le  4 \\ x_3 : |x_3| \le 1 }} x_1 \psi_1 + x_2 \psi_2 + x_3 \psi_3  = 2 |\psi_1| + 4 |\psi_2| + |\psi_3|$.
        \item За визначенням, $c(A, \psi) = \Max_{x \in \{ a \}} \langle x, \psi\rangle = \langle a, \psi\rangle $.
        \item За визначенням, 
        \begin{align*}
            c(A, \psi) &= \Max_{x \in \RR^n : \| x - a \| \le r} \langle x, \psi\rangle = \Max_{y \in \RR^n : \| y \| \le r} \langle a + y, \psi\rangle = \\
            &= \langle a, \psi\rangle + \Max_{y \in \RR^n : \| y \| \le r} \langle y, \psi\rangle = \langle a, \psi\rangle + c(\KK_r(0), \psi) = \langle a, \psi \rangle + r\|\psi\|.
        \end{align*}
    \end{enumerate}
\end{solution}

\begin{problem}
    Знайти інтеграл Аумана $\JJ = \int_0^{\pi/2} F(x) dx$ таких багатозначних відображень:
    \begin{enumerate}
        \item $F(x) = [0, \sin x]$, $x \in [0, \pi / 2]$.
        \item $F(x) = [-\sin x, \sin x]$, $x \in [0, \pi / 2]$.
        \item $F(x) = \KK_{\sin x}(0) = \{ y \in \RR^n : \| y \| \le \sin x \}$, $x \in [0, \pi / 2]$.
    \end{enumerate}
\end{problem}

\begin{solution}
    Скористаємося теоремою про зміну порядку інтегрування і взяття опорної функції:
    \begin{enumerate}
        \item $c(\JJ, \psi) = \int_0^{\pi/2} c([0, \sin x], \psi) dx = \int_0^{\pi/2} \max(0, \psi) \sin x dx = \max(0, \psi)$, звідки $\JJ = [0, 1]$.
        \item $c(\JJ, \psi) = \int_0^{\pi/2} c([-\sin x, \sin x], \psi) dx = \int_0^{\pi/2} |\psi| \sin x dx = |\psi|$, звідки $\JJ = [-1, 1]$.
        \item $c(\JJ, \psi) = \int_0^{\pi/2} c(\KK_{\sin x}(0), \psi) dx = \int_0^{\pi/2} \sin x \|\psi\| dx = \|\psi\|$, звідки $\JJ = \KK_1(0)$. 
    \end{enumerate}
\end{solution}

\begin{problem}
    Знайти множину досяжності такої системи керування:
    \[\frac{\diff x}{\diff t} = x + bu,\] 
    де $x(0) = x_0 \in \MM_0$, $u(t)\in \UU$, $t\ge0$, $b$ -- деяке ненульове число, 
    \[ \MM_0 = \{ x : | x | \le 2 \}, \]
    \[ \UU = \{ u : |u| \le 3 \}. \]
\end{problem}

\begin{solution}
    Множину досяжності знайдемо через її опорну функцію: 
    \[ c(\XX(t, \MM_0), \psi) = c(\MM_0, \Theta^*(t, t_0) \psi) + \int_{t_0}^t c(\UU(s), C^\star(s) \Theta^*(t, s)\psi) \diff s. \]
    Для цього послідовно знаходимо: \\
    
    $\Theta(t, s) = e^{t-s}$, знайдено із рівності $\dfrac{d\Theta(t,s)}{dt} = A(t)\Theta(t,s) = \Theta(t,s)$ у нашому випадку. \\
    
    $c(\MM_0, \psi) = c([-2, 2], \psi) = 2 |\psi|$, вже достатньо відома нам опорна функція. \\
    
    $c(\UU(s), \psi) = c([-3, 3], \psi) = 3 |\psi|$, ще одна вже достатньо відома нам опорна функція. \\
    
    % Послідовно пісдтавляючи знайдені вирази в формулу вище знаходимо:
    % \begin{equation*}
    % \begin{split}
    %     c(X(t, \MM_0), \psi) &= c(\MM_0, \Theta^\star(t, t_0), \psi) + \int_{t_0}^t c(\UU(s), C^\star(s) \Theta^\star(t, s)\psi) ds = \\
    %     &= c([-2,2], \Theta^\star(t, 0), \psi) + \int_0^t c([-3, 3], b \Theta^\star(t, s)\psi) ds = \\
    %     &= 2\left|\Theta^\star(t, 0)\psi\right| + \int_0^t 3\left|b \Theta^\star(t, s)\psi\right| ds = \\
    %     &= 2\left|e^{-t}\psi\right| + \int_0^t 3\left|b e^{s-t}\psi\right| ds = 2e^{-t}|\psi| + 3|b \psi| \int_0^t e^{s-t} ds = \\
    %     &= 2e^{-t}|\psi| + 3|b \psi| \left(1 - e^{-t}\right) = \left(2e^{-t} + 3|b|\left(1 - e^{-t}\right)\right) |\psi|,
    % \end{split}
    % \end{equation*}
    % звідки $X(t, \MM_0) = \left[-2e^{-t} - 3|b|\left(1 - e^{-t}\right), 2e^{-t} + 3|b|\left(1 - e^{-t}\right)\right]$.
    
    Послідовно пісдтавляючи знайдені вирази в формулу вище знаходимо:
    \begin{equation*}
    \begin{split}
        c(X(t, \MM_0), \psi) &= c(\MM_0, \Theta^\star(t, t_0), \psi) + \int_{t_0}^t c(\UU(s), C^\star(s) \Theta^\star(t, s)\psi) ds = \\
        &= c([-2,2], \Theta^\star(t, 0), \psi) + \int_0^t c([-3, 3], b \Theta^\star(t, s)\psi) ds = \\
        &= 2\left|\Theta^\star(t, 0)\psi\right| + \int_0^t 3\left|b \Theta^\star(t, s)\psi\right| ds = \\
        &= 2\left|e^t\psi\right| + \int_0^t 3\left|b e^{t-s}\psi\right| ds = 2e^t|\psi| + 3|b \psi| \int_0^t e^{t-s} ds = \\
        &= 2e^t|\psi| + 3|b \psi| \left(e^t - 1\right) = \left(2e^t + 3|b|\left(e^t - 1\right)\right) |\psi|,
    \end{split}
    \end{equation*}
    звідки $\XX(t, \MM_0) = \left[-2e^t - 3|b|\left(e^t - 1\right), 2e^t + 3|b|\left(e^t - 1\right)\right]$.
\end{solution}

\begin{problem}
Знайти опорну функцію множини досяжності для системи керування:
\begin{equation*}
    \left\{
    \begin{aligned}
    \dfrac{dx_1}{dt} &= x_1 - x_2 + 2u_1, \\
    \dfrac{dx_2}{dt} &= -4x_1 + x_2 + u_2,
    \end{aligned}
    \right.
\end{equation*}
де $x(0) = (x_{01}, x_{02}) \in \mathcal{M}_0$, $u(t) = (u_1(t), u_2(t)) \in\mathcal{U}$, $t\ge0$,
\begin{align*}
    \mathcal{M}_0 &= \{(x_{01},x_{02}): x_{01}^2 + x_{02}^2 \le 4\}, \\
    \mathcal{U} &= \{(u_1, u_2): u_1^2 + u_2^2 \le 1\}.
\end{align*}
\end{problem}

\begin{solution}
    Одразу помітимо, що $C=\begin{pmatrix}2&0\\0&1\end{pmatrix}$.\\

    $\Theta(t,s)$ знайдемо розв'язавши однорідну систему:
    \begin{equation*}
        \left\{
        \begin{aligned}
        \dfrac{dx_1}{dt} &= x_1 - x_2, \\
        \dfrac{dx_2}{dt} &= -4x_1 + x_2,
        \end{aligned}
        \right.
    \end{equation*}
    
    Її визначник $\begin{vmatrix} 1 - \lambda & - 1 \\ - 4 & 1 - \lambda \end{vmatrix} = (1 - \lambda)^2 - 4 = (\lambda + 1) (\lambda - 3) = 0$, звідки $\lambda_1 = -1$, $\lambda_2 = 3$. \\
    
    Підставляючи знайдені числа у систему, знаходимо власні вектори: $\begin{pmatrix} 1 \\ 2 \end{pmatrix}$ та $\begin{pmatrix} 1 \\ -2 \end{pmatrix}$ відповідно. \\

    Отже загальний розв'язок має вигляд \[\begin{pmatrix} x_1 \\ x_2 \end{pmatrix}(t) = c_1 \begin{pmatrix} e^{-t} \\ 2e^{-t} \end{pmatrix} + c_2 \begin{pmatrix} e^{3t} \\ -2e^{3t} \end{pmatrix}\]
    
    Розв'язуючи рівняння
    \[ c_1 \begin{pmatrix} e^{-s} \\ 2e^{-s} \end{pmatrix} + c_2 \begin{pmatrix} e^{3s} \\ -2e^{3s} \end{pmatrix} = \begin{pmatrix} 1 \\ 0 \end{pmatrix} \]
    і
    \[ c_1 \begin{pmatrix} e^{-s} \\ 2e^{-s} \end{pmatrix} + c_2 \begin{pmatrix} e^{3s} \\ -2e^{3s} \end{pmatrix} = \begin{pmatrix} 0 \\ 1 \end{pmatrix}, \]
    знаходимо фундаментальну матрицю системи, нормовану за моментом $s$, а саме 
    \[ \Theta(t,s) = \begin{pmatrix} \dfrac{e^{s-t} + e^{3(t-s)}}{2} & \dfrac{e^{s-t} - e^{3(t-s)}}{4} \\ e^{s-t} - e^{3(t-s)} & \dfrac{e^{s-t} + e^{3(t-s)}}{2} \end{pmatrix} \]
    
    
    Далі знаходимо $c(\mathcal{M}_0, \psi) = c(\mathcal{K}_2(0), \psi) = 2\|\psi\|$, та $c(\mathcal{U}, \psi) = c(\mathcal{K}_1(0), \psi) = \|\psi\|$, вже достатньо відомі нам опорні функції. \\
    
    Нарешті, можемо зібрати це все докупи: 
    \begin{align*}
        c(\mathcal{X}(t, \mathcal{M}_0), \psi) &= c(\mathcal{M}_0, \Theta^\star(t, 0) \psi) + \int_{0}^t c(\mathcal{U}(s), C^\star(s) \Theta^\star(t, s)\psi) ds = \\
        \\
        &= 2 \|\Theta^\star(t, 0) \psi\| + \int_{0}^t \left\|C^\star(s) \Theta^\star(t, s)\psi\right\| ds = \\
        \\
        &= 2 \left\|\begin{pmatrix} \dfrac{e^{-t} + e^{3t}}{2} & e^{3t} - e^{-t} \\ \dfrac{e^{3t} - e^{-t}}{4} & \dfrac{e^{-t} + e^{3t}}{2} \end{pmatrix} \begin{pmatrix} \psi_1 \\ \psi_2 \end{pmatrix}\right\| + \\
        \\
        &+ \int_{0}^t \left\|\begin{pmatrix} e^{s-t} + e^{3(t-s)} & 2(e^{3(t-s)} - e^{s-t}) \\ \dfrac{e^{3(t-s)} - e^{s-t}}{4} & \dfrac{e^{s-t} + e^{3(t-s)}}{2} \end{pmatrix} \begin{pmatrix} \psi_1 \\ \psi_2 \end{pmatrix}\right\| ds = \\
        \\
        &= 2 \left\| \begin{pmatrix} \dfrac{e^{-t} + e^{3t}}{2} \cdot \psi_1 + (e^{3t} - e^{-t}) \cdot \psi_2 \\ \dfrac{e^{3t} - e^{-t}}{4}\cdot\psi_1 + \dfrac{e^{-t} + e^{3t}}{2}\cdot\psi_2 \end{pmatrix} \right\| + ...
    \end{align*}
\end{solution}  \newpage
\subsection{Задачі для самостійного розв'язування}

 \newpage

\section{Задача про переведення системи з точки в точку. Критерії керованості лінійної системи керування}

\input{03-Algorithms.tex} \newpage
\subsection{Задачі з розв'язками}

\begin{problem}
	Перевести систему \[ \frac{\diff x}{\diff t} = u, \quad t \in [0, T], \] з точки $x (0) = x_0$ в точку $x (T) = y_0$ за допомогою керування з класу:
	\begin{enumerate}
		\item постійних функцій $u (t) = c$, $c$ -- константа;

		\item кусково-постійних функцій \[ u (t) = \begin{cases} c_1, & t \in [0, t_1], \\ c_2, & t \in [t_1, T]. \end{cases} \]

		Тут $c_1$, $c_2$ -- константи, $c_1 \ne c_2$, $0 < t_1 < T$;

		\item програмних керувань $u(t) = c t$, $c$ -- константа;

		\item керувань з оберненим зв'язком $u(x) = c x$, $c$ -- константа.
	\end{enumerate}
\end{problem}

\begin{solution}
	Скористаємося формулою $x (T) = x (0) + \int_0^T \frac{\diff x}{\diff t} \diff t$:

	\begin{enumerate}
		\item \[x (T) = x (0) + \int_0^T c \diff t = x (0) + c T,\] звідки \[c = \frac{x (T) - x (0)}{T} = \frac{y_0 - x_0}{T};\]

		\item \[ x (T) = x (0) + \int_0^{t_1} c_1 \diff t + \int_{t_1}^T c_2 \diff t = x (0) + c_1 t_1 + c_2 (T - t_1). \] Розв'язок не єдиний, \[ c_2 = \frac{x (T) - x (0) - c_1 t_1}{T - t_1}, \] де $c_1$ -- довільна стала, наприклад $c_1 = 0$, тоді \[ c_2 = \frac{x (T) - x (0)}{T - t_1} = \frac{y_0 - x_0}{T - t_1}. \]

		\item \[ x (T) = x (0) + \int_0^T c t \diff t = x (0) + \frac{cT^2}{2}, \] звідки \[ c = \frac{2 (x (T) - x (0))}{T^2} = \frac{2 (y_0 - x_0)}{T^2}. \]

		\item У цьому випадку проінтегрувати не можна, бо $u$ залежить від $x$, тому просто запишемо за формулою Коші \[ x (T) = x (0) \cdot e^{c T}, \] звідки \[c = \frac{\ln(x (T) / x (0))}{T} = \frac{\ln(y_0) - \ln(x_0)}{T}. \]

		Варто зауважити, не для всіх пар $x_0$ і $y_0$ коректно визначається значення $c$. А саме, необхідно щоб $y_0$ було того ж знаку, що і $x_0$.
	\end{enumerate}
\end{solution}

\begin{problem}
	\begin{enumerate}
		\item Використовуючи означення, знайти грамміан керованості для системи керування \[ \frac{\diff x(t)}{\diff t} = t x (t) + \cos (t) \cdot u(t), \quad t \ge 0. \]

		\item Записати диференціальне рівняння для грамміана керованості і за його допомогою знайти грамміан керованості.

		\item Використовуючи критерій керованості, вказати інтервал повної керованості вказаної системи керування. Для цього інтервала записати керування, яке розв'язує задачу про переведення системи з точки $x_0$ у стан $x_T$.
	\end{enumerate}
\end{problem}

\begin{solution}
	\begin{enumerate}
		\item Скористаємося формулою \[\Phi(T, t_0) = \int_{t_0}^T \Theta(T, s) B(s) B^*(s) \Theta^*(T, s) \diff s.\] $\Theta(T, s)$ знаходимо з системи \[ \frac{\diff \Theta(t,s)}{\diff t} = A(t) \cdot \Theta(t, s) = t \cdot \Theta(t, s),\] а саме $\Theta(t, s) = \exp\left\{\frac{t^2 - s^2}{2}\right\}$. Підставляючи всі знайдені значення, отримаємо \[\Phi(T, t_0) = \cos^2(T) \cdot e^{T^2} \int_{t_0}^T e^{-s^2} \diff s = \frac{1}{2} \sqrt{\pi} \cdot \cos^2(T) \cdot e^{T^2} \cdot \erf(T).\]

		\item Запишемо систему \[ \frac{\diff \Phi (t, t_0)}{\diff t} = A(t) \cdot \Phi(t, t_0) + \Phi(t, t_0) \cdot A^*(t) + B(t) \cdot B^*(t), \quad \Phi(t_0, t_0) = 0. \]

		І підставимо відомі значення: \[ \frac{\diff \Phi (t, 0)}{\diff t} = 2 t \Phi(t, 0)  + \cos^2(t), \quad \Phi(0, 0) = 0. \]

		Звідси \[ \Phi(t, 0) = \frac{1}{8} \sqrt{\pi} e^{t^2 - 1} ( - 2 e \erf(t) + i (\erfi(1 + i t) - i \erfi(1 - i t)), \]

		а \[ \Phi(T, 0) = \frac{1}{8} \sqrt{\pi} e^{T^2 - 1} ( - 2 e \erf(T) + i (\erfi(1 + i T) - i \erfi(1 - i T)), \]

		\item З вигляду грамміану керованості отриманого у першому пункті очевидно, що система цілком керована на півінтервалі $[0, \pi / 2)$, зокрема на інтервалі $[0, 1]$. \\

		Підставимо тепер граміан у формулу для керування що розв'язує задачу про переведення системи із стану $x_0$ у стан $x_T$:
		\begin{multline*} u(t) = B^*(t) \Theta^*(T, t) \Phi^{-1}(T, t_0) (x_T - \Theta(T, t_0) x_0) = \\ = \cos(t) \cdot \exp\left\{\frac{T^2-t^2}{2}\right\} \Phi^{-1} (T, 0) \left(x_T - \exp\left\{\frac{T^2}{2}\right\} x_0\right) \end{multline*}
	\end{enumerate}
\end{solution}

\begin{problem}
	За допомогою грамміана керованості розв'язати таку задачу оптимального керування: мінімізувати критерій якості \[ \JJ (u) = \int_0^T u^2(s) \diff s\] за умов, що \[ \frac{\diff x(t)}{\diff t} = \sin(t) \cdot x(t) + u(t), \quad x (0) = x_0, x (T) = x_T. \]

	Тут $x$ -- стан системи, $u(t)$ -- скалярне керування, $x_0$, $X_T$ -- задані точки, $t \in [t_0, T]$.
\end{problem}

\begin{solution}
	Знайдемо шукане керування за формулою \[ u(t) = B^*(t) \Theta^*(T, t) \Phi^{-1}(T, t_0) (x_T - \Theta(T, t_0) x_0). \]

	У цій задачі $\Theta(t, s) = e^{\cos (s) - \cos (t)}$, знайдене з системи $\dot \Theta = A \Theta$, $\Phi(T, t_0) = e^{-2 \cos (T)} \int_0^T e^{2 \cos (s)} \diff s$, підставляючи знаходимо \[ u(t) = \frac{e^{\cos (t) + \cos (T)} \cdot (x_T - e^{1 - \cos (T)} x_0)}{\int_0^T e^{2 \cos (s)} \diff s}. \]
\end{solution}

\begin{problem}
    За допомогою грамміана керованості розв'язати таку задачу оптимального керування: мінімізувати критерій якості
    \[ \mathcal{J}(u) = \int_0^T u^2(s) dx \]
    за умов, що
    \[ \dfrac{d^2x(t)}{dt^2} - 5\dfrac{dx(t)}{dt} + 6x(t) = u(t), \]
    \[ x(0) = x_0, x'(0) = y_0, x(T) = x'(T) = 0.\]
    Тут $x$ -- стан системми, $u(t)$ -- скалярне керування, $t \in [0, T]$.
\end{problem}

\begin{solution}
    Почнемо з того що зведемо рівняння другого порядку до системи рівнянь заміною $x_1 = x$, $x_2 = \dot x_1$, тоді маємо систему
    \[ \begin{pmatrix} \dot x_1 \\ \dot x_2 \end{pmatrix} (t) = \begin{pmatrix} 0 & 1 \\ -6 & 5 \end{pmatrix} \begin{pmatrix} x_1 \\ x_2 \end{pmatrix} (t) + \begin{pmatrix} 0 \\ 1 \end{pmatrix} u(t). \]
    
    Знайдемо власні числа матриці $A - \lambda E$: $\det(A - \lambda E) = \begin{vmatrix} -\lambda & 1 \\ -6 & 5-\lambda \end{vmatrix} = \lambda^2 - 5\lambda + 6 = (\lambda - 2) (\lambda - 3) = 0$, звідки $\lambda_1 = 2$, $\lambda_2 = 3$. Знайдемо власні вектори, вони будуть $\begin{pmatrix} 1 \\ 2 \end{pmatrix}$ і $\begin{pmatrix} 1 \\ 3 \end{pmatrix}$ відповідно. Звідси знаходимо загальний розв'язок
    \[ \begin{pmatrix} x_1 \\ x_2 \end{pmatrix} (t) = c_1 \begin{pmatrix} e^{2t} \\ 2e^{2t} \end{pmatrix} + c_2 \begin{pmatrix} e^{3t} \\ 3e^{3t} \end{pmatrix}. \]
    
    З рівняння
    \[c_1 \begin{pmatrix} e^{2s} \\ 2e^{2s} \end{pmatrix} + c_2 \begin{pmatrix} e^{3s} \\ 3e^{3s} \end{pmatrix} = \begin{pmatrix} 1 \\ 0 \end{pmatrix} \]
    знаходимо $c_1 = 3e^{-2s}$, $c_2 = -2e^{-3s}$, а з рівняння
    \[c_1 \begin{pmatrix} e^{2s} \\ 2e^{2s} \end{pmatrix} + c_2 \begin{pmatrix} e^{3s} \\ 3e^{3s} \end{pmatrix} = \begin{pmatrix} 0 \\ 1 \end{pmatrix} \]
    знаходимо $c_1 = -e^{-2s}$, $c_2 = e^{-3s}$, тобто
    \[ \Theta(t, s) = \begin{pmatrix} 3e^{2(t-s)} - 2e^{3(t-s)} & -e^{2(t-s)} + e^{3(t-s)} \\ 6e^{2(t-s)} - 6e^{3(t-s)} & -2e^{2(t-s)} + 3e^{3(t-s)} \end{pmatrix}. \]
    
    Знайдемо грамміан за формулою \[\Phi(T, 0) = \int_0^T \Theta(T, s) B(s) B^* (s) \Theta^*(T, s) ds. \]
    
    \[ \Theta(T, s) B(s) = \begin{pmatrix} -e^{2(T - s)} + e^{3(T - s)} \\ -2e^{2(T - s)} + 3e^{3(T - s)} \end{pmatrix}. \]
    \[ B^* (s) \Theta^*(T, s)  =  (\Theta(T, s) B(s))^\star = \begin{pmatrix} -e^{2(T - s)} + e^{3(T - s)} & -2e^{2(T - s)} + 3e^{3(T - s)} \end{pmatrix}. \]
    
    \begin{align*} 
        \Phi(T, 0) &= \int_0^T \begin{pmatrix} -e^{2(T - s)} + e^{3(T - s)} \\ -2e^{2(T - s)} + 3e^{3(T - s)} \end{pmatrix} \begin{pmatrix} -e^{2(T - s)} + e^{3(T - s)} & -2e^{2(T - s)} + 3e^{3(T - s)} \end{pmatrix} ds = \\
        &= \int_0^T \begin{pmatrix} e^{4(T - s)} - 2 e^{5(T - s)} + e^{6(T - s)} & 2 e^{4(T - s)} - 5 e^{5(T - s)} + 3 e^{6(T - s)} \\ 2e^{4(T - s)} - 5 e^{5(T - s)} + 3 e^{6(T - s)} & 4 e^{4(T - s)} - 12 e^{5(T - s)} + 9  e^{6(T - s)} \end{pmatrix} ds = \\
        &= \begin{pmatrix} \dfrac{e^{4T} - 1}{4} - \dfrac{2(e^{5T} - 1)}{5} + \dfrac{e^{6T} - 1}{6} & \dfrac{e^{4T} - 1}{2} - (e^{5T} - 1) + \dfrac{e^{6T} - 1}{2} \\ \\ \dfrac{e^{4T} - 1}{2} - (e^{5T} - 1) + \dfrac{e^{6T} - 1}{2} & (e^{4T} - 1) - \dfrac{12(e^{5T} - 1)}{5} + \dfrac{3(e^{6T} - 1)}{2} \end{pmatrix}
    \end{align*}
    
    Обчислення визначника грамміану є громіздкою обчислювальною задачею, ми залишаємо її як вправу для читача.
\end{solution}

\begin{problem}
	Записати систему диференціальних рівнянь для знаходження першої матриці керованості (грамміана керованості) і сформулювати критерій керованості на інтервалі $[0, T]$ у випадку, якщо система керування має вигляд:
	\begin{enumerate}
		\item \[ 
		\left\{
			\begin{aligned}
				\frac{\diff x_1(t)}{\diff t} = t x_1 (t) + x_2 (t) + u_1 (t), \\
				\frac{\diff x_2(t)}{\diff t} = - x_1 (t) + 2 x_2 (t) + t^2 u_2 (t).
			\end{aligned}
		\right.
		\]

		Тут $x = (x_1, x_2)^*$ -- вектор стану, $u = (u_1, u_2)^*$ -- вектор керування, $t \in [0, T]$.

		\item \[ \frac{\diff^2 x(t)}{\diff t^2} + \sin(t) \cdot x(t) = u(t). \]

		Тут $x$ -- стан системи, $u(t)$ -- скалярне керування, $t \in [0, T]$.
	\end{enumerate}
\end{problem}

\begin{solution}
	\begin{enumerate}
		\item $A = \begin{pmatrix} t & 1 \\ -1 & 2 \end{pmatrix}$, $B = \begin{pmatrix} 1 & 0 \\ 0 & t^2 \end{pmatrix}$, \[
		\left\{
			\begin{aligned}
				\dot \phi_{11} &= 2 t \phi_{11} + 2 \phi_{12} + 1, \\
				\dot \phi_{12} &= - \phi_{11} + (t + 2) \phi_{12} + \phi_{22}, \\
				\dot \phi_{21} &= \ldots 
			\end{aligned}
		\right.
		\]

		\item Введемо нову змінну $x_2 = \dot x$, тоді $A = \begin{pmatrix} 0 & 1 \\ -\sin(t) & 0 \end{pmatrix}$, $B = \begin{pmatrix} 0 \\ 1 \end{pmatrix}$, \[
		\left\{
			\begin{aligned}
				\dot \phi_{11} &= 2 \phi_{12}, \\
				\dot \phi_{12} &= (1 - \sin(t)) \phi_{11}, \\
				\ldots 
			\end{aligned}
		\right.
		\]
	\end{enumerate}
\end{solution}

\begin{problem}
    Дослідити системи на керованість використовуючи другий критерій керованості:
    \begin{enumerate}
        \item \[\ddot x + a \dot x + b x = u; \]
        \item \[ \left\{ \begin{aligned} \dot x_1 &= 2x_1 + x_2 + au \\ \dot x_2 &= x_1 + 4 x_2 + u \end{aligned} \right. \]
        \item \[ \left\{ \begin{aligned} \dot x_1 &= 2x_1 + x_2 + u_1 \\ \dot x_2 &= x_1 + 3 x_3 + u_2 \\ \dot x_3 &= x_2 + x_3 + u_2  \end{aligned} \right. \]
    \end{enumerate}
\end{problem}

\begin{solution}
    \begin{enumerate}
        \item Почнемо з того що зведемо рівняння другого порядку до системи рівнянь заміною $x_1 = x$, $x_2 = \dot x_1$, тоді маємо систему
        \[ \left\{ \begin{aligned} \dot x_1 &= x_2 \\ \dot x_2 &= - a x_2 - b x_1 + u  \end{aligned} \right. \]
        Тоді
        \[ A = \begin{pmatrix} 0 & 1 \\ -b & -a \end{pmatrix} \qquad B = \begin{pmatrix} 0 \\ 1 \end{pmatrix}. \]
        \[ D = \begin{pmatrix} B & AB \end{pmatrix} = \begin{pmatrix} 0 & 1 \\ 1 & - a \end{pmatrix}. \]
        Її ранг дорівнює 2 якщо за будь-яких $a$ і $b$, тобто система завжди цілком керована.
        \item 
        \[ A = \begin{pmatrix} 2 & 1 \\ 1 & 4 \end{pmatrix} \qquad B = \begin{pmatrix} a \\ 1 \end{pmatrix}. \]
        \[ D = \begin{pmatrix} B & AB \end{pmatrix} = \begin{pmatrix} a & 2 a + 1 \\ 1 & a + 4 \end{pmatrix}. \]
        Її визначник $a^2 + 4a - 2a - 1 = a^2 + 2a - 1 = 0$ якщо $a = -1 \pm \sqrt 2$, тоді система не є цілком керованою, а інакше є.
        \item 
        \[ A = \begin{pmatrix} 2 & 1 & 0 \\ 1 & 0 & 3 \\ 0 & 1 & 1 \end{pmatrix} \qquad B = \begin{pmatrix} 1 & 0 \\ 0 & 1 \\ 0 & 1  \end{pmatrix}. \]
        \[ D = \begin{pmatrix} B & AB & A^2B \end{pmatrix} = \begin{pmatrix} 1 & 0 & 2 & 1 & 5 & 5 \\ 0 & 1 & 1 & 3 & 2 & 7 \\ 0 & 1 & 0 & 2 & 1 & 5 \end{pmatrix} .\]
        Її ранг дорівнює 3, тобто система цілком керована.
    \end{enumerate} 
\end{solution}


\begin{problem}
    Перевести систему 
    \[ \dfrac{dx}{dt} = 2tx + u, t \in [0, T], \]
    з точки $x(0) = x_0$ в точку $x(T) = y_0$ за допомогою керування з класу:
    \begin{enumerate}
        \item постійних функцій $u(t) = c$, $c$ -- константа; 
        \item кусково-постійних функцій 
        \[
        u(t) = \begin{cases}
            c_1 & t \in [0, t_1), \\
            c_2 & t \in (t_1, T].
        \end{cases}
        \]
        Тут $c_1$, $c_2$ -- константи, $c_1 \ne c_2$, $0 < t_1 < T$;
        \item програмних керувань $u(t) = ct$, $c$ -- константа;
        \item керувань з оберненим зв'язок $u(x) = cx$, $c$ -- константа.
    \end{enumerate}
\end{problem}

\begin{solution}
Будемо просто підставляти керування у диференційне рівняння і розв'язувати його:
\begin{enumerate}
\item Зводимо до канонічного вигляду лінійного рівняння:
\[ \dfrac{dx}{dt} - 2t \cdot x(t) = c. \]
Домножаємо на множник що інтегрує:
\begin{align*}
    \exp\{-t^2\} \cdot \dfrac{dx}{dt} - 2 t \cdot \exp\{-t^2\} \cdot x(t) &= c \cdot \exp\{-t^2\} \\
    \\
    \exp\{-t^2\} \cdot \dfrac{dx}{dt} + x(t) \cdot \dfrac {d \exp\{-t^2\}} {dt} &= c \cdot \exp\{-t^2\}.
\end{align*}
Згортаємо похідну добутку:
\[ \dfrac {d (\exp\{-t^2\} \cdot x(t))} {dt} = c \cdot \exp\{-t^2\}. \]
Інтегруємо:
\begin{align*}
    \left.(\exp\{-t^2\} \cdot x(t))\right|_0^T &= \int_0^T c \cdot \exp\{-t^2\} dt \\
    \\
    \exp\{-T^2\} \cdot y_0 - x_0 &= c \cdot \dfrac {\sqrt \pi} 2 \cdot \erf (T),
\end{align*}
і виражаємо звідси $c$:
\[ c = 2 \cdot \dfrac{\exp\{-T^2\} \cdot y_0 - x_0} {\sqrt \pi \cdot \erf (T)}, \]
де $\erf$ позначає функцію помилок, тобто $\erf(x) = \dfrac 2 {\sqrt \pi} \cdot \int_0^x \exp\{-t^2\} dt$.\\

Зауважимо, що задача має розв'язок завжди.
\item Нескладно зрозуміти, що нас задовольнить довільне керування вигляду
\[ c_2 = 2 \cdot \dfrac{\exp\{-T^2\} \cdot y_0 - \exp\{-t_1^2\} \cdot x_1} {\sqrt \pi \cdot (\erf (T) - \erf (t_1))},\] де \[ x_1 = \dfrac{2x_0 + c_1\sqrt \pi \cdot \erf (T)}{2\cdot \exp\{-t_1^2\}}, \]
тобто  ми просто дозволили $c_1$ бути довільною сталою, обчислили $x(t_1)$, а потім розв'язали задачу переведення системи з точки $(t_1, x_1)$ у точку $(T, y_0)$ як у першому пункті, з мінімальними поправками на межі інтегрування. \\

Зокрема, якщо $c_1 = 0$, то $x_1 = \dfrac {x_0} {\exp\{-t_1^2\}}$, тому $c_2 = 2 \cdot \dfrac{\exp\{-T^2\} \cdot y_0 - x_0} {\sqrt \pi \cdot (\erf (T) - \erf (t_1))}$.\\

Зауважимо, що задача має розв'язок завжди.
\item 
\begin{align*}
    \dfrac{dx}{2x(t) + c} &= t dt \\
    \\
    \int_0^T \dfrac{dx}{2x(t) + c} &= \int_0^T t dt \\
    \\
    \left.\left(\dfrac 12 \ln(2x(t) + c)\right)\right|_0^T &= \dfrac {T^2} 2 \\
    \\
    \ln(2y_0 + c) - \ln(2x_0 + c) &= T^2 \\
    \\
    \ln\left(\dfrac{2y_0 + c}{2x_0 + c}\right) &= T^2 \\
    \\
    \dfrac{2y_0 + c}{2x_0 + c} &= \exp\{T^2\} \\
    \\
    2y_0 + c &= (2x_0 + c)\cdot \exp\{T^2\} \\
    \\
    2(y_0 - x_0 \cdot \exp\{T^2\}) &= c \cdot (\exp\{T^2\} - 1) 
\end{align*}
звідки
\[c = 2\cdot \dfrac{y_0 - x_0 \cdot \exp\{T^2\}}{\exp\{T^2\} - 1}. \]
Зауважимо, що задача має розв'язок завжди.
\item 
\begin{align*}
    \dfrac{dx}{dt} &= 2t\cdot x(t) + c\cdot x(t) \\
    \\
    \dfrac{dx}{x(t)} &= (2t + c) dt \\
    \\
    \int_0^T \dfrac{dx}{x(t)} &= \int_0^T (2t + c) dt \\
    \\
    (\ln(x(t))|_0^T &= T^2 + cT \\
    \\
    \ln(y_0) - \ln(x_0) &= T^2 + cT \\ 
    \\
    \ln\left(y_0/x_0\right) &= T^2 + cT 
\end{align*}
звідки
\[ c = \dfrac{\ln\left(y_0/x_0\right) - T^2}{T}. \]
Зауважимо, що задача має розв'язок тільки якщо $\signum(x_0) = \signum(y_0)$.
\end{enumerate}
\end{solution} 

\begin{problem}
\begin{enumerate}
    \item Знайти грамміан керованості для системи керування \[\dfrac{dx(t)}{dt} = tx(t) + u(t)\] і дослідити її на керованість, використовуючи перший критерій керованості.
    \item За допомогою грамміана керованості розв'язати таку задачу оптимального керування: \[\mathcal{J}(u) = \int_0^T u^2(s) ds \to \min\]
    за умов, що \[\dfrac{dx(t)}{dt} = tx(t) + u(t), x(0) = x_0, x(T) = x_T. \]
    Тут $x$ -- стан системи. $u(t)$ -- скалярне керування, $x_0$, $x_T$ -- задані точки, $t\in[0,T]$.
\end{enumerate}
\end{problem}

\begin{solution}
\begin{enumerate}
    \item Одразу помітимо, що $A(t) = (t)$, $B(t) = (1)$. Далі, з рівняння $\dfrac{d\Theta(t,s)}{dt} = A(t) \cdot \Theta(t,s)$ знаходимо $\Theta(t,s) = \exp\{t^2 / 2 - s^2 / 2\}$. Залишилося всього нічого, знайти власне грамміан:
    \begin{multline*} \Phi(T,0) = \int_0^T \Theta(T,s)B(s)B^*(s),\Theta^*(T,s) ds = \\ = \int_0^T (\exp\{T^2 - s^2\}) ds = \begin{pmatrix}\dfrac{\sqrt \pi}{2} \cdot \exp\{T^2\} \cdot \erf(T) \end{pmatrix}, \end{multline*} і $\det\Phi(T,0)\ne0$, тобто система цілком керована на $[0, T]$.
    \item Пригадаємо наступний результат: розв'язком вищезгаданої задачі про оптимальне керування є функція
    \begin{align*}
        u(t) &= B^*(t) \Theta^*(T,t)\Phi^{-1}(T,0)(x_T-\Theta(T,0) x_0) = \\
        \\
        &= \exp\{T^2 / 2 - t^2 / 2\} \begin{pmatrix}\dfrac2{\sqrt \pi} \cdot \exp\{-T^2\} \cdot \dfrac1{\erf(T)} \end{pmatrix} (x_T - \exp\{T^2 / 2\} x_0) = \\
        \\
        &= \dfrac{2}{\sqrt{\pi}\cdot \erf(T)} \cdot \left(x_T \cdot \exp\left\{-\dfrac{T^2+t^2}2\right\} - x_0 \cdot \exp\left\{- \dfrac{t^2}2\right\}\right).
    \end{align*} 
\end{enumerate}
\end{solution} 

\begin{problem}
    Мінімізувати критерій якості 
    \[ \mathcal{J}(u) = \int_0^T (u_1^1(s) + u_2^2(s)) ds \]
    за умов \[ \left\{ \begin{aligned} \dfrac{dx_1(t)}{dt} = 6x_1(t) - 2x_2(t) + u_1(t), \\ \dfrac{dx_2(t)}{dt} = 5x_1(t) - x_2(t) + u_2(t), \end{aligned} \right. \]
    \[ x_1(0) = x_{10}, x_2(0) = x_{20}, \] 
    \[ x_1(T) = x_2(T) = 0. \]
    Тут $x = (x_1, x_2)^\star$ -- вектор фазових координат з $\RR^2$, $u=(u_1,u_2)^\star$ -- вектор керування, $x = (x_{10}, x_{20})^\star$ -- відома точка, $t \in [0, T]$.
\end{problem}

\begin{solution}
    Запишемо систему у людському вигляді:
    \[ \begin{pmatrix} \dot x_1 \\ \dot x_2 \end{pmatrix} (t) = \begin{pmatrix} 6 & -2 \\ 5 & -1 \end{pmatrix} \begin{pmatrix} x_1 \\ x_2 \end{pmatrix} (t) + \begin{pmatrix} 1 & 0 \\ 0 & 1 \end{pmatrix} \begin{pmatrix} u_1 \\ u_2 \end{pmatrix} (t) \]
    
    Знайдемо власні числа матриці $A - \lambda E$: $\det(A - \lambda E) = \begin{vmatrix} 6 - \lambda & -2 \\ 5 & -1 - \lambda \end{vmatrix} = \lambda^2 - 5\lambda + 4 = (\lambda - 1) (\lambda - 4) = 0$, звідки $\lambda_1 = 1$, $\lambda_2 = 4$. Знайдемо власні вектори, вони будуть $\begin{pmatrix} 2 \\ 5 \end{pmatrix}$ і $\begin{pmatrix} 1 \\ 1 \end{pmatrix}$ відповідно. Звідси знаходимо загальний розв'язок
    \[ \begin{pmatrix} x_1 \\ x_2 \end{pmatrix} (t) = c_1 \begin{pmatrix} 2e^t \\ 5e^t \end{pmatrix} + c_2 \begin{pmatrix} e^{4t} \\ e^{4t} \end{pmatrix}. \]
    
    З рівняння
    \[ c_1 \begin{pmatrix} 2e^s \\ 5e^s \end{pmatrix} + c_2 \begin{pmatrix} e^{4s} \\ e^{4s} \end{pmatrix} = \begin{pmatrix} 1 \\ 0 \end{pmatrix} \]
    знаходимо $c_1 = -\dfrac13 e^{-s}$, $c_2 = \dfrac53 e^{-4s}$, а з рівняння
    \[ c_1 \begin{pmatrix} 2e^s \\ 5e^s \end{pmatrix} + c_2 \begin{pmatrix} e^{4s} \\ e^{4s} \end{pmatrix} = \begin{pmatrix} 0 \\ 1 \end{pmatrix} \]
    знаходимо $c_1 = \dfrac13 e^{-s}$, $c_2 = -\dfrac23 e^{-4s}$, тобто
    \[ \Theta(t, s) = \dfrac13\begin{pmatrix} -2 e^{t-s} + 5 e^{4(t-s)} & 2 e^{t-s} - 2 e^{4(t-s)} \\ -5 e^{t-s} + 5 e^{4(t-s)} & 5 e^{t-s} - 2 e^{4(t-s)} \end{pmatrix}. \]
    
    Знайдемо грамміан за формулою \[\Phi(T, 0) = \int_0^T \Theta(T, s) B(s) B^* (s) \Theta^*(T, s) ds. \]
    
    \[ \Theta(T, s) B(s) =  \dfrac13\begin{pmatrix} -2 e^{t-s} + 5 e^{4(t-s)} & 2 e^{t-s} - 2 e^{4(t-s)} \\ -5 e^{t-s} + 5 e^{4(t-s)} & 5 e^{t-s} - 2 e^{4(t-s)} \end{pmatrix}. \]
    \[ B^* (s) \Theta^*(T, s)  =  (\Theta(T, s) B(s))^\star =  \dfrac13\begin{pmatrix} -2 e^{t-s} + 5 e^{4(t-s)} & 5 e^{t-s} - 5 e^{4(t-s)} \\ -2 e^{t-s} + 2 e^{4(t-s)} & 5 e^{t-s} - 2 e^{4(t-s)} \end{pmatrix}. \]
    
    Обчислення грамміану є громіздкою обчислювальною задачею, ми залишаємо її як вправу для читача.
\end{solution}

\begin{problem}
    Записати систему диференціальних рівнянь для знаходження першої матриці керованості (грамміана керованості) і сформулювати критерій керованості на інтервалі $[0, T]$ у випадку, якщо система керування має вигляд:
    \[ \dfrac{d^2x(t)}{dt^2} + tx(t) = u(t). \]
    Тут $x$ -- стан системи, $u(t)$ -- скалярне керування, $t \in [0, T]$.
\end{problem}

\begin{solution}
    Зробимо заміну $x_1 = x$, $x_2 = \dot x$, тоді маємо систему \[ \begin{pmatrix} \dot x_1 \\ \dot x_2 \end{pmatrix} (t) = \begin{pmatrix} 0 & 1 \\ -t & 0 \end{pmatrix} \begin{pmatrix} x_1 \\ x_2 \end{pmatrix} (t) + \begin{pmatrix} 0 \\ 1 \end{pmatrix} \begin{pmatrix} u \end{pmatrix} (t). \]
    Звідси можемо записати систему диференціальних рівнянь для знаходження грамміана керованості:
    \[ \dfrac{\Phi(t, t_0)}{dt} = A(t) \Phi(t, t_0) + \Phi(t, t_0) A^\star(t) + B(t) B^*(t). \]
    \[ \dfrac{\Phi(t, 0)}{dt} = \begin{pmatrix} 0 & 1 \\ -t & 0 \end{pmatrix} \Phi(t, 0) + \Phi(t, 0) \begin{pmatrix} 0 & t \\ -1 & 0 \end{pmatrix} + \begin{pmatrix} 0 & 0 \\ 0 & 1 \end{pmatrix}. \]
    Окрім цього, не забуваємо про умову $\Phi(0, 0) = 0$.\\
    
    Щодо критерію керованості, то тут все просто (чи радше стандартно), для того щоб система була цілком керованою на $[0, T]$ необхідно і достатньо, щоб грамміан керованості $\Phi(T, 0)$ був невиродженим, тобто щоб $\det \Phi(T, 0) \ne 0$ або (що те саме у випадку невід'ємно-визначеної матриці) щоб $\det \Phi(T, 0) > 0$.
\end{solution}

\begin{problem}
    Дослідити на керованість, використовуючи другий критерій керованості:
    \begin{enumerate}
        \item \[ \left\{ \begin{aligned} \dfrac{dx_1}{dt} &= -x_1 + x_2 + au, \\ \dfrac{dx_2}{dt} &= x_1 + \dfrac ua; \end{aligned} \right. \]
        \item \[ \left\{ \begin{aligned} \dfrac{dx_1}{dt} &= x_1 - x_2 + au, \\ \dfrac{dx_2}{dt} &= x_1 + \dfrac ua; \end{aligned} \right. \] 
        \item \[ \left\{ \begin{aligned} \dfrac{dx_1}{dt} &= x_1 + x_2 + au, \\ \dfrac{dx_2}{dt} &= - x_1 + x_2 + a^2u; \end{aligned} \right. \]
        \item \[ \left\{ \begin{aligned} \dfrac{dx_1}{dt} &= 2x_1 + x_2 - au, \\ \dfrac{dx_2}{dt} &= - x_1 + au; \end{aligned} \right. \]
        \item \[ x^{(n)}(t) + a_1 x^{(n-1)}(t) + \ldots + a_{n-1}x'(t) + a_n x(t) = u(t). \]
        \item \[ \left\{ \begin{aligned} \dfrac{dx_1}{dt} &= x_1 + 2x_2 - x_3 + u_1 - u_2 \\ \dfrac{dx_2}{dt} &= -x_1 + x_2 + 3x_3 + u_1 \\ \dfrac{dx_3}{dt} &= x_2 + x_3 + 2u_2 \end{aligned} \right. \]
    \end{enumerate}
\end{problem}

\begin{solution}
    \begin{enumerate}
        \item \[ A = \begin{pmatrix} -1 & 1 \\ 1 & 0 \end{pmatrix} \qquad B = \begin{pmatrix} a \\ 1 / a \end{pmatrix} \]
        \[ D = \begin{pmatrix} B & AB \end{pmatrix} = \begin{pmatrix} a & 1 / a - a \\ 1 / a & a \end{pmatrix} \]
        \[ \det D = a^2 + 1 - 1/a^2 \ne 0, \]
        тобто система цілком керована якщо тільки $a \ne \pm \sqrt{\dfrac{\sqrt{5} - 1}{2}}$.
        \item \[ A = \begin{pmatrix} 1 & -1 \\ 1 & 0 \end{pmatrix} \qquad B = \begin{pmatrix} a \\ 1 / a \end{pmatrix} \]
        \[ D = \begin{pmatrix} B & AB \end{pmatrix} = \begin{pmatrix} a & a - 1 / a \\ 1 / a & a \end{pmatrix} \]
        \[ \det D = a^2 - 1 + 1/a^2 \ne 0, \]
        тобто система цілком керована для будь-яких $a$ (навіть $\det D \ge 1$ за нерівністю Коші).
        \item \[ A = \begin{pmatrix} 1 & 1 \\ -1 & 1 \end{pmatrix} \qquad B = \begin{pmatrix} a \\ a^2 \end{pmatrix} \]
        \[ D = \begin{pmatrix} B & AB \end{pmatrix} = \begin{pmatrix} a & a + a^2 \\ a^2 & a^2 - a \end{pmatrix} \]
        \[ \det D = a^3 - a^2 - a^4 - a^3 = -a^4 - a^2 \ne 0, \]
        тобто система цілком керована якщо тільки $a \ne 0$.
        \item \[ A = \begin{pmatrix} 2 & 1 \\ -1 & 0 \end{pmatrix} \qquad B = \begin{pmatrix} -a \\ a \end{pmatrix} \]
        \[ D = \begin{pmatrix} B & AB \end{pmatrix} = \begin{pmatrix} -a & -a \\ a & a \end{pmatrix} \]
        \[ \det D = 0, \]
        тобто система не є цілком керованою для будь-яких $a$.
        \item Зробимо заміну $x_0 = x$, $x_1 = x'$, $\ldots$, $x_n = x^{(n)}$, тоді отримаємо систему
        \[ \left\{ \begin{aligned} \dot x_0 &= x_1 \\ \dot x_1 &= x_2 \\ \cdots \\ \dot x_{n-1} &= x_n \\ \dot x_n &= u - a_n x_0 - a_{n-1} x_1 - \ldots - a_1 x_{n-1} \end{aligned} \right. \]
        тобто
        \[ A = \begin{pmatrix} 0 & 1 & \ddots & 0 & 0 \\ 0 & 0 & \ddots & \ddots & 0 \\ \vdots & \vdots & \ddots & \ddots & \ddots \\ 0 & 0 & \cdots & 0 & 1 \\ -a_n & -a_{n-1} & \cdots & -a_1 & 0 \end{pmatrix} \qquad B = \begin{pmatrix} 0 \\ 0 \\ \vdots \\ 0 \\ 1 \end{pmatrix} \]
        \[ D = \begin{pmatrix} B & AB & A^2B & \cdots & A^nB \end{pmatrix} = \begin{pmatrix} 0 & \cdots & 0 & 0 & 1 \\ \vdots & \reflectbox{$\ddots$} & \reflectbox{$\ddots$} & \reflectbox{$\ddots$} & \cdots \\ 0 & 0 & 1 & 0 & \cdots \\ 0 & 1 & 0 & -a_1 & \cdots \\ 1 & 0 & -a_1 & -a_2 & \cdots \end{pmatrix} \]
        \[ \det D = -1 \ne 0, \] тобто система цілком керована для довільних $a_1$, $a_2$, $\ldots$, $a_n$.
        \item \[ A = \begin{pmatrix} 1 & 2 & -1 \\ -1 & 1 & 3 \\ 0 & 1 & 1 \end{pmatrix} \qquad B = \begin{pmatrix} 1 & -1 \\ 1 & 0 \\ 0 & 2 \end{pmatrix} \]
        \[ D = \begin{pmatrix} B & AB & A^2B  \end{pmatrix} = \begin{pmatrix} 1 & -1 & 3 & -3 & 2 & 8 \\ 1 & 0 & 0 & 7 & 0 & 16 \\ 0 & 2 & 1 & 2 & 1 & 9 \end{pmatrix} \]
        Її ранг дорівнює 3, тобто система цілком керована.
    \end{enumerate}
\end{solution}

 \newpage
\subsection{Задачі для самостійного розв'язування}

\begin{problem}
    Знайти диференціальне рівняння грамміана керованості для системи керування \[ \left\{ \begin{aligned}
        \frac{\diff x_1 (t)}{\diff t} &= \cos(t) \cdot x_1(t)-\sin(t) \cdot x_2(t) + u_1(t) - 2 u_2(t), \\
        \frac{\diff x_2 (t)}{\diff t} &= \sin(t) \cdot x_1(t)+\cos(t) \cdot x_2(t) - 3 u_1(t) + 4 u_2(t).
    \end{aligned} \right. \]
\end{problem}

\begin{problem}
    За допомогою грамміана керованості розв'язати таку задачу оптимального керування: мінімізувати критерій якості \[ \JJ (u) = \int_0^T u^2(s) \diff s \] за умов, що \[ \frac{\diff^2 x(t)}{\diff t^2} - 5 \frac{\diff x(t)}{\diff t} + 6 x(t) = u(t), \] \[ x(0) = x_0, x'(0) = y_0,x(T)=x'(T)=0.\] Тут $x$ -- стан системи, $u(t)$ -- скалярне керування, $t\in[0,T]$.
\end{problem}

\begin{problem}
    Записати систему диференціальних рівнянь для знаходження першої матриці керованості (грамміана керованості) і сформулювати критерій керованості на інтервалі $[0,T]$ у випадку, якщо системи керування має вигляд \[ \left\{ \begin{aligned} \frac{\diff x_1(t)}{\diff t} = tx_1(t) + t^2x_2(t) + u_1(t)-u_2(t), \\ \frac{\diff x_2(t)}{\diff t} = -x_1(t) + x_2(t) + 2u_2(t), \end{aligned} \right. \] Тут $x = (x_1, x_2)^\star$ -- вектор фазових координат з $\RR^2$, $u=(u_1,u_2)^\star$ -- вектор керування, $t \in [0, T]$.
\end{problem}

 \newpage

\section{Критерії спостережуваності. Критерій двоїстості}

\subsection{Алгоритми}

\begin{problem*}
    Побудувати систему для знаходження грамміана спостережуваності для системи $\dot x = A x$, $y = H x$.
\end{problem*}

\begin{algorithm} \tt
    Записуємо систему \[ \dot \NN(t, t_0) = -A(t) \cdot \NN(t, t_0) - \NN(t, t_0) \cdot A^*(t) + H^*(t) \cdot H(t). \]
\end{algorithm}

\begin{problem*}
    Чи буде стаціонарна система $\dot x = A x + B u$ цілком спостережуваною?
\end{problem*}

\begin{algorithm} \tt 
    \begin{enumerate}
        \item Знаходимо \[ \mathcal{R} = \left(H^* \vdots A^* H^* \vdots (A^*)^2 H^* \vdots \ldots \vdots (A^*)^{n-1} H^*\right).\] 
        \item Якщо $rang \mathcal{R} = n$ то система цілком спостережувана інакше ні.
    \end{enumerate}
\end{algorithm}

\begin{problem*}
    Дослідити на спостережуваність систему $\dot x = A x$, $y = H x$, використовуючи критерій двоїстості і відповідний критерій керованості.
\end{problem*}

\begin{algorithm} \tt
    \begin{enumerate}
        \item Будуємо спряжену систему \[ \frac{\diff z(t)}{\diff t} = - A^*(t) \cdot z(t) + H^*(t) \cdot u(t). \]
        \item Досліджуємо її на керованість, якщо вона керована, то по\-чат\-ко\-ва \allowbreak сис\-те\-ма спостережувана, інакше ні.
    \end{enumerate}
\end{algorithm}

\begin{problem*}
    Побудувати спостерігач для системи $\dot x = A x$, $y = H x$.
\end{problem*}

\begin{algorithm} \tt
    \begin{enumerate}
        \item Записуємо систему (спостерігач) \[ \dot{\hat{x}} (t) = (A (t) - K (t) H (t)) \cdot \hat x (t) + K(t) \cdot y(t).\]
        \item Або систему (спостерігач) \[ \dot{\hat{x}} (t) = A (t) \cdot \hat x (t) + K(t) \cdot (y(t) - H(t) \cdot \hat x(t)).\]
    \end{enumerate}
    Вибір вільний, це різні форми запису одного і того ж.
\end{algorithm}

\begin{problem*}
    Задана динамічна система $\dot x = A x$, $y = H x$. Знайти розв'язок задачі спостереження з використання грамміана спостержуваності.
\end{problem*}

\begin{algorithm} \tt
    \begin{enumerate}
        \item Знаходимо грамміан спостережуваності $\NN(t,t_0)$ з сис\-те\-ми \[ \dot \NN(t, t_0) = -A(t) \cdot \NN(t, t_0) - \NN(t, t_0) \cdot A^*(t) + H^*(t) \cdot H(t). \]

        \item Знаходимо $R(t) = \NN^{-1}(t, t_0)$.

        \item Розв'язуємо рівняння \[ \dot x(t) = A(t) \cdot x(t) + R(t) \cdot H^*(t) \cdot (y(t) - H(t) \cdot x(t)). \]
    \end{enumerate}
\end{algorithm}

 \newpage
\subsection{Аудиторне заняття}

\begin{problem}
	Побудувати систему для знаходження грамміана спостережуваності і записати умову спостережуваності на інтервалі для системи: \[
	\left\{
		\begin{aligned}
			\dot x_1 &= \cos(t) \cdot x_1 + \sin(t) \cdot x_2, \\
			\dot x_2 &= - \sin(t) \cdot x_1 + \cos(t) \cdot x_2, \\
			y(t) &= k x_1 + x_2,
		\end{aligned}
	\right.
	\]
	$k > 0$.
\end{problem}

\begin{solution}
	Диференціальне рівняння для знаходження грамміана спостережуваності має вигляд \[ \frac{\diff \NN(t, t_0)}{\diff t} = -A(t) \cdot \NN(t, t_0) - \NN(t, t_0) \cdot A^*(t) + H^*(t) \cdot H(t), \] а після підстановки відомих значень вона набує вигляду \begin{multline*} 
		\frac{\diff \NN(t, t_0)}{\diff t} = -\begin{pmatrix} \cos (t) & \sin(t) \\ - \sin(t) & \cos(t) \end{pmatrix} \cdot \NN(t, t_0) - \\
		- \NN(t, t_0) \cdot \begin{pmatrix} \cos (t) & -\sin(t) \\ \sin(t) & \cos(t) \end{pmatrix} + \begin{pmatrix} k \\ 1 \end{pmatrix} \cdot \begin{pmatrix} k & 1 \end{pmatrix}, 
	\end{multline*} або, у розгорнутому вигляді: \[
	\left\{
		\begin{aligned}
			\dot n_{11} (t) &= - 2 \cos(t) \cdot n_{11} (t) - 2 \sin(t) \cdot n_{12} (t) + k^2, \\
			\dot n_{12} (t) &= \sin (t) \cdot n_{11} (t) - 2\cos(t) \cdot n_{12} (t) - \sin (t) \cdot n_{22} (t) + k, \\
			\dot n_{22} (t) &= 2 \sin(t) \cdot n_{12} (t) - 2 \cos(t) \cdot n_{22} (t) + 1.
		\end{aligned}
	\right.
	\]

	Умова спостережуваності цієї системи на $[t_0, T]$ має вигляд  $n_{11} (t) \cdot n_{22} (t) - n_{12}^2 (t) \ne 0$, $t \in [t_0, T]$.
\end{solution}

\begin{problem}
    Чи буде система цілком спостережуваною?

    \begin{enumerate}
    	\item \[ \ddot x = a^2 x, y(t) = x(t); \]

    	\item \[ \left \{ \begin{aligned}
    		\dot x_1 &= x_1 + \alpha x_2, \\
    		\dot x_2 &= \alpha x_1 + x_2, \\
    		y(t) &= \beta x_1 (t) + x_2 (t).
    	\end{aligned} \right. \]

    	\item \[ \left \{ \begin{aligned}
    		\dot x_1 &= a x_1, \\
    		\dot x_2 &= b x_2, \\
    		y(t) &= x_1 (t) + x_2 (t).
    	\end{aligned} \right. \]
    \end{enumerate}
\end{problem}

\begin{solution}
    Всі системи є стаціонарними, тому будемо застосовувати другий критерій спостережуваності: \[ rang \mathcal{R} = rang \left(H^* \vdots A^* H^* \vdots (A^*)^2 H^* \vdots \ldots \vdots (A^*)^{n-1} H^*\right) = n. \]

    \begin{enumerate}
    	\item Введемо нову змінну $x_2 = \dot x_1$, тоді $\dot x_2 = a^2 x_1$, $y = x_1$, тому \[ A = \begin{pmatrix} 0 & 1 \\ a^2 & 0 \end{pmatrix}, \quad H = \begin{pmatrix} 1 & 0 \end{pmatrix}. \] Підставляючи у критерій, знаходимо: \[ \mathcal{R} = \left(\begin{pmatrix} 1 \\ 0 \end{pmatrix} \vdots \begin{pmatrix} 0 & a^2 \\ 1 & 0 \end{pmatrix} \begin{pmatrix} 1 \\ 0 \end{pmatrix}\right) = \begin{pmatrix} 1 & 0 \\ 0 & 1 \end{pmatrix}, \] її ранг 2, тому система цілком спостережувана.

    	\item \[ A = \begin{pmatrix} 1 & \alpha \\ \alpha & 1 \end{pmatrix}, \quad H = \begin{pmatrix} \beta & 1 \end{pmatrix}. \] Підставляючи у критерій, знаходимо: \[ \mathcal{R} = \left(\begin{pmatrix} \beta \\ 1 \end{pmatrix} \vdots \begin{pmatrix} 1 & \alpha \\ \alpha & 1 \end{pmatrix} \begin{pmatrix} \beta \\ 1 \end{pmatrix}\right) = \begin{pmatrix} \beta & \alpha + \beta \\ 1 & \alpha \beta + 1 \end{pmatrix}, \] її ранг 2 тоді і тільки тоді, коли її визначник $det \mathcal{R} = \alpha (\beta^2 - 1) \ne 0$, тому система цілком спостережувана тоді і тільки тоді, коли $\alpha \ne 0$, $\beta \ne \pm 1$.

    	\item \[ A = \begin{pmatrix} a & 0 \\ 0 & b \end{pmatrix}, \quad H = \begin{pmatrix} 1 & 1 \end{pmatrix}. \] Підставляючи у критерій, знаходимо: \[ \mathcal{R} = \left(\begin{pmatrix} 1 \\ 1 \end{pmatrix} \vdots \begin{pmatrix} a & 0 \\ 0 & b \end{pmatrix} \begin{pmatrix} 1 \\ 1 \end{pmatrix}\right) = \begin{pmatrix} 1 & a \\ 1 & b \end{pmatrix}, \] її ранг 2 тоді і тільки тоді, коли її визначник $det \mathcal{R} = b - a \ne 0$, тому система цілком спостережувана тоді і тільки тоді, коли $a \ne b$.
    \end{enumerate}
\end{solution}

\begin{problem}
    Дослідити на спостережуваність, використовуючи критерій двоїстості і відповідний критерій керованості: \[ \left\{ \begin{aligned}
    	\dot x_1 &= x_2 - 2 x_3, \\
    	\dot x_2 &= x_1 - x_3, \\
    	\dot x_3 &= - 2 x_3, \\
    	y(t) &= -x_1 + x_2 - x_3.
    \end{aligned} \right. \]
\end{problem}

\begin{solution}
	За принципом двоїстості Калмана, ця система є цілком спостережуваною на $[t_0, T]$ тоді і тільки тоді, коли система \[ \frac{\diff z(t)}{\diff t} = - A^*(t) \cdot z(t) + H^*(t) \cdot u(t) \] є цілком керованою на $[t_0, T]$. \\

	Підставляючи відомі значення, отримуємо систему \[ \frac{\diff z(t)}{\diff t} = \begin{pmatrix} 0 & -1 & 0 \\ -1 & 0 & 0 \\ 2 & 1 & 2 \end{pmatrix} \cdot z(t) + \begin{pmatrix} -1 \\ 1 \\ -1 \end{pmatrix} \cdot u(t), \] або, у розгорнутому вигляді \[ \left\{ \begin{aligned}
		\dot z_1 &= - z_2 - u, \\
		\dot z_2 &= - z_1 + u, \\
		\dot z_3 &= 2 z_1 + z_2 + 2 z_3 - u.
	\end{aligned} \right. \]

	Система стаціонарна, тому використаємо другий критерій керованості: \[ rang \mathcal{D} = rang \left(B \vdots AB \vdots A^2 B \vdots \ldots \vdots A^{n-1} B\right) = n. \]

	\[ A = \begin{pmatrix} 0 & -1 & 0 \\ -1 & 0 & 0 \\ 2 & 1 & 2 \end{pmatrix}, \quad B = \begin{pmatrix} -1 \\ 1 \\ -1 \end{pmatrix}. \] Підставляючи у критерій, знаходимо: \begin{multline*} \mathcal{D} = \left(\begin{pmatrix} -1 \\ 1 \\ -1 \end{pmatrix} \vdots \begin{pmatrix} 0 & -1 & 0 \\ -1 & 0 & 0 \\ 2 & 1 & 2 \end{pmatrix} \begin{pmatrix} -1 \\ 1 \\ -1 \end{pmatrix} \vdots \begin{pmatrix} 0 & -1 & 0 \\ -1 & 0 & 0 \\ 2 & 1 & 2 \end{pmatrix}^2 \begin{pmatrix} -1 \\ 1 \\ -1 \end{pmatrix} \right) = \\ 
	= \left(\begin{pmatrix} -1 \\ 1 \\ -1 \end{pmatrix} \vdots \begin{pmatrix} -1 \\ 1 \\ -3 \end{pmatrix} \vdots \begin{pmatrix} 0 & -1 & 0 \\ -1 & 0 & 0 \\ 2 & 1 & 2 \end{pmatrix} \begin{pmatrix} -1 \\ 1 \\ -3 \end{pmatrix} \right) = \begin{pmatrix} -1 & -1 & -1 \\ 1 & 1 & 1 \\ -1 & -3 & -7 \end{pmatrix}, \end{multline*} її ранг 2, тому система не цілком керована, а початкова -- не цілком спостережувана.
\end{solution}

\begin{problem}
    Побудувати спостерігач такої системи у загальному вигляді:
    \begin{enumerate}
    	\item \[ \left\{ \begin{aligned}
    		\dot x_1 &= t x_1 + x_2, \\
    		\dot x_2 &= x_1 - x_2, \\
    		y(t) &= x_1(t) + b x_2(t).
    	\end{aligned} \right. \]

    	\item \[ \left\{ \begin{aligned}
    		\frac{\diff^2 x(t)}{\diff t^2} &= -kx, \\
    		y(t) &= x(t) + \beta \frac{\diff x(t)}{\diff t}.
    	\end{aligned} \right. \]
    \end{enumerate}
\end{problem}

\begin{solution}
    \begin{enumerate}
    	\item За теоремою про структуру спостерігача, він має вигляд \[ \frac{\diff \hat x (t)}{\diff t} = (A (t) - K (t) H (t)) \cdot \hat x (t) + K(t) \cdot y(t).\] Підставляючи відомі значення, знаходимо \[ \frac{\diff \hat x (t)}{\diff t} = \left(\begin{pmatrix} t & 1 \\ 1 & -1 \end{pmatrix} - \begin{pmatrix} k_1(t) \\ k_2(t) \end{pmatrix} \begin{pmatrix} 1 & b \end{pmatrix} \right) \cdot \hat x (t) + \begin{pmatrix} k_1(t) \\ k_2(t) \end{pmatrix} \cdot y(t),\] або, у розгорнутому вигляді \[ \left\{ \begin{aligned}
    		\dot{\hat{x}}_1 &= (t - k_1(t)) \cdot \hat x_1 + (1 - b k_1(t)) \cdot \hat x_2 + k_1(t) \cdot y(t), \\
    		\dot{\hat{x}}_2 &= (1 - k_2(t)) \cdot \hat x_1 - (1 + b k_2(t)) \cdot \hat x_2 + k_2 (t) \cdot y(t).
    	\end{aligned} \right. \]
    	\item Введемо нову змінну $x_2 = \dot x_1$, тоді $\dot x_1 = x_2$, $\dot x_2 = - k x_1$, $y = x_1 + \beta x_2$. \\

    	За теоремою про структуру спостерігача, він має вигляд \[ \frac{\diff \hat x (t)}{\diff t} = A (t) \cdot \hat x (t) + K(t) \cdot (y(t) - H(t) \cdot \hat x(t)).\] Підставляючи відомі значення, знаходимо \[ \frac{\diff \hat x (t)}{\diff t} = \begin{pmatrix} 0 & 1 \\ -k & 0 \end{pmatrix} \hat x(t) + \begin{pmatrix} k_1(t) \\ k_2(t) \end{pmatrix} \cdot \left(y(t) - \hat x_1 - \beta \hat x_2\right), \] або, у розгорнутому вигляді \[ \left\{ \begin{aligned}
    		\dot{\hat{x}}_1 &= \hat x_2 + k_1(t) \cdot \left(y(t) - \hat x_1 - \beta \hat x_2\right), \\
    		\dot{\hat{x}}_2 &= -k \hat x_1 + k_2(t) \cdot \left(y(t) - \hat x_1 - \beta \hat x_2\right).
    	\end{aligned} \right. \]
    \end{enumerate}
\end{solution}

\begin{problem}
    Задана динамічна система \[ \left\{ \begin{aligned} 
    	\frac{\diff x(t)}{\diff t} &= 2 x(t), \\
    	y(t) &= \sin (t) \cdot x(t),
    \end{aligned} \right. \]
    де $x(t) \in \RR^1$ -- вектор стану, $y(t) \in \RR^1$ -- відомі спостереження, $t \in [0, 3]$. Знайти розв'язок задачі спостереження з використанням грамміана спростережуваності.
\end{problem}

\begin{solution}
    Розв'язок задачі спостереження задовольняє диференціальному рівнянню \[ \frac{\diff x(t)}{\diff t} = A(t) \cdot x(t) + R(t) \cdot H^*(t) \cdot (y(t) - H(t) \cdot x(t)), \] де $R(t) = \NN^{-1](t, 0)}$. \\

    Знайдемо $\NN(t, 0)$: з рівняння \[ \frac{\diff \Theta(t, s)}{\diff t} = A(t)\cdot \Theta(t, s) = 2\Theta(t, s) \] знаходимо $\Theta(t, s) = e^{2 (t - s)}$, тому \begin{multline*} \NN(t, 0) = \int_0^t e^{4 (s - t)} \cdot \sin^2(s) \diff s = e^{-4t} \int_0^t e^{4 s} \cdot \sin^2(s) \diff s = \\
    = \frac{-2 \sin (2t) - 4 \cos(2t) + 5 - e^{-4t}}{40}. \end{multline*} Звідси \[ \frac{\diff x(t)}{\diff t} = A(t) \cdot x(t) + \frac{40 \cdot H^*(t) \cdot (y(t) - H(t) \cdot x(t))}{-2 \sin (2t) - 4 \cos(2t) + 5) - e^{-4t}} , \] або, у розгорнутому вигляді, \[ \dot x = 2 x + \frac{40 \cdot \sin(t) \cdot (y(t) - \sin(t) \cdot x(t))}{-2 \sin (2t) - 4 \cos(2t) + 5 - e^{-4t}}. \]
\end{solution}

\begin{problem}
    Задана динамічна система \[ \ddot x = x, y(t) = x(t),\] де $x(t) \in \RR^1$ -- вектор стану, $y(t) \in \RR^1$ -- відомі спостереження, $t \in [0, T]$. Знайти розв'язок задачі спостереження з використанням грамміана спростережуваності.
\end{problem}

\begin{solution}
    % 4.6
\end{solution}
 \newpage
\input{04-NoSolutions.tex} \newpage

\section{Задача фільтрації. Множинний підхід}

\input{05-Algorithms.tex} \newpage
\subsection{Аудиторне заняття}

\begin{problem}
	Задана динамічна система \[ \left\{ \begin{aligned}
		\frac{\diff x(t)}{\diff t} &= t x(t) + v(t), \\
		y(t) &= p x(t) + w(t),
	\end{aligned} \right. \]
	де $x(t) \in \RR^1$ -- вектор стану, $v(t) \in \RR^1$, $w(t) \in \RR^1$ -- невідомі шуми, $x_0 \in \RR^1$ -- невідома початкова умова, $y(t) \in \RR^1$ -- відомі спостереження. Побудувати інформаційну множину такої системи в момент $\tau \in [0, T]$ за умови, що \[ \int_0^\tau (v^2(s) + w^2(s)) \diff s + x^2(0) \le 1, \quad \tau \in [0, T]. \]
\end{problem}

\begin{solution}
	Загальна постановка задачі фільтрації має вигляд \[ \dot x (t)= A (t) \cdot x (t)+ v (t), \quad y (t)= G (t) \cdot x(t) + w(t),\] \[\int_{t_0}^t ( \langle M(t) \cdot v(t), v(t)\rangle + \langle N(t) \cdot w(t), w(t)\rangle )  + \langle p_0 x(t_0), x(t_0) \rangle \le \mu^2. \] У нашій задачі $A (t)= \begin{pmatrix} t \end{pmatrix}$, $G (t)= \begin{pmatrix} p \end{pmatrix}$, $M (t)= \begin{pmatrix} 1 \end{pmatrix}$, $N (t)= \begin{pmatrix} 1 \end{pmatrix}$, $p_0 = 1$, $\mu = 1$. \\

	Знайдемо фільтр (спостерігач) цієї задачі у вигляді \[ \dot{\hat{x}} (t) = A (t) \cdot \hat x (t) + K (t) \cdot (y (t) - G (t) \cdot \hat x (t)), \] де $K (t)= R (t) \cdot G^* (t) \cdot N(t)$, де у свою чергу \[\dot R (t)= A (t) \cdot R (t)+ R (t) \cdot A^* (t)- R (t) \cdot G^* (t) \cdot N (t) \cdot G (t) \cdot R(t), \quad R(t_0) = p_0^{-1}. \]

 	Підставляючи відомі функції знаходимо \[\dot R (t)= 2 t \cdot R (t) - p^2 \cdot R^2(t), \quad R(t_0) = 1. \]

 	Це рівняння Бернуллі, його розв'язок \[ R(t) = \frac{2 e^{t^2}}{2 e^{t_0^2} + p^2 \sqrt{\pi} (\erfi(t) - \erfi(t_0))}. \]

 	Далі, \[ K(t) = \frac{2 p e^{t^2}}{2 e^{t_0^2} + p^2 \sqrt{\pi} (\erfi(t) - \erfi(t_0))}, \] і \[ \dot{\hat{x}} (t) = t \cdot \hat x (t) + \frac{2 p e^{t^2} \cdot (y (t) - p \cdot \hat x (t))}{2 e^{t_0^2} + p^2 \sqrt{\pi} (\erfi(t) - \erfi(t_0))}. \]

 	Нарешті, \[ \XX(\tau) = \EE (\hat x(\tau), (\mu^2 - k(\tau)) \cdot R(\tau)), \] де \[ \dot k (s) = \langle N(s) (y(s) - G(s) \cdot \hat x(s)), y(s) - G(s) \cdot \hat x(s)\rangle, \quad k(t_0) = 0, \] тобто \[ \dot k (s) = \langle (y(s) - p \hat x(s)), y(s) - p \hat x(s)\rangle = |y(s) - p \hat x(s)|^2, \quad k(t_0) = 0. \]
\end{solution}

\begin{problem}
	Задана динамічна система \[ \left\{ \begin{aligned}
		\frac{\diff x(t)}{\diff t} &= x(t) + v(t), \\
		y(t) &= 2 x(t) + w(t),
	\end{aligned} \right. \]
	де $x(t) \in \RR^1$ -- вектор стану, $v(t) \in \RR^1$, $w(t) \in \RR^1$ -- невідомі шуми, $x_0 \in \RR^1$ -- невідома початкова умова. Побудувати оцінку стану заданої системи (фільтр) за заданими спостереженнями $y(t) \in \RR^1$ за умови, що \[ \int_0^\tau (v^2(s) + w^2(s)) \diff s + x^2(0) \le 2, \] $\tau \in [0, T]$. Знайти похибку оцінювання.
\end{problem}

\begin{solution}
	Загальна постановка задачі фільтрації має вигляд \[ \dot x (t)= A (t) \cdot x (t)+ v (t), \quad y (t)= G (t) \cdot x(t) + w(t),\] \[\int_{t_0}^t ( \langle M(t) \cdot v(t), v(t)\rangle + \langle N(t) \cdot w(t), w(t)\rangle )  + \langle p_0 x(t_0), x(t_0) \rangle \le \mu^2. \] У нашій задачі $A (t)= \begin{pmatrix} 1 \end{pmatrix}$, $G (t)= \begin{pmatrix} 2 \end{pmatrix}$, $M (t)= \begin{pmatrix} 1 \end{pmatrix}$, $N (t)= \begin{pmatrix} 1 \end{pmatrix}$, $p_0 = 1$, $\mu = \sqrt{2}$. \\

	Знайдемо фільтр (спостерігач) цієї задачі у вигляді \[ \dot{\hat{x}} (t) = A (t) \cdot \hat x (t) + K (t) \cdot (y (t) - G (t) \cdot \hat x (t)), \] де $K (t)= R (t) \cdot G^* (t) \cdot N(t)$, де у свою чергу \[\dot R (t)= A (t) \cdot R (t)+ R (t) \cdot A^* (t)- R (t) \cdot G^* (t) \cdot N (t) \cdot G (t) \cdot R(t), \quad R(t_0) = p_0^{-1}. \]

 	Підставляючи відомі функції знаходимо \[\dot R (t)= 2 \cdot R (t) - 4 \cdot R^2(t), \quad R(t_0) = 1. \]

 	Це рівняння зі змінними що роздяліються, його розв'язок \[ R(t) = \frac{e^{2t}}{2e^{2t}-e^{2t_0}}. \]

 	Далі, \[ K(t) = \frac{2 e^{2t}}{2e^{2t}-e^{2t_0}}, \] і \[ \dot{\hat{x}} (t) = \hat x (t) + \frac{2 e^{2t} \cdot (y (t) - 2 \cdot \hat x (t))}{2e^{2t}-e^{2t_0}}. \]

 	Похибка $e(\tau)$ оцінювання задовольняє оцінці \[ |e(\tau)| \le \sqrt{\mu^2-k(\tau)} \cdot \sqrt{\lambda_* (R(\tau))} = \frac{ \sqrt{2 - k(\tau)} \cdot e^{2\tau}}{2e^{2\tau}-e^{2t_0}},\] де \[ \dot k (s) = \langle N(s) (y(s) - G(s) \cdot \hat x(s)), y(s) - G(s) \cdot \hat x(s)\rangle, \quad k(t_0) = 0, \] тобто \[ \dot k (s) = \langle (y(s) - 2 \hat x(s)), y(s) - 2 \hat x(s)\rangle = |y(s) - 2 \hat x(s)|^2, \quad k(t_0) = 0. \]
\end{solution}

\begin{problem}
	Задана динамічна система \[ \left\{ \begin{aligned}
		\frac{\diff x_1(t)}{\diff t} &= 2 x_1(t) + x_2(t) + v_1(t), \\
		\frac{\diff x_2(t)}{\diff t} &= - x_1(t) + x_2(t) + v_2(t), \\
		y(t) &= x_1(t) + 2 x_2(t) + w(t),
	\end{aligned} \right. \] і відомі спостереження за цією системою $y(t) \in \RR^1$.  Побудувати оцінку стану (фільтр) і знайти похибку оцінювання. Тут $x = (x_1, x_2)^*$ -- вектор фазових координат з $\RR^2$, $v_1 (t) \in \RR^1$, $v_2 (t) \in \RR^1$, $w(t) \in \RR^1$ -- невідомі шуми, \[ \int_0^\tau (v_1^2(s) + v_2^2(s) + w^2(s)) \diff s + x_1^2(0) + 2 x_2^2(0) \le 1,\] $\tau \in [0, T]$, момент часу $T$ є заданим.
\end{problem}

\begin{solution}
	Загальна постановка задачі фільтрації має вигляд \[ \dot x (t)= A (t) \cdot x (t)+ v (t), \quad y (t)= G (t) \cdot x(t) + w(t),\] \[\int_{t_0}^t ( \langle M(t) \cdot v(t), v(t)\rangle + \langle N(t) \cdot w(t), w(t)\rangle )  + \langle p_0 x(t_0), x(t_0) \rangle \le \mu^2. \] У нашій задачі $A (t) = \begin{pmatrix} 2 & 1 \\ -1 & 1 \end{pmatrix}$, $G (t)= \begin{pmatrix} 1 & 2 \end{pmatrix}$, $M (t)= \begin{pmatrix} 1 & 0 \\ 0 & 1 \end{pmatrix}$, $N (t)= \begin{pmatrix} 1 \end{pmatrix}$, $p_0 = \begin{pmatrix} 1 & 0 \\ 0 & 2 \end{pmatrix}$, $\mu = 1$. \\

	Знайдемо фільтр (спостерігач) цієї задачі у вигляді \[ \dot{\hat{x}} (t) = A (t) \cdot \hat x (t) + K (t) \cdot (y (t) - G (t) \cdot \hat x (t)), \] де $K (t)= R (t) \cdot G^* (t) \cdot N(t)$, де у свою чергу \[\dot R (t)= A (t) \cdot R (t)+ R (t) \cdot A^* (t)- R (t) \cdot G^* (t) \cdot N (t) \cdot G (t) \cdot R(t), \quad R(t_0) = p_0^{-1}. \]

 	Підставляючи відомі функції знаходимо \[\dot R (t) = \begin{pmatrix} 2 & 1 \\ -1 & 1 \end{pmatrix} \cdot R(t) + R(t) \cdot \begin{pmatrix} 2 & -1 \\ 1 & 1 \end{pmatrix} - R(t) \cdot \begin{pmatrix} 1 & 2 \\ 2 & 4 \end{pmatrix} \cdot R(t), \] або, у розгорнутому вигляді \[ \left\{ \begin{aligned}
 		\dot r_{11} &= 4 r_{11} + 2 r_{12} + r_{11}^2 + 4 r_{11}r_{12} + 4 r_{12}^2, \\
 		\dot r_{12} &=  - r_{11} + 3 r_{12} + r_{22} + r_{11}r_{12}+2r_{12}^2+2r_{11}r_{22}+4r_{12}r_{22}, \\
 		\dot r_{22} &= - 2 r_{12} + 2 r_{22} + r_{12}^2 + 4 r_{12} r_{22} + r_{22}^2, \\
 		r_{11} (0) &= 1, \\
 		r_{12} (0) &= 0, \\
 		r_{22} (0) &= 1 / 2.
	\end{aligned} \right. \] 	

	Похибка $e(\tau)$ оцінювання задовольняє оцінці \[ |e(\tau)| \le \sqrt{\mu^2-k(\tau)} \cdot \sqrt{\lambda_* (R(\tau))},\] де \[ \dot k (s) = \langle N(s) (y(s) - G(s) \cdot \hat x(s)), y(s) - G(s) \cdot \hat x(s)\rangle, \quad k(t_0) = 0, \] тобто \[ \dot k (s) = \langle (y(s) - \hat x_1(s) - 2 \hat x_2(s)), y(s) - \hat x_1(s) - 2 \hat x_2(s)\rangle = |y(s) - \hat x_1(s) - 2 \hat x_2(s)|^2. \]

\end{solution}
 \newpage
\input{05-NoSolutions.tex} \newpage

\section{Варіаційний метод в задачі оптимального керування}

\subsection{Алгоритми}

\begin{problem*}
	Знайти першу варіацію за Лагранжем і похідну Фреше в просторі інтегрованих з квадратом функцій для функціоналу $\JJ = \int f(u) \diff s$.
\end{problem*}

\begin{algorithm} \tt
	\begin{enumerate}
		\item Записуємо $\JJ(u + \alpha h)$.
		\item Знаходимо $\frac{\diff}{\diff \alpha} \JJ(u + \alpha h)$.
		\item Знаходимо першу варіацію $\delta \JJ (u, h)$ за Лагранжем за формулою \[ \delta \JJ(u, h) = \frac{\diff}{\diff \alpha} \left.\JJ(u + \alpha h)\right|_{\alpha = 0}. \]
		\item Якщо \[\delta \JJ (u, h) = \int h(s) \cdot g(s) \diff s,\] то $g(s)$ -- похідна за Фреше.
	\end{enumerate}
\end{algorithm}

\begin{problem*}
	Побудувати рівняння у варіаціях для системи керування $\dot x = A x + B u$.
\end{problem*}

\begin{algorithm} \tt
	Рівняння у варіаціях має загальний вигляд \[ \frac{\diff z(t)}{\diff t} = \frac{\partial f(x(t),u(t),t)}{\partial x} \cdot z(t) + \frac{\partial f(x(t),u(t),t)}{\partial u} \cdot h(t), \quad z(0) = 0. \]
\end{algorithm}

\begin{problem*}
	Знайти першу варіацію за Лагранжем і похідну Фреше в просторі інтегрованих з квадратом функцій для задачи оптимального керування варіаційним методом \[ \JJ = \int f \diff s + \Phi(T) \to \inf \] за умови, що \[ \dot x = f_0 (x, u), \] і розв'язати цю задачу.
\end{problem*}

\begin{algorithm} \tt
	\begin{enumerate}
	\item Позначимо $\phi (\alpha) = \JJ (u + \alpha h)$. 

	\item Знайдемо $\phi' (\alpha)$.

	\item Підставляючи $\alpha = 0$, знаходимо \[\phi' (0) = \int ... \diff s + \underset{=-\psi(T)}{\underbrace{\Phi_1(T)}} \cdot z (T). \]

	\item Запишемо рівняння у варіаціях на функцію $z(t)$.

	\item Введемо додаткові, спряжені змінні $\psi$ такі, що \[ \psi (T) = - \frac{\partial \Phi (x (T))}{\partial x}. \] 

	\item Тоді \[\left\langle \frac{\partial \Phi (x (T))}{\partial x}, z (T) \right\rangle = - \langle \psi (T), z (T) \rangle.\] 

	\item Враховуючи рівняння у варіаціях, маємо
	\begin{align*}
		\psi (T) \cdot z (T) &= \psi (T) \cdot z (T) - \psi (t_0) \cdot z (t_0) = \\
		&= \int_{t_0}^T \left( \psi (s) \cdot z' (s) + \psi' (s) \cdot z (s) \right) \diff s = ...
	\end{align*}

	\item Підставимо це у вигляд $\phi' (0)$:
	\[ \phi' (0) = - \int (\psi' + ...) \cdot z \diff s + \int (...) \cdot h(s)) \diff s.\]

	\item Накладаємо на функцію $\psi(t)$ умову (спряжену систему) \[ \frac{\diff \psi (t)}{\diff t} = - \frac{\partial f (x (t), u (t), t)}{\partial x} \cdot \psi (t) + \frac{\partial f_0 (x (t), u (t), t)}{\partial x} = 0, \]

	\item Завдяки цьому у $\delta \JJ (u, h) = \phi' (0)$ перший інтеграл зануляється. \\

	\item Знаходимо $\JJ ' (u)$

	\item З необхідної умову екстремуму функціоналу, $\JJ' (u_*) = 0$, зна\-хо\-ди\-мо $u_*$.

	\item Далі \[x_*(t) = x_0 + \int_0^t f(x(s), u_*(s), s) \diff s.\]

	\item Покладаючи $t = T$ знаходимо $x (T)$.

	\item Остаточно знаходимо $u_*$, $x_*$.
	\end{enumerate}
\end{algorithm}

 \newpage
\input{06-WithSolutions.tex} \newpage
\subsection{Домашнє завдання}

\begin{problem}
	Знайти першу варіацію за Лагранжем і похідну Фреше в просторі інтегрованмх з квадратом функцій для функціоналів:
	\begin{enumerate}
		\item $\JJ(u) = \int_0^T \cos(u(s)) \diff s$;
		\item $\JJ(u) = \int_0^T (s^2 u_1^4(s) + u_2^2(s)) \diff s$, $u=(u_1,u_2)^*$.
	\end{enumerate}
\end{problem}

\begin{solution}
	\begin{enumerate}
		\item Перша варіація за Лагранжем:
		\begin{multline*} 
			\delta \JJ(u, \psi) = \frac{\diff}{\diff \alpha} \left.\JJ(u + \alpha \psi)\right|_{\alpha = 0} = \frac{\diff}{\diff \alpha} \left.\int_0^T \cos(u(s) + \alpha \psi(s)) \diff s\right|_{\alpha = 0} = \\
			= \left.\int_0^T - \psi(s) \sin(u(s) + \alpha \psi(s)) \diff s\right|_{\alpha = 0} = - \int_0^T \psi(s) \sin(u(s)) \diff s.
		\end{multline*}
		Як наслідок, похідна за Фреше $\JJ'(u) = - \sin u(\cdot)$.

		\item \begin{multline*} 
			\delta \JJ(u, \psi) = \frac{\diff}{\diff \alpha} \left.\JJ(u + \alpha \psi)\right|_{\alpha = 0} = \\
			= \frac{\diff}{\diff \alpha} \left.\int_0^T (s^2 (u_1 + \alpha \psi_1)^4(s) + (u_2 + \alpha \psi_2)^2(s)) \diff s\right|_{\alpha = 0} = \\
			= \left.\int_0^T (4 s^2 \psi_1(s) (u_1 + \alpha \psi_1)^3(s) + 2 \psi_2(s) (u_2 + \alpha \psi_2)(s)) \diff s\right|_{\alpha = 0} = \\
			= \int_0^T (4 s^2 \psi_1(s) u_1^3(s) + 2 \psi_2(s) u_2 (s)) \diff s.
		\end{multline*}
	\end{enumerate}
\end{solution}

\begin{problem}
	Побудувати рівняння у варіаціях для системи керування \[ \frac{\diff x(t)}{\diff t} = \cos(x(t) + u(t)), \quad x(0) = x_0. \] Тут $x(t)\in\RR^1$, $u(t)\in\RR^1$, $t\in[0,T]$. Точки $x_0\in\RR^1$ і момент часу $T$ є заданими.
\end{problem}

\begin{solution}
	Рівняння у варіаціях має загальний вигляд \[ \frac{\diff z(t)}{\diff t} = \frac{\partial f(x(t),u_*(t),t)}{\partial x} \cdot z(t) + \frac{\partial f(x(t),u_*(t),t)}{\partial u} \cdot h(t), \quad z(0) = 0. \] У нашій задачі \[f(x(t), u(t), t) = \cos(x(t) + u(t)), \] тому маємо \[ \frac{\diff z(t)}{\diff t} = -\sin(x(t) + u(t)) \cdot z(t) -\sin(x(t) + u(t)) \cdot h(t), \quad z(0) = 0. \]
\end{solution}

\begin{problem}
	Побудувати рівняння у варіаціях для системи керування \[ \left\{ \begin{aligned}
		\frac{\diff x_1(t)}{\diff t} &= x_1(t)\cdot x_2(t) + u_1(t), \\
		\frac{\diff x_2(t)}{\diff t} &= x_1(t) - x_2(t) \cdot u_2(t),
	\end{aligned} \right. \]
	\[ x_1(0) = -1, x_2 (0) = 4. \]

	Тут $x = (x_1, x_2)^*$ -- вектор фазових координат з $\RR^2$, $u = (u_1, u_2)^*$, $t\in[0, T]$, момент часу $T$ є заданим.
\end{problem}

\begin{solution}
	Рівняння у варіаціях має загальний вигляд \[ \frac{\diff z(t)}{\diff t} = \frac{\partial f(x(t),u_*(t),t)}{\partial x} \cdot z(t) + \frac{\partial f(x(t),u_*(t),t)}{\partial u} \cdot h(t), \quad z(0) = 0. \] У нашій задачі \[f(x(t), u(t), t) = \begin{pmatrix} x_1 \cdot x_2 + u_1 \\ x_1 - x_2 \cdot u_2 \end{pmatrix}, \] тому маємо \[ \frac{\diff z(t)}{\diff t} = \begin{pmatrix} x_2 & x_1 \\ 1 & -u_2 \end{pmatrix} \cdot z(t) + \begin{pmatrix} 1 & 0 \\ 0 & - x_2 \end{pmatrix} \cdot h(t), \quad z(0) = 0, \] або, у розгорнутому вигляді, \[ \left\{ \begin{aligned}
		\dot z_1 &= x_2 z_1 + x_1 z_2 + h_1, \\
		\dot z_2 &= z_1 - u_2 z_2 - x_2 h_2, \\
		0 &= z_1(0) = z_2(0).
	\end{aligned} \right. \]
\end{solution}

\begin{problem}
	Знайти першу варіацію за Лагранжем і похідну Фреше в просторі інтегрованих з квадратом функцій для задачи оптимального керування варіаційним методом \[ \JJ(u) = \int_0^T (u^2(s) + x^4(s)) \diff s + x^4(T) \to \inf \] за умови, що \[ \frac{\diff x(t)}{\diff t} = x(t) \cdot u(t), \quad x(0) = x_0. \] Тут $x(t) \in \RR^1$, $u(t)\in\RR^1$, $t\in[0,T]$. Точки $x_0\in\RR^1$ і момент часу $T$ є заданими.
\end{problem}

\begin{solution}
	Перш за все запишемо
	\[ \phi(\alpha) = \JJ(u + \alpha h) = \int_0^T ((u + \alpha h)^2(s) + x^4(s, \alpha)) \diff s + x^4(T, \alpha). \]

	Далі, \[ \phi'(\alpha) = \int_0^T (2 h(s) \cdot (u(s) + \alpha h(s)) + 4 x^3(s, \alpha) x_\alpha'(s, \alpha)) \diff s + 4 x^3 (T, \alpha) x_\alpha'(T, \alpha). \]

	Підставимо $\alpha = 0$: \[ \delta \JJ(u, h) = \int_0^T (2 h(s) u(s) + 4 x^3(s) z(s)) \diff s + \underset{=-\psi(T)}{\underbrace{4 x^3(T)}} \cdot z(T).\] 

	Тоді рівняння у варіаціях \[ \left\{ \begin{aligned} 
		z' &= u \cdot z + x \cdot h, \\
		z(0) &= 0.
	\end{aligned} \right. \]

	\begin{multline*} 
		\psi(T) \cdot z(T) = \psi(T) \cdot z(T) - \psi(0) \cdot z(0) = \int_0^T (\psi(s) \cdot z(s))' \diff s = \\
		= \int_0^T (\psi'(s) \cdot z(s) + \psi(s) \cdot z'(s)) \diff s = \\
		= \int_0^T \phi'(s) z(s) + \phi(s) (u(s) z(s) + x(s) h(s)) = \\
		= \int_0^T \phi'(s) z(s) u(s) z(s) \diff s + \int_0^T \psi(s) x(s) h(s) \diff s.
	\end{multline*}

	\begin{multline*} 
		\delta \JJ(u, h) = \int_0^T 2 h(s) u(s) + 4 x^3(s) z(s) \diff s - \\
		- \int_0^T \psi'(s) z(s) u(s) z(s) \diff s - \int_0^T \psi(s) x(s) h(s) \diff s = \\
		= \int_0^T (\psi'(s) z(s) + \psi(s) x(s) h(s) + u(s) z(s)) \diff s.
	\end{multline*}

	\[ \int_0^T z(s) (\psi'(s) + \psi(s) u(s)) \diff s + \int_0^T h(s) \psi(s) x(s( \diff s + \int)) \]

	\[ ... ??? ... \]

	% Тоді спряжена система \[ \left\{ \begin{aligned} 
	% 	[]
	% \end{aligned} \right. \]

	% Остаточно, $\JJ'(u) = \psi(s) s$.
\end{solution}

\begin{problem}
	Розв'язати задачу оптимального керування варіаційним методом: \[ \JJ (u) = \int_0^T (u (s) - v (s))^2 \diff s + (x (T) - 3)^2 \to \inf \] за умови, що \[ \frac{\diff x (t)}{\diff t} = u(t), \quad x(0) = x_0. \] Тут $x (t) \in \RR^1$, $u (t) \in \RR^1$, $t \in [0, T]$. Точки $x_0 \in \RR^1$, момент часу $T$ і функція $v (t) \in \RR^1$ є заданими.
\end{problem}

\begin{solution}
	Нагадаємо постановку задачі варіаційного методу: \[ \JJ (u) = \int_{t_0}^T f_0 (x(t), u(t), t) \diff t + \Phi(x(T)) \to \inf, \] \[ \frac{\diff x (t)}{\diff t} = f(x(t), u(t), t), \quad x(t_0) = x_0. \]

	Спочатку випишемо всі функції з теоретичної частини які фігурують в задачі: 
	\begin{align*}
		f_0(x(t), u(t), t) &= (u (t) - v (t))^2, \\
		\Phi(x(T)) &= (x (T) - 3)^2, \\
		f(x(t), u(t), t) &= u(t).
	\end{align*}

	Позначимо \[ \phi (\alpha) = \JJ (u + \alpha h) = \int_0^T ((u + \alpha h) (s) - v (s))^2 \diff s + (x (T, \alpha) - 3)^2. \]

	Необхідна умова екстремуму через першу варіацію функціоналу має вигляд $\delta \JJ (u_*, h) = \phi' (0) = 0$, тому знайдемо \[ \phi' (\alpha) = \int_0^T \left( 2 h (s) \cdot ((u + \alpha h) (s) - v (s)) \right) \diff s + 2 (x (T, \alpha) - 3) \cdot \underset{=z(T)}{\underbrace{\frac{\partial x (T, \alpha)}{\partial \alpha}}}. \]

	Підставляючи $\alpha = 0$, знаходимо \[ \phi' (0) = \int_0^T \left( 2 h (s) \cdot (u(s) - v (s)) \right) \diff s + \underset{=-\psi(T)}{\underbrace{2 (x (T) - 3)}} \cdot z (T). \]

	Запишемо рівняння у варіаціях на функцію $z(t)$. Його загальний вигляд \[ \frac{\diff z(t)}{\diff t} = \frac{\partial f(x(t), u(t), t)}{\partial x} \cdot z(t) + \frac{\partial f(x(t), u(t), t)}{\partial u} \cdot h(t), \quad z(t_0) = 0. \]

	У контексті нашої задачі маємо \[ \frac{\diff z(t)}{\diff t} = 0 \cdot z(t) + 1 \cdot h(t), \quad z(0) = 0. \]

	Введемо додаткові, спряжені змінні $\psi$ такі, що \[ \psi (T) = - \frac{\partial \Phi (x (T))}{\partial x}. \] 

	Тоді $\left\langle \frac{\partial \Phi (x (T))}{\partial x}, z (T) \right\rangle = - \langle \psi (T), z (T) \rangle$ (у контексті нашої задачі ``скалярний'' добуток зайвий бо функції і так скалярні). Враховуючи рівняння у варіаціях, маємо
	\begin{align*}
		\psi (T) \cdot z (T) &= \psi (T) \cdot z (T) - \psi (t_0) \cdot z (t_0) = \\
		&= \int_{t_0}^T \left( \psi (s) \cdot z' (s) + \psi' (s) \cdot z (s) \right) \diff s = \\
		&= \int_0^T \left( \psi (s) \cdot h (s) + \psi' (s) \cdot z (s) \right) \diff s.
	\end{align*}

	Підставимо це у вигляд $\phi' (0)$:
	\begin{align*}
		\phi' (0) &= \int_{t_0}^T \left( \frac{\partial f_0 (x (t), u (t), t)}{\partial x} \cdot z(t) + \frac{\partial f_0 (x (t), u (t), t)}{\partial u} \cdot h(t) \right) + \\
		& \left.\right. \quad + \frac{\partial \Phi (x (T))}{ \partial x} \cdot z (T) = \\
		&= \int_{t_0}^T \left( \frac{\partial f_0 (x (t), u (t), t)}{\partial x} \cdot z(t) + \frac{\partial f_0 (x (t), u (t), t)}{\partial u} \cdot h(t) \right) - \\
		& \left.\right. \quad - \int_0^T \left( \psi (s) \cdot h (s) + \psi' (s) \cdot z (s) \right) \diff s = \\
		&= \int_0^T \left( 2 h (s) \cdot (u(s) - v (s)) \right) \diff s - \\
		& \left.\right. \quad - \int_0^T \left( \psi (s) \cdot h (s) + \psi' (s) \cdot z (s) \right) \diff s = \\
		&= \int_0^T - \psi'(s) \cdot z(s) \diff s + \int_0^T (2 (u(s) - v(s)) - \psi(s)) \cdot h(s) \diff s.
	\end{align*}

	Накладаємо на функцію $\psi(t)$ умову (спряжену систему) \[ \frac{\diff \psi (t)}{\diff t} = - \frac{\partial f (x (t), u (t), t)}{\partial x} \cdot \psi (t) + \frac{\partial f_0 (x (t), u (t), t)}{\partial x} = 0, \] \[ \psi(T) = - \frac{\partial \Phi( x (T))}{\partial x} = 2 (x (T) - 3), \] звідки $\psi (t) = 2 (x (T) - 3)$. \\

	Завдяки цьому \[ \delta \JJ (u, h) = \phi' (0) = \int_0^T (2 (u(s) - v(s)) - \psi(s)) \cdot h(s) \diff s. \]

	Як наслідок, \[ \JJ ' (u) = 2 (u (\cdot) - v (\cdot)) - \psi(\cdot)). \]

	Пригадуючи необхідну умову екстремуму функціоналу, знаходимо \[ u_* (t) = v (t) + \psi (t) / 2 = v (t) + x (T) - 3. \]

	Далі \begin{multline*} 
		x_*(t) = x_0 + \int_0^t f(x(s), u_*(s), s) \diff s = \\
		= x_0 + \int_0^t (v(s) + x(T) - 3) \diff s = t x (T) - 3 t + \int_0^t v(s) \diff s.
	\end{multline*}

	покладаючи $t = T$ знаходимо \[ x (T) = T x (T) - 3 T + \int_0^T v(s) \diff s, \] звідки \[ x(T) = \frac{\int_0^T v(s) \diff s - 3  T}{1 - T}, \] і остаточно \[ u_* (t) = v (t) + \frac{\int_0^T v(s) \diff s - 3  T}{1 - T} - 3, \] \[ x_* (t) =  \frac{t \cdot \left(\int_0^T v(s) \diff s - 3  T\right)}{1 - T} - 3 t + \int_0^t v(s) \diff s .\]
\end{solution}

\begin{problem}
	% 6.13
\end{problem}

\begin{solution}
	% 6.13
\end{solution}

\begin{problem}
	% 6.14
\end{problem}

\begin{solution}
	% 6.14
\end{solution}

\begin{problem}
	% 6.15
\end{problem}

\begin{solution}
	% 6.15
\end{solution}
 \newpage

\section{Принцип максимуму Понтрягіна для задачі з вільним правим кінцем}

\subsection{Алгоритми}

\begin{problem*}
    Записати крайову задачу принципу максимуму для задачі оптимального керування: \[ \JJ = \int f \diff s + \Phi(T) \to \inf \] за умови, що \[ \dot x = f_0. \] Розв'язати задачу оптимального керування.
\end{problem*}

\begin{algorithm} \tt
    \begin{enumerate}
        \item Записуємо функцію Гамільтона-Понтрягіна: \[ \mathcal{H} (x, u, \psi, t) = - f_0(x, u, t) + \langle \psi, f(x, u, t) \rangle. \]
    
        \item Записуємо спряжену систему: \[ \dot \psi = - \nabla_x \mathcal{H}, \quad \psi(T) = - \nabla \Phi(x(T)). \]
    
    
        \item Знаходимо $u(\psi)$ з умови оптимальності: \[ \dfrac{\partial \mathcal{H}(x, u, \psi, t)}{\partial u} = 0. \]

        \item Підставляємо знайдене керування у початкову систему, от\-ри\-ма\-ли \allowbreak край\-о\-ву задачу, систему диференціальних рівнянь на $x$ і $\psi$ з гра\-нич\-ни\-ми  \allowbreak у\-мо\-ва\-ми.

        \item Розв'язуємо крайову задачу і знаходимо $x$.

        \item Відновлюємо $u = u (\psi)$ за знайденим $\psi$.
    \end{enumerate}
\end{algorithm}
 \newpage
\input{07-WithSolutions.tex} \newpage
\input{07-NoSolutions.tex} \newpage

\section{Принцип максимуму Понтрягіна: загальний випадок}

\subsection{Алгоритми}

\begin{problem*}
	Розв'язати задачу оптимального керування за допомогою принципу максимуму Понтрягіна: \[ \JJ  = \int f_0 \diff s + \Phi_0 \to \inf \] за умов, що \[ \dot x = f, \] а також \[ \int f_i \diff s + \Phi_i = 0 , \quad i = \overline{1..k}. \]
\end{problem*}

\begin{algorithm} \tt
	\begin{enumerate}
		\item Запишемо функцію Гамільтона-Понтрягіна: \[ \HH = -F + \langle \psi, f \rangle. \]
		\item Запишемо термінант: \[ F = \sum_i \lambda_i f_i, \quad \ell = \sum_i \lambda_i \Phi_i. \]
		\item Випишемо тепер всі (необхідні) умови принципу максимуму:
		\begin{enumerate}
			\item оптимальність: \[\frac{\partial \HH}{\partial u} = 0;\]
			\item стаціонарність (спряжена система): \[\dot \psi = - \nabla_x \HH;\]
			\item трансверсальність: \[\psi(t_0) = \frac{\partial \ell}{\partial x_0}, \quad \psi(T) = - \frac{\partial \ell}{\partial x_T};\]
			\item стаціонарність за кінцями: відсутня, бо час фіксований;
			\item доповнююча нежорсткість: відсутня, бо немає інтегральних \allowbreak об\-ме\-жень виду нерівність на задачу;
			\item невід'ємність: $\lambda_i \ge 0$.
		\end{enumerate}
		\item Методом від супротивного показуємо, що $\lambda_i \ne 0$.
		\item З умов принципу максимуму визначаємо $u = u(\psi)$.
		\item Записуємо крайову задачу -- систему диференціальних рівнянь на $x$ і $\psi$ з граничними умовами.
		\item Знаходимо її розв'язок $x_*$.
		\item Відновлюємо $u_* = u_*(\psi)$.
	\end{enumerate}
\end{algorithm}

\begin{problem*}
	Розв'язати задачу оптимальної швидкодії за допомогою принципу максимуму Понтрягіна: \[ \JJ  = \int f_0 \diff s + \Phi_0 \to \inf \] за умов, що \[ \dot x = f, \] а також \[ \int f_i \diff s + \Phi_i = 0 , \quad i = \overline{1..k}. \]
\end{problem*}


\begin{algorithm} \tt
	\begin{enumerate}
		\item Запишемо функцію Гамільтона-Понтрягіна: \[ \HH = -F + \langle \psi, f \rangle. \]
		\item Запишемо термінант: \[ F = \sum_i \lambda_i f_i, \quad \ell = \sum_i \lambda_i \Phi_i. \]
		\item Випишемо тепер всі (необхідні) умови принципу максимуму:
		\begin{enumerate}
			\item оптимальність: \[\frac{\partial \HH}{\partial u} = 0;\]
			\item стаціонарність (спряжена система): \[\dot \psi = - \nabla_x \HH;\]
			\item трансверсальність: \[\psi(t_0) = \frac{\partial \ell}{\partial x_0}, \quad \psi(T) = - \frac{\partial \ell}{\partial x_T};\]
			\item стаціонарність за кінцями: \[ \HH(T) = \frac{\partial \ell}{\partial T}; \]
			\item доповнююча нежорсткість: відсутня, бо немає інтегральних \allowbreak об\-ме\-жень виду нерівність на задачу;
			\item невід'ємність: $\lambda_i \ge 0$.
		\end{enumerate}
		\item Методом від супротивного показуємо, що $\lambda_i \ne 0$.
		\item З умов принципу максимуму визначаємо $u = u(\psi)$.
		\item Записуємо крайову задачу -- систему диференціальних рівнянь на $x$ і $\psi$ з граничними умовами.
		\item З умов принципу максимуму і крайової задачі визначаємо $T$.
		\item Знаходимо розв'язок крайової задачі $x_*$.
		\item Відновлюємо $u_* = u_*(\psi)$.
	\end{enumerate}
\end{algorithm}



 \newpage
\input{08-WithSolutions.tex} \newpage
\subsection{Домашнє завдання}

\begin{problem}
	Розв'язати задачу оптимального керування за допомогою принципу максимуму Понтрягіна: \[ \JJ (u) = \frac{1}{2} \int_{-1}^1 (u^2 (s) + x^2 (s)) \diff s \to \inf \] за умови, що \[ \frac{\diff x(t)}{\diff t} = u (t), \quad x(-1) = x(1) = 1. \] Тут $x (t) \in \RR^1$, $u (t) \in \RR^1$, $t \in [-1, 1]$.
\end{problem}

\begin{solution}
	% 8.5
\end{solution}

\begin{problem}
	Розв'язати задачу оптимального керування за допомогою принципу максимуму Понтрягіна: \[ \JJ (u) = \frac{1}{2} \int_0^1 (u^2 (s) + x^2 (s)) \diff s \to \inf \] за умови, що \[ \frac{\diff^2 x(t)}{\diff t^2} = u (t), \quad x(0) = 1, \dot x(0) = -2, x(1) = 0, \dot x(1) = 0. \] Тут $x (t) \in \RR^1$, $u (t) \in \RR^1$, $t \in [0, 1]$.
\end{problem}

\begin{solution}
	% 8.6
\end{solution}

\begin{problem}
	Розв'язати задачу оптимального керування за допомогою принципу максимуму Понтрягіна: \[ \JJ (u) = \frac{1}{2} \int_0^1 (x(s) + u^2 (s)) \diff s \to \inf \] за умови, що \[ \frac{\diff x(t)}{\diff t} = x (t) + u (t), \quad x(1) = 0. \] Тут $x (t) \in \RR^1$, $u (t) \in \RR^1$, $t \in [0, 1]$.
\end{problem}

\begin{solution}
	% 8.7
\end{solution}

\begin{problem}
	Розв'язати задачу оптимального керування за допомогою принципу максимуму Понтрягіна: \[ \JJ (u) = \int_{-\pi}^\pi x(s) \sin(s) \diff s \to \inf \] за умови, що \[ \dot x = u, \quad x(-\pi) = x(\pi) = 0, u(t) \in [-1,1], \] де $x (t) \in \RR^1$, $t \in [-\pi, \pi]$.
\end{problem}

\begin{solution}
	% 8.8
\end{solution}
 \newpage

\section{Дискретний варіант методу динамічного програмування}

\input{09-Algorithms.tex} \newpage
\subsection{Аудиторне заняття}

\begin{problem}
	Знайти оптимальне керування, оптимальну траєкторію, функцію Белмана і оптимальне значення критерія якості задачі оптимального керування \[ \JJ(\{u(k)\},\{x(k)\}) = \sum_{k=0}^2 u^2(k) + x^2(3) \to \inf \] за умов \[ x(k + 1) = 2 x(k) + u(k), \quad x(0) = 1, k = 0,1,2.\] Тут $x,u\in\RR^1$.
\end{problem}

\begin{solution}
	Випишемо функції що фігурують в задачі: \[ g_k (x (k), u (k)) = u^2 (k), \quad \Phi (x (N)) = x^2 (3), \quad f_k (x (k), u (k)) = 2 x (k) + u (k). \]

	\[\BB_3 (z) = \Phi (z) = z^2. \]

	Послідовно знаходимо $u_*$:
	\begin{enumerate}

	\item Запишемо визначення $u_*(2)$:
	\begin{multline*} 
		u_*(2) = \argmin_{u(2)} (g_2(z, u(2)) + \BB_3(f_2(z,u(2)))) = \\
		= \argmin_{u(2)} (u^2(2) + f_2^2(z, u(2))) = \\
		= \argmin_{u(2)} (u^2(2) + (2 z + u(2))^2).
	\end{multline*}

	Знайдемо $u_*(2)$ з умови 
	\[ 0 = \frac{\partial}{\partial u(2)} \left(u_*^2(2) + (2 z + u_*(2))^2 \right) = 2 u_*(2) + 2 (2 z + u_*(2)) = 4 u_*(2) + 4 z,\]

	звідки $u_*(2) = - z$. \\

	Знайдемо
	\begin{multline*}
		\BB_2 (z) = (g_2(z, u_*(2)) + \BB_3(f_2(z,u_*(2)))) = \\
		= u_*^2(2) + (2 z + u_*(2))^2 = z^2 + (2 z - z)^2 = 2 z^2.
	\end{multline*}

	\item Запишемо визначення $u_*(1)$:
	\begin{multline*} 
		u_*(1) = \argmin_{u(1)} (g_1(z, u(1)) + \BB_2(f_1(z,u(1)))) = \\
		= \argmin_{u(1)} (u^2(1) + 2 f_1^2(z, u(1))) = \\
		= \argmin_{u(1)} (u^2(1) + 2 (2 z + u(1))^2).
	\end{multline*}

	Знайдемо $u_*(1)$ з умови 
	\[ 0 = \frac{\partial}{\partial u(1)} \left(u_*^2(1) + 2 (2 z + u_*(1))^2 \right) = 2 u_*(1) + 4 (2 z + u_*(1)) = 6 u_*(1) + 8 z,\]

	звідки $u_*(1) = - \frac43 z$. \\

	Знайдемо
	\begin{multline*}
		\BB_1 (z) = (g_1(z, u_*(1)) + \BB_2(f_1(z,u_*(1)))) = \\
		= u_*^2(1) + 2 (2 z + u_*(1))^2 = \frac{16}{9} z^2 + 2 \left(2 z - \frac43z\right)^2 = \frac{8}{3} z^2.
	\end{multline*}


	\item Запишемо визначення $u_*(0)$:
	\begin{multline*} 
		u_*(0) = \argmin_{u(0)} (g_0(z, u(0)) + \BB_1(f_0(z,u(0)))) = \\
		= \argmin_{u(0)} (u^2(0) + \frac{8}{3} f_0^2(z, u(0))) = \\
		= \argmin_{u(0)} (u^2(0) + \frac{8}{3} (2 z + u(0))^2).
	\end{multline*}

	Знайдемо $u_*(0)$ з умови 
	\begin{multline*} 
		0 = \frac{\partial}{\partial u(0)} \left(u_*^2(0) + \frac{8}{3} (2 z + u_*(0))^2 \right) = \\
		= 2 u_*(0) + \frac{16}{3} (2 z + u_*(0)) = \frac{22}{3} u_*(0) + \frac{32}{3} z,
	\end{multline*}	

	звідки $u_*(0) = - \frac{16}{11} z$. \\

	Знайдемо
	\begin{multline*}
		\BB_0 (z) = (g_0(z, u_*(0)) + \BB_1(f_0(z,u_*(0)))) = \\
		= u_*^2(0) + \frac{8}{3} (2 z + u_*(0))^2 = \frac{256}{121} z^2 + \frac{8}{3} \left(2 z - \frac{16}{11} z\right)^2 = \frac{32}{11} z^2.
	\end{multline*}
	\end{enumerate}

	Оскільки $\XX_0 = \{ x_0 \}$, то $x_*(0) = x_0$, $\JJ_* = \frac{32 x_0^2}{11}$. \\

	Відновимо тепер траєкторію: 
	\begin{enumerate}
		\item \[x_*(1) = f_0(x_*(0), u_*(0)) = 2 x_0 + u_*(0) = 2 x_0 - \frac{16}{11} x_0 = \frac{6 x_0}{11}.\]
		\item \[x_*(2) = f_1(x_*(1), u_*(1)) = 2 \frac{6 x_0}{11} + u_*(1) = \frac{12 x_0}{11} - \frac{4}{3} \frac{6 x_0}{11} = \frac{4 x_0}{11}.\]
		\item \[x_*(3) = f_2(x_*(2), u_*(2)) = 2 \frac{4 x_0}{11} + u_*(2) = \frac{8 x_0}{11} -  \frac{4 x_0}{11} = \frac{4 x_0}{11}.\]
	\end{enumerate}
\end{solution}

\begin{problem}
	Знайти оптимальне керування і функцію Белмана задачі оптимального керування \[ \JJ ( \{ u (k) \}, \{ x (k) \} ) = \sum_{k = 0}^{N - 1} u^2 (k) + x^2 (N) \to \inf \] за умов \[ x (k + 1) = x (k) + u (k), x (0) = x_0, k = 0, 1, \ldots, N - 1. \] Тут $x, u \in \RR^1$. Точка $x_0 \in \RR^1$ -- відома.
\end{problem}

\begin{solution}
	Будемо шукати функцію Белмана у вигляді \[ \BB_s(z) = b (s) \cdot z^2. \] Зауважимо, що $\BB_N(z) = z^2$, тому $b(N) = 1$. \\

	Далі записуємо дискретне рівняння Белмана \[ b (s) \cdot z^2 = \min_u ( u^2 + b (s + 1) \cdot ( z + u )^2 ). \]

	Знайдемо $u_*$ з умови \[ 0 = \frac{\partial}{\partial u} \left( u^2 + b (s + 1) \cdot ( z + u )^2 \right) = 2 u + 2 b (s + 1) ( z + u ), \] звідки \[ u_* (s) = - \frac{ b (s + 1) \cdot x (s) }{ 1 + b (s + 1) }. \]

	Підставляючи це у дискретне рівняння Белмана отримаємо дискретне рівняння для знаходження $b (s)$: \[ b (s) = \frac{ b (s + 1) }{ b (s + 1) + 1 }. \]

	Звідси нескладно отримати $b (N - k) = \frac{1}{k + 1}$, зокрема $b (0) = \frac{1}{N + 1}$. \\

	Далі нескладно отримати \[u_* (s) = - \frac{x_0}{N + 1},\] а \[x_* (s) = \frac{ N - s }{ N + 1 } \cdot x_0.\]

	Воно й не дивно, бо задача має вигляд \[ u_0^2 + u_1^2 + \ldots + u_{n-1}^2 + (x_0 - u_0 - u_1 - \ldots - u_{n-1})^2 \to \min \] Тобто мінімізуємо суму квадратів чисел зі сталою ($x_0$) сумою. \\

	За теоремою Штурма, мінімум суми квадратів досягається коли всі ці квадрати рівні (і дорівнюють $\frac{1}{(N + 1)^2}$), це відповідає знайденому нами керуванню.
\end{solution}

\begin{problem}
	Знайти оптимальне керування і функцію Белмана задачі оптимального керування \[ \JJ ( \{ u (k) \}, \{ x (k) \} ) = \sum_{k = 0}^{N - 1} (u(k) - v(k))^2 + x^2 (N) \to \inf \] за умов \[ x (k + 1) = x (k) + u (k), x (0) = x_0, k = 0, 1, \ldots, N - 1. \] Тут $x, u \in \RR^1$. Точка $x_0 \in \RR^1$ -- відома, $v(k)$ -- відомі, $k=0,1,\ldots,N-1$.
\end{problem}

\begin{solution}
	Будемо шукати функцію Белмана у вигляді \[ \BB_s (z) = p (s) \cdot z^2 + q (s) \cdot z + r (s). \] Зауважимо, що $\BB_N (z) = z^2$, тому $p (N) = 1$, $q (N) = r (N) = 0$. \\

	Далі записуємо дискретне рівняння Белмана \begin{multline*} p (s) \cdot z^2 + q (s) \cdot z + r (s) = \\ = \min_u ( (u - v)^2 + p (s + 1) \cdot (z + u)^2 + q (s + 1) \cdot (z + u) + r (s + 1) ). \end{multline*}

	Знайдемо $u_*$ з умови \begin{multline*} 0 = \frac{\partial}{\partial u} \left( (u - v)^2 + p (s + 1) \cdot (z + u)^2 + q (s + 1) \cdot (z + u) + r (s + 1) \right) = \\ = 2 (u - v) + 2 p (s + 1) \cdot (z + u) + q (s + 1), \end{multline*} звідки \[ u_* (s) = \frac{ 2 v (s) - 2 p (s + 1) x (s) - q (s + 1) }{ 2 + 2 p (s + 1) }. \]

	Підставляючи це у дискретне рівняння Белмана отримаємо дискретне рівняння для знаходження $p (s)$, $q (s)$, $r (s)$:
	\begin{multline*} p (s) \cdot z^2 + q (s) \cdot z + r (s) = \left( \frac{2 v (s) - 2 p (s + 1) z - q (s + 1)}{2 + 2 p (s + 1)} - v (s) \right) + \\ + p (s + 1) \cdot \left( z + 2 v (s) - 2 p (s + 1) z - q (s + 1) \right)^2 + \\ + q (s + 1) \cdot \left( z + 2 v (s) - 2 p (s + 1) z - q (s + 1) \right) + r (s + 1) \end{multline*}

	Збираючи коефіцієнти при відповідних степенях $z$ знаходимо систему для знаходження $p (s)$, $q (s)$, $r (s)$.
\end{solution}

\begin{problem}
	% 9.4
\end{problem}

\begin{solution}
	% 9.4
\end{solution}

\begin{problem}
	% 9.5
\end{problem}

\begin{solution}
	% 9.5
\end{solution}
 \newpage
\subsection{Домашнє завдання}

\begin{problem}
	% 9.6
\end{problem}

\begin{solution}
	% 9.6
\end{solution}

\begin{problem}
	% 9.7
\end{problem}

\begin{solution}
	% 9.7
\end{solution}

\begin{problem}
	% 9.8
\end{problem}

\begin{solution}
	% 9.8
\end{solution}

\begin{problem}
	% 9.9
\end{problem}

\begin{solution}
	% 9.9
\end{solution}

\begin{problem}
	% 9.10
\end{problem}

\begin{solution}
	% 9.10
\end{solution}
 \newpage

\section{Метод динамічного програмування}

\subsection{Алгоритми}

\begin{problem*}
	Розв'язати задачу оптимального керування і знайти функцію Белмана: \[ \JJ = \int f_0 \diff s + \Phi(T) \to \inf \] за умови, що \[ \dot x = f. \]
\end{problem*}

\begin{algorithm} \tt
	\begin{enumerate}
		\item Відрізок $[t_0, T]$ розбивається сіткою $t_0 < t_1 < t_2 < \ldots < t_N = T$ з деяким кроком $h$.
		\item Позначаємо $\XX_i = \XX (t_i)$, $\UU_i = \UU (t_i)$, $i = 0,1,2,\ldots,N$.
		\item $\BB_N(z) = \Phi(z)$.
		\item Для $s=\overline{N-1..0}$ записуємо і розв'язуємо рівняння Белмана: \begin{multline*}\BB_s(z) = \inf_{u\in \UU_s} \left(\int_{t_s}^{t_{s+1}} f_0(x(\tau,u(\tau),\tau)\diff\tau\right.+\\+\left.\BB_{s+1}\left(z+\int_{t_s}^{t_{s+1}} f(x(\tau),u(\tau),\tau) \diff \tau\right)\right) \end{multline*} для всіх $z \in \XX_s$, запам'ятовуючи $\{u_*(s)\}$. \\

		Розв'язуємо ми його через рівняння Гамільтона-Якобі-Белмана, \[\dot \BB(z) + \inf_u (\langle \nabla_z \BB(z), f \rangle + f_0(z) ) = 0. \]

		\item Знаходимо $x_*(t_0)$ як \[ x_*(t_0) = \argmin_{z\in \XX_0} \BB_0(z).\]
		\item Знаходимо $\JJ_*$ як $\JJ_* = \BB_0(x_*(t_0))$.
		\item Для $s=\overline{0..N-1}$ відновлюємо $x_*(t_{s+1})$ за відомим керуванням: \[x_*(t_{s+1}) = x_*(t_s) + \int_{t_s}^{t_{s+1}} f(x_*(\tau),u_*(\tau),\tau) \diff \tau.\] 
	\end{enumerate}
\end{algorithm} \newpage
\subsection{Аудиторне заняття}

\begin{problem}
	Розв'язати задачу оптимального керування і знайти функцію Белмана: \[ \JJ (u) = \frac12 \int_0^T u^2 (s) \diff s + \frac{x^2 (T)}{2} \to \inf \] за умови, що \[ \frac{\diff x (t)}{\diff t} = u (t), x (0) = x_0. \] Тут $x (t) \in \RR^1$, $u (t) \in \RR^1$, $t \in [0, T]$. Точка $x_0 \in \RR^1$ і момент часу $T$ є заданими.
\end{problem}

\begin{solution}
	Запишемо диференціальне рівняння Гамільтона-Якобі-Белмана: \[ \frac{\partial \BB (z, t)}{\partial t} + \int_u \left( \frac{\partial \BB (z, t)}{\partial z} \cdot u (t) + \frac{u^2 (t)}{2} \right) = 0. \]
\end{solution}

\begin{problem}
	Розв'язати задачу оптимального керування і знайти функцію Белмана: \[ \JJ (u) = \frac12 \int_0^T u^2 (s) \diff s + \frac{x^2 (T)}{2} \to \inf \] за умови, що \[ \ddot x = u, x (0) = x_0, \dot x (0) = y_0. \] Тут $x (t) \in \RR^1$, $u (t) \in \RR^1$, $t \in [0, T]$. Точка $x_0 \in \RR^1$ і момент часу $T$ є заданими.
\end{problem}

\begin{solution}
	% 10.2
\end{solution}

\begin{problem}
	Розв'язати задачу оптимального керування і знайти функцію Белмана: \[ \JJ (u) = \frac12 \int_0^T (u (s) - s)^2 \diff s + \frac{x^2 (T)}{2} \to \inf \] за умови, що \[ \frac{\diff x (t)}{\diff t} = u (t) + t^2, x (0) = x_0. \] Тут $x (t) \in \RR^1$, $u (t) \in \RR^1$, $t \in [0, T]$. Точка $x_0 \in \RR^1$ і момент часу $T$ є заданими.
\end{problem}

\begin{solution}
	% 10.3
\end{solution}

\begin{problem}
	Знайти функцію Белмана такої задачі оптимального керування: \[ \JJ (u) = \frac12 \int_0^T u^2 (s) \diff s \to \inf \] за умови, що \[ \frac{\diff x (t)}{\diff t} = u (t), x (0) = x_0, x (1) = 1. \] Тут $x (t) \in \RR^1$, $u (t) \in \RR^1$, $t \in [0, T]$. Точка $x_0 \in \RR^1$ і момент часу $T$ є заданими.
\end{problem}

\begin{solution}
	% 10.4
\end{solution}
 \newpage
\subsection{Домашнє завдання}

\begin{problem}
	% 10.5
\end{problem}

\begin{solution}
	% 10.5
\end{solution}

\begin{problem}
	% 10.6
\end{problem}

\begin{solution}
	% 10.6
\end{solution}

\begin{problem}
	% 10.7
\end{problem}

\begin{solution}
	% 10.7
\end{solution}

\begin{problem}
	% 10.8
\end{problem}

\begin{solution}
	% 10.8
\end{solution}

\begin{problem}
	% 10.9
\end{problem}

\begin{solution}
	% 10.9
\end{solution}

\begin{problem}
	% 10.10
\end{problem}

\begin{solution}
	% 10.10
\end{solution}



\end{document}