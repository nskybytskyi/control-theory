% cd ..\..\Users\NikitaSkybytskyi\Desktop\control-theory

% cls && pdflatex ahw.tex && cls && pdflatex ahw.tex && start ahw.pdf

\documentclass[a5paper, 10pt]{article}
\usepackage[T2A,T1]{fontenc}
\usepackage[utf8]{inputenc}
\usepackage[english, ukrainian]{babel}
\usepackage{amsmath, amssymb}
\usepackage[top = 2 cm, left = 1 cm, right = 1 cm, bottom = 2 cm]{geometry} 

\title{{\Huge ТЕОРІЯ КЕРУВАННЯ}}
\date{}

\usepackage{fancyhdr}
\pagestyle{fancy}
\lhead{Семінари з теорії керування, 2018}
\rhead{Нікіта Скибицький, ОМ-3}
\cfoot{\thepage}

\usepackage{amsthm}
\newtheorem{definition}{Визначення}
\theoremstyle{definition}
\newtheorem*{problem*}{\normalfont{\textit{Задача}}}
\newtheorem{problem}{\normalfont{\textit{Задача}}}[section]
\newtheorem*{solution}{Розв'язок}

\allowdisplaybreaks
\setlength\parindent{0pt}
\numberwithin{equation}{section}

\usepackage{xcolor}
\usepackage{hyperref}
\hypersetup{unicode=true,colorlinks=true,linktoc=all,linkcolor=red}

\usepackage{graphicx}

\newcommand*\diff{\mathop{}\!\mathrm{d}}
\newcommand{\JJ}{\mathcal{J}}
\newcommand{\KK}{\mathcal{K}}
\newcommand{\MM}{\mathcal{M}}
\newcommand{\UU}{\mathcal{U}}
\newcommand{\XX}{\mathcal{X}}
\newcommand{\NN}{\mathcal{N}}
\newcommand{\EE}{\mathcal{E}}
\newcommand{\RR}{\mathbb{R}}
\renewcommand{\SS}{\mathcal{S}}
\newcommand{\Max}{\displaystyle\max\limits}
\DeclareMathOperator{\erf}{erf}
\DeclareMathOperator{\erfi}{erfi}
\DeclareMathOperator{\signum}{sgn}

\begin{document}

\maketitle \thispagestyle{empty} \newpage 

\tableofcontents \newpage

\subsection{Аудиторне заняття}

\begin{problem}
	Задана скалярна система керування 
	
	\begin{equation}
		\label{eq:1.1}
		\frac{\diff x(t)}{\diff t} = u(t), \quad x(0) = 1.
	\end{equation}

	Тут $x$ -- стан системи, $t \in [0, 1]$. Керування задане у вигляді

	\begin{equation}
		\label{eq:1.2}
		u(x) = a x.
	\end{equation}

	Тут $a$ -- скалярний параметр.

	\begin{enumerate}
		\item Знайти траєкторію системи (\ref{eq:1.1}) при керуванні (\ref{eq:1.2}).

		\item Знайти програмне керування $u(t) = a x (t)$, яке відповідає знайденій траєкторії. 

		\item Оцінити, при якому значення параметра $a \in \{2, 4, -3\}$ критерій якості 

		\[ \JJ(u) = x^2 (1) \]

		буде мати менше значення.
	\end{enumerate}
\end{problem}

\begin{solution}
	\begin{enumerate}
		\item Підставляючи керування (\ref{eq:1.2}) у систему (\ref{eq:1.1}), отримаємо систему \[ \frac{\diff x(t)}{\diff t} = a x(t), \quad x(0) = 1. \]

		Її розв'язок має вигляд \[ x(t) = x(0) \cdot e^{a t} = e^{a t}. \]

		\item Просто підставляємо знайдену у попередньому пункті траєкторію у вигляд (\ref{eq:1.2}) керування: \[ u(t) = a x(t) = a \cdot e^{a t}. \]

		\item Множина $\{2, 4, -3\}$ скінченна, тому можна просто перебрати всі її елементи та обчислити значення критерію якості на кожному з них: \[ \begin{aligned} \left. \JJ (u) \right|_{a = 2} &= x^2 (1) = \left. e^{2 a t} \right|_{t = 1} = e^4, \\ \left. \JJ (u) \right|_{a = 4} &= x^2 (1) = \left. e^{2 a t} \right|_{t = 1} = e^8, \\ \left. \JJ (u) \right|_{a = -3} &= x^2 (1) = \left. e^{2 a t} \right|_{t = 1} = e^{-6}. \end{aligned} \]

		%Як бачимо, н
		Найменшим з цих значень є $e^{-6}$ яке досягається при $a = -3$. \\

		% Насправді це передбачувано, бо з загального вигляду траєкторії для $a > 0$ видно, що модуль $x(t)$ зростає з часом $t$, а для $a < 0$ -- зменшується.
	\end{enumerate}
\end{solution}

\begin{problem}
	Задана система керування

	\begin{equation}
		\label{eq:1.3}
		\left\{
			\begin{aligned}
				\frac{\diff x_1 (t)}{\diff t} &= x_1 (t) + x_2 (t) + u (t), \\
				\frac{\diff x_2 (t)}{\diff t} &= - x_1 (t) + x_2 (t) + u(t),
			\end{aligned}
		\right.
		\quad
		x_1 (0) = 2, x_2 (0) = 1.
	\end{equation}

	Тут $x = (x_1, x_2)^*$ -- вектор фазових координат з $\RR^2$, $t \in [0, T]$. Керування задане у вигляді

	\begin{equation}
		\label{eq:1.4}
		u(x_1, x_2) = 2 x_1 + x_2.
	\end{equation}

	\begin{enumerate}
		\item Знайти траєкторію системи (\ref{eq:1.3}) при керуванні (\ref{eq:1.4}).

		\item Знайти програмне керування $u(t) = 2 x_1 (t) + x_2 (t)$, яке відповідає знайденій траєкторії.

		\item Якою буде фундаментальна матриця, нормована за моментом $s$, системи. що одержана при підстановці керування (\ref{eq:1.4}) у систему (\ref{eq:1.3})?

		\item Побудувати спряжену систему до системи, одержаної при підстановці керування (\ref{eq:1.4}) у систему (\ref{eq:1.3}), та її фундаментальну матрицю.
	\end{enumerate}
\end{problem}

\begin{solution}
	\begin{enumerate}
		\item Підставляючи керування (\ref{eq:1.4}) у систему (\ref{eq:1.3}), отримаємо систему \[
		\left\{
			\begin{aligned}
				\frac{\diff x_1 (t)}{\diff t} &= 3 x_1 (t) + 2 x_2 (t), \\
				\frac{\diff x_2 (t)}{\diff t} &= x_1 (t) + 2 x_2 (t),
			\end{aligned}
		\right.
		\quad
		x_1 (0) = 2, x_2 (0) = 1.
		\]

		Її розв'язок \[
		\left\{
			\begin{aligned}
				x_1 (t) &= 2 e^{4 t}, \\
				x_2 (t) &= e^{4 t}.
			\end{aligned}
		\right.
		\]

		\item Просто підставляємо знайдену у попередньому пункті траєкторію у вигляд (\ref{eq:1.4}) керування: \[ u(t) = 2 x_1 (t) + x_2 = 2 \cdot \left( 2 e^{4 t} \right) + e^{4 t} = 5 e^{4 t}. \]

		\item Враховуючи, що загальним розв'язком системи (\ref{eq:1.3}) з підставленим керуванням (\ref{eq:1.4}) є \[
		\left\{
			\begin{aligned}
				x_1 (t) &= c_1 e^t + 2 c_2 e^{4 t}, \\
				x_2 (t) &= - c_1 e^t + c_2 e^{4 t}.
			\end{aligned}
		\right.
		\]

		Це означає, що фундаментальна матриця цієї системи матиме вигляд \[
			\Theta(t) = 
			\begin{pmatrix}
				c_{11} e^t + 2 c_{12} e^{4 t} & c_{21} e^t + 2 c_{22} e^{4 t} \\
				- c_{11} e^t + c_{12} e^{4 t} & - c_{21} e^t + c_{22} e^{4 t}
			\end{pmatrix}
		\]

		Залишається пронормувати її за моментом $s$, тобто знайти такі $c_{11} (s)$, $c_{12} (s)$, $c_{21} (s)$, $c_{22} (s)$, що $\Theta(s, s) = E$. Коли це зробити, то отримаємо \[
			\Theta(t, s) = \frac{1}{3}
			\begin{pmatrix}
				e^{t - s} + 2 e^{4 (t - s)} & -2 e^{t - s} + 2 e^{4 (t - s)} \\
				e^{t - s} - e^{4 (t - s)} & 2 e^{t - s} + e^{4 (t - s)}
			\end{pmatrix}
		\]

		\item Спряжена система буде \[
		\left\{
			\begin{aligned}
				\frac{\diff z_1 (t)}{\diff t} &= - 3 z_1 (t) - z_2 (t), \\
				\frac{\diff z_2 (t)}{\diff t} &= - 2 z_1 (t) - 2 z_2 (t),
			\end{aligned}
		\right.
		\]

		а відповідна фундаментальна матриця \[ \Psi(t, s) = \Theta^*(s, t) = \frac{1}{3}
			\begin{pmatrix}
				e^{s - t} + 2 e^{4 (s - t)} & e^{s - t} - e^{4 (s - t)} \\
				-2 e^{s - t} + 2 e^{4 (s - t)} & 2 e^{s - t} + e^{4 (s - t)}
			\end{pmatrix}
		\]
	\end{enumerate}
\end{solution}

\begin{problem}
	Розглядається задача Больца \[ \JJ (u) = \int_0^1 u^2 (s) \diff s + ( x (1) - 2 )^2 \to \inf \]

	за умови, що \[ \frac{\diff x (t)}{\diff t} = x^2 (t) + u (t), \quad x (0) = x_0. \]

	Тут $x (t) \in \RR^1$, $u (t) \in \RR^1$, $t \in [0, 1]$. Точка $x_0 \in \RR^1$ задана. Звести цю задачу до задачі Майєра.
\end{problem}

\begin{solution}
	Введемо нову змінну \[ x_2 = \int_0^t u^2 (s) \diff s, \] 

	тоді \[ \JJ (u) = x_2 (1) + ( x_1 (1) - 2)^2 \to \inf, \]

	за умов, що \[ 
	\left\{
		\begin{aligned}
			\frac{\diff x_1 (t)}{\diff t} &= x_1^2 (t) + u (t), \\
			\frac{\diff x_2 (t)}{\diff t} &= u^2 (t).
		\end{aligned}
	\right.
	\quad
	x_1 (0) = x_0, x_2 (0) = 0.
	\]
\end{solution}

\begin{problem}
	Задана система керування 
	\[
		\left\{
			\begin{aligned}
				\dfrac{dx_1(t)}{dt} &= 2x_1(t) + x_2(t) + u(t), \\
				\dfrac{dx_2(t)}{dt} &= 3x_1(t) + 4x_2(t),
			\end{aligned}
		\right.
		\quad
		x_1(0) = 1, x_2(0) = -1.
	\]

	Тут $x = (x_1, x_2)^*$ -- вектор фазових координат з $\RR^2$, $t \in [0, 2]$. Керування задане у вигляді
	\[
		u(t) 
		=
		\begin{cases}
			0, & \text{якщо } t \in [0, 1], \\
			1, & \text{якщо } t \in (1, 2].
		\end{cases}
	\]
	
	\begin{enumerate}
		\item Знайти траєкторію системи, яка відповідає цьому керуванню.
		\item Чи буде ця траєкторія неперервно диференційовною?
		\item Чи буде таке керування кращим в порівнянні з керуванням 
		\[
			u(t) = 0, t \in [0, 2]
		\]
		за умови, що критерій якості має вигляд 
		\[
			\mathcal{J}(u) = x_1^2(2) + x_2^2(2) \to \min.
		\]
	\end{enumerate}
\end{problem}

\begin{solution}
	\begin{enumerate}
		\item При $t \in [0, 1]$ маємо 
		\[
			\begin{pmatrix}
				\dot x_1 \\ 
				\dot x_2
			\end{pmatrix}
			=
			\begin{pmatrix}
				2 & 1 \\
				3 & 4
			\end{pmatrix}
			\begin{pmatrix}
				x_1 \\
				x_2
			\end{pmatrix}.
		\]
		Характеристичне рівняння 
		\[
			\begin{vmatrix}
				2 - \lambda & 1 \\
				3 & 4 - \lambda 
			\end{vmatrix}
			=
			\lambda^2 - 6\lambda + 5 
			=
			(\lambda - 1) (\lambda - 5)
			=
			0,
		\]
		звідки $\lambda_1 = 1$, $\lambda_2 = 5$.\\
		
		З курсу диференційних рівнянь відомо, що тоді загальний розв'язок має вигляд
		\[
			\begin{pmatrix}
			x_1 \\
			x_2
			\end{pmatrix}
			=
			c_1 v_1 e^t + c_2 v_2 e^{5t},
		\]
		де $v_1$, $v_2$ -- власні вектори, що відповідають $\lambda_1$ та $\lambda_2$ відповідно.\\
		
		Нескладно бачити, що 
		\[
			\begin{pmatrix}
				x_1 \\
				x_2
			\end{pmatrix}
			=
			c_1 
			\begin{pmatrix}
				1 \\
				-1
			\end{pmatrix} 
			e^t 
			+ 
			c_2 
			\begin{pmatrix}
				1 \\
				3
			\end{pmatrix} 
			e^{5t}.
		\]
		
		Підставляючи $t = 0$ отримуємо $c_1 = 1$, $c_2 = 0$.
		
		При $t \in (1, 2]$ маємо 
		\[
			\begin{pmatrix}
				x_1 \\
				x_2
			\end{pmatrix}
			=
			c_1 v_1 e^t 
			+ 
			c_2 v_2 e^{5t} 
			+ 
			\begin{pmatrix} 
				c_3 \\
				c_4
			\end{pmatrix},
		\]
		де $c_3$, $c_4$ задовольняють систему 
		\[
			\left\{
				\begin{aligned}
					2c_3 &+ c_4 + 1 &= 0 \\
					3c_3 &+ 4c_4 &= 0
				\end{aligned}
			\right.,
		\]
		звідки $c_3 = -4/5$, $c_4 = 3/5$ і 
		\[
			\begin{pmatrix}
				x_1 \\
				x_2
			\end{pmatrix}
			=
			c_1 v_1 e^t 
			+ 
			c_2 v_2 e^{5t} 
			+ 
			\begin{pmatrix} 
				-4/5 \\
				3/5
			\end{pmatrix},
		\]
		Підставляючи $t = 1$ отримуємо $c_1 = \left(1 + \dfrac{3}{4e}\right)$, $c_2 = \dfrac{1}{20e^5}$.\\
		
		Остаточно маємо 
		\[
			\begin{pmatrix}
			x_1 \\
			x_2
			\end{pmatrix}
			=
			\begin{cases}
				\begin{pmatrix}
					1 \\
					-1
				\end{pmatrix} 
				e^t, & t \in [0, 1] \\
				\left(1 + \dfrac{3}{4e}\right) 
				\begin{pmatrix}
				1 \\
				-1
				\end{pmatrix} 
				e^t 
				+ 
				\dfrac{1}{20e^5} 
				\begin{pmatrix}
				1 \\
				3
				\end{pmatrix} 
				e^{5t} 
				+ 
				\begin{pmatrix} 
					-4/5 \\
					3/5
				\end{pmatrix}			
				, & t \in (1, 2]
			\end{cases}.
		\]
		\item 
		\[
			\begin{pmatrix}
				\dot x_1 \\
				\dot x_2
			\end{pmatrix}
			(1-) 
			= 
			\begin{pmatrix}
				2 & 1 \\
				3 & 4
			\end{pmatrix}
			\begin{pmatrix}
				x_1(1-) \\
				x_2(1-)
			\end{pmatrix}
		\]
		З неперервності $x_1$, $x_2$ маємо:
		\[
			\begin{pmatrix}
				2 & 1 \\
				3 & 4
			\end{pmatrix}
			\begin{pmatrix}
				x_1(1-) \\
				x_2(1-)
			\end{pmatrix}
			=
			\begin{pmatrix}
			2 & 1 \\
			3 & 4
			\end{pmatrix}
			\begin{pmatrix}
			x_1(1) \\
			x_2(1)
			\end{pmatrix}
		\]
		З іншого боку,
		\[
			\begin{pmatrix}
				\dot x_1 \\
				\dot x_2
			\end{pmatrix}
			(1+) 
			= 
			\begin{pmatrix}
				2 & 1 \\
				3 & 4
			\end{pmatrix}
			\begin{pmatrix}
				x_1(1+) \\
				x_2(1+)
			\end{pmatrix}
			+
			\begin{pmatrix}
				1 \\
				0
			\end{pmatrix}
			= 
			\begin{pmatrix}
				2 & 1 \\
				3 & 4
			\end{pmatrix}
			\begin{pmatrix}
				x_1(1) \\
				x_2(1)
			\end{pmatrix}
			+
			\begin{pmatrix}
				1 \\
				0
			\end{pmatrix}
		\]
		Нескладно бачити, що 
		\[
			\begin{pmatrix}
				\dot x_1 \\
				\dot x_2
			\end{pmatrix}
			(1-)
			\ne
			\begin{pmatrix}
				\dot x_1 \\
				\dot x_2
			\end{pmatrix}
			(1+),
		\]
		тобто траєкторія не є неперервно диференційовною в точці $1$.
		\item 
		Просто підставимо $t=2$ в розв'язки для обох керувань (попутно зауваживши, що для нового керування розв'язок ми вже знаємо, це просто продовження вже знайденого розв'язку для $t \in [0, 1]$):
		\[
			\left(e^2 + \dfrac34e + \dfrac{e^5}{20} - \dfrac45\right)^2 + \left(-e^2 - \dfrac34e + \dfrac{3e^5}{20} + \dfrac35\right)^2
			\lor
			(e^2)^2 + (-e^2)^2
		\]
		Після марудних обчислень знаходимо, що права частина менше, тобто нове керування є кращим за початкове.
	\end{enumerate}
\end{solution}
 % OK, incomplete, missing 1.7

\subsection{Домашнє завдання}

\begin{problem}
	Задана система керування 
	\begin{equation}
		\label{eq:1.5}
		\left\{
			\begin{aligned}
				\dfrac{dx_1(t)}{dt} &= -8x_1(t)  -x_2(t) + u(t),\\
				\dfrac{dx_2(t)}{dt} &= 6x_1(t) + 3x_2(t),
			\end{aligned}
		\right.
		\quad
		x_1(0) = -2, x_2(0) = 1.
	\end{equation}

	Тут $ x =(x_1, x_2)^*$ -- вектор фазових координат з $\RR^2$, $t \in [0, 1]$. Керування задане у вигляді
	\begin{equation}
		\label{eq:1.6}
		u(x_1, x_2) = 4x_1 - x_2.
	\end{equation}

	\begin{enumerate}
		\item До якого класу керувань належить керування (\ref{eq:1.6}): програмних керувань, чи керувань з оберненим зв'язком?
		\item Знайти траєкторію системи при керуванні (\ref{eq:1.6}).
		\item Знайти програмне керування $u(t) = 4x_1(t) - x_2(t)$, яке відповідає знайденій траєкторії.
		\item Якою буде фундаментальна матриця, нормована за моментом $s$, системи, що одержана при підстановці керування (\ref{eq:1.6}) в систему (\ref{eq:1.5})?
		\item Побудувати спряжену систему до системи, одержаної при підстановці керування (\ref{eq:1.6}) в систему (\ref{eq:1.5}), та її фундаментальну матрицю.
	\end{enumerate}
\end{problem}

\begin{solution}
	\begin{enumerate}
		\item З оберненим зв'язком.
		\item 
		\[
			\begin{pmatrix}
				\dot x_1 \\ 
				\dot x_2
			\end{pmatrix}
			=
			\begin{pmatrix}
				-4 & -2 \\
				6 & 3
			\end{pmatrix}
			\begin{pmatrix}
				x_1 \\
				x_2
			\end{pmatrix}.
		\]
		Характеристичне рівняння 
		\[
			\begin{vmatrix}
				-4 - \lambda & -2 \\
				6 & 3 - \lambda 
			\end{vmatrix}
			=
			\lambda^2 + \lambda 
			=
			(\lambda + 1) \lambda
			=
			0,
		\]
		звідки $\lambda_1 = -1$, $\lambda_2 = 0$.\\
		
		З курсу диференційних рівнянь відомо, що тоді загальний розв'язок має вигляд
		\[
			\begin{pmatrix}
				x_1 \\
				x_2
			\end{pmatrix}
			=
			c_1 v_1 e^{-t} + c_2 v_2,
		\]
		де $v_1$, $v_2$ -- власні вектори, що відповідають $\lambda_1$ та $\lambda_2$ відповідно.\\
		
		Нескладно бачити, що 
		\[
			\begin{pmatrix}
				x_1 \\
				x_2
			\end{pmatrix}
			=
			c_1 
			\begin{pmatrix}
				1 \\
				-2
			\end{pmatrix} 
			+ 
			c_2 
			\begin{pmatrix}
				2 \\
				-3
			\end{pmatrix}
			e^{-t} .
		\]
		
		Підставляючи $t=0$ отримуємо $c_1 = 4$, $c_2 = -3$.\\
		
		Остаточно маємо:
		\[
			\begin{pmatrix}
				x_1 \\
				x_2
			\end{pmatrix}
			=
			4
			\begin{pmatrix}
				1 \\
				-2
			\end{pmatrix} 
			-3
			\begin{pmatrix}
				2 \\
				-3
			\end{pmatrix}
			e^{-t}
			.
		\]
		\item Просто підставляємо знайдені $x_1(t)$, $x_2(t)$ в $u(x_1, x_2)$:
		\[
			u(t)
			=
			4\left(4\cdot (1) - 3\cdot (2)\cdot e^{-t}\right)
			-
			\left(4\cdot (-2) - 3\cdot (-3)\cdot e^{-t}\right)
			=
			24 - 33e^{-t}.
		\]
		\item З вигляду загального розв'язку бачимо, що вищезгадана фундаментальна матриця матиме вигляд
		\[
			\Theta(t,s)
			=
			\begin{pmatrix}
				c_1 + 2c_2 e^{s-t} & c_3 + 2c_4 e^{s-t} \\
				-2c_1 - 3c_2 e^{s-t} & -2c_3 - 3c_4 e^{s-t} 
			\end{pmatrix},
		\]
		причому 
		\[
			\left\{
				\begin{aligned}
					c_1   &+ 2c_2 &= 1 \\
					-2c_1 &- 3c_2 &= 0
				\end{aligned}
			\right.		
		\]
		(і аналогічна система для $c_3$, $c_4$).\\
		
		Знаходимо $c_1 = -3$, $c_2 = 2$, $c_3 = -2$, $c_4 = 1$ і підставляємо у матрицю:
		\[
			\Theta(t,s)
			=
			\begin{pmatrix}
				-3 + 4e^{s-t} & -2 + 2e^{s-t} \\
				6 - 6e^{s-t} & 4 - 3e^{s-t} 
			\end{pmatrix},
		\]
		
		\item Спряжена система буде
		\[
			\begin{pmatrix}
				\dot y_1 \\
				\dot y_2
			\end{pmatrix}
			=
			\begin{pmatrix}
				4 & -6 \\
				2 & -3
			\end{pmatrix}
			\begin{pmatrix}
				y_1 \\
				y_2
			\end{pmatrix},
		\]
		а відповідна фундаметальна матриця
		\[
			\Psi(t,s)
			=
			\Theta^*(s,t)
			=
			\begin{pmatrix}
				-3 + 4e^{t-s} & 6 - 6e^{t-s} \\
				-2 + 2e^{t-s} & 4 - 3 e^{t-s}
			\end{pmatrix},
		\]
	\end{enumerate}
\end{solution}

\begin{problem}
	Розглядається задача Лагранжа
		\[
		\JJ(u)
		=
		\int_0^T u^2(s) \diff s \to \inf
		\]
		за умови, що
		\[
			\left\{
				\begin{aligned}
					\dfrac{dx_1(t)}{dt} &= -x_1(t) + x_2(t) + u(t), \\
					\dfrac{dx_2(t)}{dt} &= x_1(t)x_2(t),
				\end{aligned}
			\right.
			\quad
			x_1(0)=0,x_2(0)=1.
		\]
		Тут $x=(x_1,x_2)^*$ -- вектор фазових координат з $\RR^2$, $t\in[0,T]$. Звести цю задачу до задачі Майєра.
\end{problem}

\begin{solution}
	Введемо нову фазову координату $x_3(t)=\int_0^t u^2(s) \diff s$, тоді до системи додається початкова умова $x_3(0)=0$, рівняння $\dot x_3 = u^2$, а функціонал якості переписується у вигляді $x_3(T) \to \inf$.
\end{solution}


\begin{problem}
	% 1.7
\end{problem}

\begin{solution}
	% 1.7
\end{solution}
 \newpage

% OK, complete

\section{Елементи багатозначного аналізу. Множина досяжності}

\subsection{Аудиторне заняття}

\begin{problem}
	Знайти $A + B$ і $\lambda A$, а також метрику Хаусдорфа $\alpha (A, B)$, якщо:
	
	\begin{enumerate}
	    \item $A = \{-3, 2, -1\},  B = \{-2, 5, 1\},  \lambda = 3$;
	    
	    \item $A = \{4, 2, -4\},  B = [-2, 3],  \lambda = -1$;
	    
	    \item $A = [-1, 2],  B = [3, 7],  \lambda = -2$;
	    
	\end{enumerate}
	
\end{problem}

\begin{solution}

	\begin{enumerate}
	    \item За визначенням операції $A + B = \{-5, 2, -2, 0, 7, 3, -3, 4, 0\}$, $\lambda A = \{-9, 6, -3 \}$. \\
	    
	    Метрика Хаусдорфа визначатиметься як \[\alpha (A, B) = \max\{\beta (A, B), \beta (B, A)\},\] в свою чергу $\beta (A, B)$ визначається як максимум з мінімумів відхилень множини, тобто, у нашому випадку, $\beta (A, B) = \max\{1, 1, 1\}; \beta (B, A) = \max\{1, 3, 1\}$, тоді $\alpha (A, B) = 3$.
	    
	    \item За визначенням операції $A + B = [-6,7]$, $\lambda A = \{-4, -2, 4 \}$.\\
	    
	    Метрика Хаусдорфа визначатиметься як \[\alpha (A, B) = \max\{\beta (A, B), \beta (B, A)\},\] в свою чергу $\beta (A, B)$ визначається як максимум з мінімумів відхилень множини, тобто, у нашому випадку, $\beta (A, B) = \max\{1, 0, 2\}; \beta (B, A) = \max[0, 3]$, оскільки $-1$ відхиляється від найближчих елементів на $3$ і це є максимумом, тоді $\alpha (A, B) = 3$.
	    
	    \item За визначенням операції $A + B = [2,9]$, $\lambda A = [-4,2]$. \\
	    
	    Метрика Хаусдорфа визначатиметься як \[\alpha (A, B) = \max\{\beta (A, B), \beta (B, A)\},\] в свою чергу $\beta (A, B)$ визначається як максимум з мінімумів відхилень множини, тобто, у нашому випадку, $\beta (A, B) = \max[1,4]; \beta (B, A) = \max[1, 5]$, оскільки відповідні краї відхиляються на $4$ та $5$ відповідно ($-1$ від $3$ та $7$ від $2$), тоді $\alpha (A, B) = 5$.
	    
	\end{enumerate}
	
\end{solution}

\begin{problem}
	Знайти $MA$, якщо
	\[
	M=
  \begin{pmatrix}
    -2 & 4 \\
    3 & 5
  \end{pmatrix}
  , A= 
  \left\{
  \begin{pmatrix}
    -1 \\
    2 
  \end{pmatrix},
    \begin{pmatrix}
    3 \\
    -4 
  \end{pmatrix},
    \begin{pmatrix}
    0 \\
    -2 
  \end{pmatrix}
  \right\}
  .\]
\end{problem}

\begin{solution}
	За означенням $MA = \{Ma\in \RR^m, a\in A\}$, тому
	\newline
   \[ \begin{pmatrix}
   -2 & 4 \\
   3 & 5
  \end{pmatrix}
  \begin{pmatrix}
    -1 \\
    2 
  \end{pmatrix}=
  \begin{pmatrix}
    10 \\
    7 
  \end{pmatrix};
   \begin{pmatrix}
   -2 & 4 \\
   3 & 5
  \end{pmatrix}
  \begin{pmatrix}
    3 \\
    -4 
  \end{pmatrix}=
  \begin{pmatrix}
    -22 \\
    -11 
  \end{pmatrix};
   \begin{pmatrix}
   -2 & 4 \\
   3 & 5
  \end{pmatrix}
  \begin{pmatrix}
    0 \\
    -2 
  \end{pmatrix}=
  \begin{pmatrix}
    -8 \\
    -10 
  \end{pmatrix};
  \]
  
  Отже, отримаємо \[MA = 
  \left\{
  \begin{pmatrix}
    10 \\
    7 
  \end{pmatrix},
  \begin{pmatrix}
    -22 \\
    -11 
  \end{pmatrix},
  \begin{pmatrix}
    -8 \\
    -10 
  \end{pmatrix}
  \right\}.\]
\end{solution}

\begin{problem}
	Знайти опорні функції таких множин:

	\begin{enumerate}
		\item $A = [0, r]$;

		\item $A = [-r, r]$;

		\item $A = \{ (x_1, x_2): |x_1| \le 1, |x_2| \le 2 \}$;

		\item $A = \KK_r (0) = \{ x \in \RR^n: \|x\| \le r \}$;

		\item $A = \SS^n = \{ x \in \RR^n: \|x\| = 1 \}$.
	\end{enumerate}
\end{problem}

\begin{solution}
	\begin{enumerate}
		\item За означення опорної функції, \[ c(A, \psi) = \max_{a \in A} \langle a, \psi \rangle = \begin{cases} 0, & \psi < 0 \\ r \psi, & 0 \le \psi \end{cases} = \max \{ 0, r \psi \}. \]

		\item За означення опорної функції, \[ c(A, \psi) = \max_{a \in A} \langle a, \psi \rangle = \begin{cases} - r \psi, & \psi < 0 \\ r \psi, & 0 \le \psi \end{cases} = r |\psi |. \]

		\item За означення опорної функції, \[ c(A, \psi) = \max_{a \in A} \langle a, \psi \rangle = \max_{a \in A} (\psi_1 x_1 + \psi_2 x_2) = |\psi_1| + 2 |\psi_2|. \]

		\item За властивістю опорної функції (вона дорівнює орієнтованій відстані від початку координат до опорної площини множини $A$ яка відповідає напрямку $\psi$), маємо $c(\KK_r (0), \psi) = r \| \psi \|$.

		\item За тією ж властивістю опорної функції маємо $c(\SS^n, \psi) = \| \psi \|$.
	\end{enumerate}
\end{solution}

\begin{problem}
	Знайти інтеграл Аумана $\JJ = \int_0^1 F(x) \diff x$ таких багатозначних відображень:

	\begin{enumerate}
		\item $F(x) = [0, x]$, $x \in [0, 1]$;

		\item $F(x) = \KK_x (0) = \{ y \in \RR^n: \|y\| \le x \}$, $x \in [0, 1]$.
	\end{enumerate}
\end{problem}

\begin{solution}
	Скористаємося рівністю \[ c \left(\int_0^1 F(x), \psi\right) \diff x = \int_0^1 c(F(x), \psi) \diff x, \]

	яка виконується в умовах теореми Ляпунова про опуклість інтегралу Аумана.

	\begin{enumerate}
		\item \[c(\JJ) = \int_0^1 c([0, x], \psi) \diff x = \begin{cases} \psi / 2, & 0 \le \psi \\ 0, & \psi < 0 \end{cases}. \]

		А далі наші знання опорних функцій підказують, що $\JJ = [0, 1 / 2]$.

		\item \[c(\JJ) = \int_0^1 c(\KK_x(0), \psi) \diff x = \int_0^1 x \| \psi \| \diff x = \| \psi \| / 2. \]

		А далі наші знання опорних функцій підказують, що $\JJ = \KK_{1 / 2} (0)$.
	\end{enumerate}
\end{solution}

\begin{problem}
	Знайти множину досяжності такої системи керування: \[ \frac{\diff x}{\diff t} = x + u, \]

	де $x (0) = x_0 \in \MM_0$, $u (t) \in \UU$, $t \ge 0$, \[ \MM_0 = \{ x: |x| \le 1 \}, \] \[ \UU = \{ u: |u| \le 1 \}. \]
\end{problem}

\begin{solution}
	Скористаємося теоремою про вигляд множини досяжності лінійної системи керування: \[ \XX(t, \MM_0) = \Theta(t, t_0) \MM_0 + \int_{t_0}^t \Theta(t, s) B(s) \UU(s) \diff s. \]

	Підставимо вже відомі значення: \[ \XX(t, [-1, 1]) = \Theta(t, 0) \cdot [-1, 1] + \int_0^t \left( \Theta(t, s) \cdot 1 \cdot [-1, 1] \right) \diff s, \] 

	тобто залишилося знайти $\Theta$. Знайдемо її з системи \[ \frac{\diff \Theta(t, s)}{\diff t} = A(t) \cdot \Theta(t, s) = \Theta(t, s). \]

	Нескладно бачити, що $\Theta(t, s) = e^{t - s}$, тому \begin{multline*} \XX(t, [-1, 1]) = [-e^t, e^t] + \int_0^t [-e^{t - s}, e^{t - s}] \diff s = \\ = [-e^t, e^t] + [1 - e^t, e^t - 1] = [1 - 2 e^t, 2 e^t - 1]. \end{multline*} 
\end{solution}

\begin{problem}
	Знайти опорну функцію множини досяжності для системи керування: \[
	\left\{
		\begin{aligned}
			\frac{\diff x_1}{\diff t} &= 2x_1 + x_2 + u_1, \\
			\frac{\diff x_2}{\diff t} &= 3x_1 + 4x_2 + u_2,
		\end{aligned}
	\right.
	\]

	де $x (0) = (x_{01}, x_{02}) \in \MM_0$, $u(t) = (u_1(t), u_2(t)) \in \UU$, $t \ge 0$, \[ \MM_0 = \{ (x_{01}, x_{02}): |x_{01}| \le 1, |x_{02}| \le 1 \}, \] \[ \UU = \{(u_1, u_2): |u_1| \le 1, |u_2| \le 1\}. \]
\end{problem}

\begin{solution}
	Скористаємося теоремою про вигляд опорної функції множини досяжності лінійної системи керування: \[ c(\XX(t, \MM_0), \psi) = c(\MM_0, \Theta^*(t, t_0) \psi) + \int_{t_0}^t c(\UU(s), B^*(s) \Theta^*(t, s) \psi) \diff s. \]

	Підставимо вже відомі значення: \[ c(\XX(t, [-1,1]^2), \psi) = c([-1,1]^2, \Theta^*(t, 0) \psi) + \int_0^t c([-1,1]^2, \begin{pmatrix} 1 & 1 \end{pmatrix} \Theta^*(t, s) \psi) \diff s, \] 

	тобто залишилося знайти $\Theta$. Знайдемо її з системи \[ \frac{\diff \Theta(t, s)}{\diff t} = A(t) \cdot \Theta(t, s) = \begin{pmatrix} 2 & 1 \\ 3 & 4 \end{pmatrix} \Theta(t, s). \]

	Нескладно бачити, що \[ \Theta(t, s) = \frac{1}{4}
	\begin{pmatrix}
		3 e^{t - s} + e^{5 (t - s)} & - e^{t - s} + e^{5 (t - s)} \\ -3 e^{t - s} + 3 e^{5 (t - s)} & e^{t - s} + 3 e^{5 (t - s)}
	\end{pmatrix} 
	\]

	Тому
	\begin{multline*} 
		c\left(\XX(t, [-1,1]^2), \begin{pmatrix} \psi_1 \\ \psi_2 \end{pmatrix}\right) = c\left([-1,1]^2, \frac{1}{4} \begin{pmatrix} 3 e^t + e^{5 t} & -3 e^t + 3 e^{5 t} \\ - e^t + e^{5 t} & e^t + 3 e^{5 t}	\end{pmatrix} \begin{pmatrix} \psi_1 \\ \psi_2 \end{pmatrix} \right) + \\
		+ \int_0^t c\left([-1,1]^2, \begin{pmatrix} 1 & 0 \\ 0 & 1 \end{pmatrix} \frac{1}{4} \begin{pmatrix} 3 e^{t - s} + e^{5 (t - s)} & -3 e^{t - s} + 3 e^{5 (t - s)} \\ - e^{t - s} + e^{5 (t - s)} & e^{t - s} + 3 e^{5 (t - s)} \end{pmatrix} \begin{pmatrix} \psi_1 \\ \psi_2 \end{pmatrix} \right) \diff s = \\
		= c\left([-1,1]^2, \frac{1}{4} \begin{pmatrix} (3 e^t + e^{5 t}) \psi_1 + (-3 e^t + 3 e^{5 t}) \psi_2 \\ (- e^t + e^{5 t}) \psi_1 + (e^t + 3 e^{5 t}) \psi_2 \end{pmatrix}  \right) + \\
		+ \int_0^t c\left([-1,1]^2, \frac{1}{4} \begin{pmatrix} (3 e^{t - s} + e^{5 (t - s)}) \psi_1 + (-3 e^{t - s} + 3 e^{5 (t - s)}) \psi_2 \\ (- e^{t - s} + e^{5 (t - s)}) \psi_1 + (e^{t - s} + 3 e^{5 (t - s)}) \psi_2 \end{pmatrix} \right) \diff s = \\
		= \frac{1}{4} \left( \left|(3 e^t + e^{5 t}) \psi_1 + (-3 e^t + 3 e^{5 t}) \psi_2\right| + \left|(- e^t + e^{5 t}) \psi_1 + (e^t + 3 e^{5 t}) \psi_2\right| \right) + \\
		+ \frac{1}{4} \int_0^t \left|(3 e^{t - s} + e^{5 (t - s)}) \psi_1 + (-3 e^{t - s} + 3 e^{5 (t - s)}) \psi_2\right| \diff s + \\
		+ \frac{1}{4} \int_0^t \left|(- e^{t - s} + e^{5 (t - s)}) \psi_1 + (e^{t - s} + 3 e^{5 (t - s)}) \psi_2\right| \diff s.
	\end{multline*}
\end{solution} % OK, incomplete, missing 2.7, 2.8

\subsection*{Домашнє завдання}

% \begin{problem*}
%     $A = \{-5, -3, 2\}$, $B = \{-2, 0, 3\}$, $\lambda = -3$. Знайти $A + B$ і $\lambda A$.
% \end{problem*}

% \begin{solution}
%     \begin{align*}
%         A + B &= \{(-5) + (-2), (-5) + 0, (-5) + 3, (-3) + (-2), (-3) + 0, \\
%         & \left.\right. \quad (-3) + 3, 2 + (-2), 2 + 0, 2 + 3\} = \{-7, -5, -2, -5, -3, 0, 0, 2, 5\} = \\
%         &= \{-7, -5, -3, -2, 0, 2, 5\}, \\
%         \lambda A &= \{(-3) \cdot (-5), (-3) \cdot (-3), (-3) \cdot 2\} = \{ 15, 9, -6 \}.
%     \end{align*}
% \end{solution}


% \begin{problem*}
%     $A = [-4, -3]$, $B = [-2, 6]$, $\lambda = 2$. Знайти $A + B$ і $\lambda A$.
% \end{problem*}

% \begin{solution}
%     $A + B = [(-4) + (-2), (-3) + 6] = [-6, 3]$, $\lambda A = [2\cdot(-4), 2\cdot(-3)] = [-8, -6]$.
% \end{solution}

% \begin{problem*}
%     $M = \begin{pmatrix} -2 & 1 \\ 3 & 7 \\ 0 & 4 \end{pmatrix}$, $A = \left\{ \begin{pmatrix} 1 \\ 0 \end{pmatrix}, \begin{pmatrix} -1 \\ 2 \end{pmatrix}, \begin{pmatrix} 3 \\ -2 \end{pmatrix} \right\}$. Знайти $MA$.
% \end{problem*}

% \begin{solution}
%     \begin{align*}
%         MA &= \left\{ \begin{pmatrix} -2 & 1 \\ 3 & 7 \\ 0 & 4 \end{pmatrix} \begin{pmatrix} 1 \\ 0 \end{pmatrix}, \begin{pmatrix} -2 & 1 \\ 3 & 7 \\ 0 & 4 \end{pmatrix} \begin{pmatrix} -1 \\ 2 \end{pmatrix}, \begin{pmatrix} -2 & 1 \\ 3 & 7 \\ 0 & 4 \end{pmatrix}  \begin{pmatrix} 3 \\ -2 \end{pmatrix} \right\} = \\
%         &= \left\{ \begin{pmatrix} -2 \\ 3 \\ 0 \end{pmatrix}, \begin{pmatrix} 4 \\ 11 \\ 8 \end{pmatrix}, \begin{pmatrix} -8 \\ -5 \\ -8 \end{pmatrix}\right\}.
%     \end{align*}
% \end{solution}

\begin{problem}
    % 2.7
\end{problem}

\begin{solution}
    % 2.7
\end{solution}

\begin{problem}
    % 2.8
\end{problem}

\begin{solution}
    % 2.8
\end{solution}


\begin{problem}
    Знайти опорні функції таких множин:
    \begin{enumerate}
        \item $A = \{ -1, 1 \}$;
        \item $A = \{ (x_1, x_2, x_3) : |x_1| \le 2, |x_2| \le  4, |x_3| \le 1 \}$;
        \item $A = \{ a \}$;
        \item $A = \KK_r(a) = \{ x\in \RR^n : \| x - a \| \le r \}$.
    \end{enumerate}
\end{problem}

\begin{solution}
    \begin{enumerate}
        \item За визначенням, $c(A, \psi) = \Max_{x \in \{-1, 1\}} \langle x, \psi\rangle = \max(-\psi,\psi) = |\psi|$.
        \item За визначенням, $c(A, \psi) = \Max_{\substack{ x_1 : |x_1| \le 2 \\ x_2 : |x_2| \le  4 \\ x_3 : |x_3| \le 1 }} x_1 \psi_1 + x_2 \psi_2 + x_3 \psi_3  = 2 |\psi_1| + 4 |\psi_2| + |\psi_3|$.
        \item За визначенням, $c(A, \psi) = \Max_{x \in \{ a \}} \langle x, \psi\rangle = \langle a, \psi\rangle $.
        \item За визначенням, 
        \begin{align*}
            c(A, \psi) &= \Max_{x \in \RR^n : \| x - a \| \le r} \langle x, \psi\rangle = \Max_{y \in \RR^n : \| y \| \le r} \langle a + y, \psi\rangle = \\
            &= \langle a, \psi\rangle + \Max_{y \in \RR^n : \| y \| \le r} \langle y, \psi\rangle = \langle a, \psi\rangle + c(\KK_r(0), \psi) = \langle a, \psi \rangle + r\|\psi\|.
        \end{align*}
    \end{enumerate}
\end{solution}

\begin{problem}
    Знайти інтеграл Аумана $\JJ = \int_0^{\pi/2} F(x) dx$ таких багатозначних відображень:
    \begin{enumerate}
        \item $F(x) = [0, \sin x]$, $x \in [0, \pi / 2]$.
        \item $F(x) = [-\sin x, \sin x]$, $x \in [0, \pi / 2]$.
        \item $F(x) = \KK_{\sin x}(0) = \{ y \in \RR^n : \| y \| \le \sin x \}$, $x \in [0, \pi / 2]$.
    \end{enumerate}
\end{problem}

\begin{solution}
    Скористаємося теоремою про зміну порядку інтегрування і взяття опорної функції:
    \begin{enumerate}
        \item $c(\JJ, \psi) = \int_0^{\pi/2} c([0, \sin x], \psi) dx = \int_0^{\pi/2} \max(0, \psi) \sin x dx = \max(0, \psi)$, звідки $\JJ = [0, 1]$.
        \item $c(\JJ, \psi) = \int_0^{\pi/2} c([-\sin x, \sin x], \psi) dx = \int_0^{\pi/2} |\psi| \sin x dx = |\psi|$, звідки $\JJ = [-1, 1]$.
        \item $c(\JJ, \psi) = \int_0^{\pi/2} c(\KK_{\sin x}(0), \psi) dx = \int_0^{\pi/2} \sin x \|\psi\| dx = \|\psi\|$, звідки $\JJ = \KK_1(0)$. 
    \end{enumerate}
\end{solution}

\begin{problem}
    Знайти множину досяжності такої системи керування:
    \[\frac{\diff x}{\diff t} = x + bu,\] 
    де $x(0) = x_0 \in \MM_0$, $u(t)\in \UU$, $t\ge0$, $b$ -- деяке ненульове число, 
    \[ \MM_0 = \{ x : | x | \le 2 \}, \]
    \[ \UU = \{ u : |u| \le 3 \}. \]
\end{problem}

\begin{solution}
    Множину досяжності знайдемо через її опорну функцію: 
    \[ c(\XX(t, \MM_0), \psi) = c(\MM_0, \Theta^*(t, t_0) \psi) + \int_{t_0}^t c(\UU(s), C^\star(s) \Theta^*(t, s)\psi) \diff s. \]
    Для цього послідовно знаходимо: \\
    
    $\Theta(t, s) = e^{t-s}$, знайдено із рівності $\dfrac{d\Theta(t,s)}{dt} = A(t)\Theta(t,s) = \Theta(t,s)$ у нашому випадку. \\
    
    $c(\MM_0, \psi) = c([-2, 2], \psi) = 2 |\psi|$, вже достатньо відома нам опорна функція. \\
    
    $c(\UU(s), \psi) = c([-3, 3], \psi) = 3 |\psi|$, ще одна вже достатньо відома нам опорна функція. \\
    
    % Послідовно пісдтавляючи знайдені вирази в формулу вище знаходимо:
    % \begin{equation*}
    % \begin{split}
    %     c(X(t, \MM_0), \psi) &= c(\MM_0, \Theta^\star(t, t_0), \psi) + \int_{t_0}^t c(\UU(s), C^\star(s) \Theta^\star(t, s)\psi) ds = \\
    %     &= c([-2,2], \Theta^\star(t, 0), \psi) + \int_0^t c([-3, 3], b \Theta^\star(t, s)\psi) ds = \\
    %     &= 2\left|\Theta^\star(t, 0)\psi\right| + \int_0^t 3\left|b \Theta^\star(t, s)\psi\right| ds = \\
    %     &= 2\left|e^{-t}\psi\right| + \int_0^t 3\left|b e^{s-t}\psi\right| ds = 2e^{-t}|\psi| + 3|b \psi| \int_0^t e^{s-t} ds = \\
    %     &= 2e^{-t}|\psi| + 3|b \psi| \left(1 - e^{-t}\right) = \left(2e^{-t} + 3|b|\left(1 - e^{-t}\right)\right) |\psi|,
    % \end{split}
    % \end{equation*}
    % звідки $X(t, \MM_0) = \left[-2e^{-t} - 3|b|\left(1 - e^{-t}\right), 2e^{-t} + 3|b|\left(1 - e^{-t}\right)\right]$.
    
    
    Послідовно пісдтавляючи знайдені вирази в формулу вище знаходимо:
    \begin{equation*}
    \begin{split}
        c(X(t, \MM_0), \psi) &= c(\MM_0, \Theta^\star(t, t_0), \psi) + \int_{t_0}^t c(\UU(s), C^\star(s) \Theta^\star(t, s)\psi) ds = \\
        &= c([-2,2], \Theta^\star(t, 0), \psi) + \int_0^t c([-3, 3], b \Theta^\star(t, s)\psi) ds = \\
        &= 2\left|\Theta^\star(t, 0)\psi\right| + \int_0^t 3\left|b \Theta^\star(t, s)\psi\right| ds = \\
        &= 2\left|e^t\psi\right| + \int_0^t 3\left|b e^{t-s}\psi\right| ds = 2e^t|\psi| + 3|b \psi| \int_0^t e^{t-s} ds = \\
        &= 2e^t|\psi| + 3|b \psi| \left(e^t - 1\right) = \left(2e^t + 3|b|\left(e^t - 1\right)\right) |\psi|,
    \end{split}
    \end{equation*}
    звідки $\XX(t, \MM_0) = \left[-2e^t - 3|b|\left(e^t - 1\right), 2e^t + 3|b|\left(e^t - 1\right)\right]$.
\end{solution}

\begin{problem}
Знайти опорну функцію множини досяжності для системи керування:
\begin{equation*}
    \left\{
    \begin{aligned}
    \dfrac{dx_1}{dt} &= x_1 - x_2 + 2u_1, \\
    \dfrac{dx_2}{dt} &= -4x_1 + x_2 + u_2,
    \end{aligned}
    \right.
\end{equation*}
де $x(0) = (x_{01}, x_{02}) \in \mathcal{M}_0$, $u(t) = (u_1(t), u_2(t)) \in\mathcal{U}$, $t\ge0$,
\begin{align*}
    \mathcal{M}_0 &= \{(x_{01},x_{02}): x_{01}^2 + x_{02}^2 \le 4\}, \\
    \mathcal{U} &= \{(u_1, u_2): u_1^2 + u_2^2 \le 1\}.
\end{align*}
\end{problem}

\begin{solution}
    Одразу помітимо, що $C=\begin{pmatrix}2&0\\0&1\end{pmatrix}$.\\

    $\Theta(t,s)$ знайдемо розв'язавши однорідну систему:
    \begin{equation*}
        \left\{
        \begin{aligned}
        \dfrac{dx_1}{dt} &= x_1 - x_2, \\
        \dfrac{dx_2}{dt} &= -4x_1 + x_2,
        \end{aligned}
        \right.
    \end{equation*}
    
    Її визначник $\begin{vmatrix} 1 - \lambda & - 1 \\ - 4 & 1 - \lambda \end{vmatrix} = (1 - \lambda)^2 - 4 = (\lambda + 1) (\lambda - 3) = 0$, звідки $\lambda_1 = -1$, $\lambda_2 = 3$. \\
    
    Підставляючи знайдені числа у систему, знаходимо власні вектори: $\begin{pmatrix} 1 \\ 2 \end{pmatrix}$ та $\begin{pmatrix} 1 \\ -2 \end{pmatrix}$ відповідно. \\

    Отже загальний розв'язок має вигляд \[\begin{pmatrix} x_1 \\ x_2 \end{pmatrix}(t) = c_1 \begin{pmatrix} e^{-t} \\ 2e^{-t} \end{pmatrix} + c_2 \begin{pmatrix} e^{3t} \\ -2e^{3t} \end{pmatrix}\]
    
    Розв'язуючи рівняння
    \[ c_1 \begin{pmatrix} e^{-s} \\ 2e^{-s} \end{pmatrix} + c_2 \begin{pmatrix} e^{3s} \\ -2e^{3s} \end{pmatrix} = \begin{pmatrix} 1 \\ 0 \end{pmatrix} \]
    і
    \[ c_1 \begin{pmatrix} e^{-s} \\ 2e^{-s} \end{pmatrix} + c_2 \begin{pmatrix} e^{3s} \\ -2e^{3s} \end{pmatrix} = \begin{pmatrix} 0 \\ 1 \end{pmatrix}, \]
    знаходимо фундаментальну матрицю системи, нормовану за моментом $s$, а саме 
    \[ \Theta(t,s) = \begin{pmatrix} \dfrac{e^{s-t} + e^{3(t-s)}}{2} & \dfrac{e^{s-t} - e^{3(t-s)}}{4} \\ e^{s-t} - e^{3(t-s)} & \dfrac{e^{s-t} + e^{3(t-s)}}{2} \end{pmatrix} \]
    
    
    Далі знаходимо $c(\mathcal{M}_0, \psi) = c(\mathcal{K}_2(0), \psi) = 2\|\psi\|$, та $c(\mathcal{U}, \psi) = c(\mathcal{K}_1(0), \psi) = \|\psi\|$, вже достатньо відомі нам опорні функції. \\
    
    Нарешті, можемо зібрати це все докупи: 
    \begin{align*}
        c(\mathcal{X}(t, \mathcal{M}_0), \psi) &= c(\mathcal{M}_0, \Theta^\star(t, 0) \psi) + \int_{0}^t c(\mathcal{U}(s), C^\star(s) \Theta^\star(t, s)\psi) ds = \\
        \\
        &= 2 \|\Theta^\star(t, 0) \psi\| + \int_{0}^t \left\|C^\star(s) \Theta^\star(t, s)\psi\right\| ds = \\
        \\
        &= 2 \left\|\begin{pmatrix} \dfrac{e^{-t} + e^{3t}}{2} & e^{3t} - e^{-t} \\ \dfrac{e^{3t} - e^{-t}}{4} & \dfrac{e^{-t} + e^{3t}}{2} \end{pmatrix} \begin{pmatrix} \psi_1 \\ \psi_2 \end{pmatrix}\right\| + \\
        \\
        &+ \int_{0}^t \left\|\begin{pmatrix} e^{s-t} + e^{3(t-s)} & 2(e^{3(t-s)} - e^{s-t}) \\ \dfrac{e^{3(t-s)} - e^{s-t}}{4} & \dfrac{e^{s-t} + e^{3(t-s)}}{2} \end{pmatrix} \begin{pmatrix} \psi_1 \\ \psi_2 \end{pmatrix}\right\| ds = \\
        \\
        &= 2 \left\| \begin{pmatrix} \dfrac{e^{-t} + e^{3t}}{2} \cdot \psi_1 + (e^{3t} - e^{-t}) \cdot \psi_2 \\ \dfrac{e^{3t} - e^{-t}}{4}\cdot\psi_1 + \dfrac{e^{-t} + e^{3t}}{2}\cdot\psi_2 \end{pmatrix} \right\| + ...
    \end{align*}
\end{solution}  \newpage

% OK, complete

\section{Задача про переведення системи з точки в точку. Критерії керованості лінійної системи керування}

\subsection{Аудиторне заняття}

\begin{problem}
	Перевести систему \[ \frac{\diff x}{\diff t} = u, \quad t \in [0, T], \] з точки $x (0) = x_0$ в точку $x (T) = y_0$ за допомогою керування з класу:
	\begin{enumerate}
		\item постійних функцій $u (t) = c$, $c$ -- константа;

		\item кусково-постійних функцій \[ u (t) = \begin{cases} c_1, & t \in [0, t_1], \\ c_2, & t \in [t_1, T]. \end{cases} \]

		Тут $c_1$, $c_2$ -- константи, $c_1 \ne c_2$, $0 < t_1 < T$;

		\item програмних керувань $u(t) = c t$, $c$ -- константа;

		\item керувань з оберненим зв'язком $u(x) = c x$, $c$ -- константа.
	\end{enumerate}
\end{problem}

\begin{solution}
	Скористаємося формулою $x (T) = x (0) + \int_0^T \frac{\diff x}{\diff t} \diff t$:

	\begin{enumerate}
		\item \[x (T) = x (0) + \int_0^T c \diff t = x (0) + c T,\] звідки \[c = \frac{x (T) - x (0)}{T} = \frac{y_0 - x_0}{T};\]

		\item \[ x (T) = x (0) + \int_0^{t_1} c_1 \diff t + \int_{t_1}^T c_2 \diff t = x (0) + c_1 t_1 + c_2 (T - t_1). \] Розв'язок не єдиний, \[ c_2 = \frac{x (T) - x (0) - c_1 t_1}{T - t_1}, \] де $c_1$ -- довільна стала, наприклад $c_1 = 0$, тоді \[ c_2 = \frac{x (T) - x (0)}{T - t_1} = \frac{y_0 - x_0}{T - t_1}. \]

		\item \[ x (T) = x (0) + \int_0^T c t \diff t = x (0) + \frac{cT^2}{2}, \] звідки \[ c = \frac{2 (x (T) - x (0))}{T^2} = \frac{2 (y_0 - x_0)}{T^2}. \]

		\item У цьому випадку проінтегрувати не можна, бо $u$ залежить від $x$, тому просто запишемо за формулою Коші \[ x (T) = x (0) \cdot e^{c T}, \] звідки \[c = \frac{\ln(x (T) / x (0))}{T} = \frac{\ln(y_0) - \ln(x_0)}{T}. \]

		Варто зауважити, не для всіх пар $x_0$ і $y_0$ коректно визначається значення $c$. А саме, необхідно щоб $y_0$ було того ж знаку, що і $x_0$.
	\end{enumerate}
\end{solution}

\begin{problem}
	\begin{enumerate}
		\item Використовуючи означення, знайти грамміан керованості для системи керування \[ \frac{\diff x(t)}{\diff t} = t x (t) + \cos (t) \cdot u(t), \quad t \ge 0. \]

		\item Записати диференціальне рівняння для грамміана керованості і за його допомогою знайти грамміан керованості.

		\item Використовуючи критерій керованості, вказати інтервал повної керованості вказаної системи керування. Для цього інтервала записати керування, яке розв'язує задачу про переведення системи з точки $x_0$ у стан $x_T$.
	\end{enumerate}
\end{problem}

\begin{solution}
	\begin{enumerate}
		\item Скористаємося формулою \[\Phi(T, t_0) = \int_{t_0}^T \Theta(T, s) B(s) B^*(s) \Theta^*(T, s) \diff s.\] $\Theta(T, s)$ знаходимо з системи \[ \frac{\diff \Theta(t,s)}{\diff t} = A(t) \cdot \Theta(t, s) = t \cdot \Theta(t, s),\] а саме $\Theta(t, s) = \exp\left\{\frac{t^2 - s^2}{2}\right\}$. Підставляючи всі знайдені значення, отримаємо \[\Phi(T, t_0) = \cos^2(T) \cdot e^{T^2} \int_{t_0}^T e^{-s^2} \diff s = \frac{1}{2} \sqrt{\pi} \cdot \cos^2(T) \cdot e^{T^2} \cdot \erf(T).\]

		\item Запишемо систему \[ \frac{\diff \Phi (t, t_0)}{\diff t} = A(t) \cdot \Phi(t, t_0) + \Phi(t, t_0) \cdot A^*(t) + B(t) \cdot B^*(t), \quad \Phi(t_0, t_0) = 0. \]

		І підставимо відомі значення: \[ \frac{\diff \Phi (t, 0)}{\diff t} = 2 t \Phi(t, 0)  + \cos^2(t), \quad \Phi(0, 0) = 0. \]

		Звідси \[ \Phi(t, 0) = \frac{1}{8} \sqrt{\pi} e^{t^2 - 1} ( - 2 e \erf(t) + i (\erfi(1 + i t) - i \erfi(1 - i t)), \]

		а \[ \Phi(T, 0) = \frac{1}{8} \sqrt{\pi} e^{T^2 - 1} ( - 2 e \erf(T) + i (\erfi(1 + i T) - i \erfi(1 - i T)), \]

		\item З вигляду грамміану керованості отриманого у першому пункті очевидно, що система цілком керована на півінтервалі $[0, \pi / 2)$, зокрема на інтервалі $[0, 1]$. \\

		Підставимо тепер граміан у формулу для керування що розв'язує задачу про переведення системи із стану $x_0$ у стан $x_T$:
		\begin{multline*} u(t) = B^*(t) \Theta^*(T, t) \Phi^{-1}(T, t_0) (x_T - \Theta(T, t_0) x_0) = \\ = \cos(t) \cdot \exp\left\{\frac{T^2-t^2}{2}\right\} \Phi^{-1} (T, 0) \left(x_T - \exp\left\{\frac{T^2}{2}\right\} x_0\right) \end{multline*}
	\end{enumerate}
\end{solution}

\begin{problem}
	За допомогою грамміана керованості розв'язати таку задачу оптимального керування: мінімізувати критерій якості \[ \JJ (u) = \int_0^T u^2(s) \diff s\] за умов, що \[ \frac{\diff x(t)}{\diff t} = \sin(t) \cdot x(t) + u(t), \quad x (0) = x_0, x (T) = x_T. \]

	Тут $x$ -- стан системи, $u(t)$ -- скалярне керування, $x_0$, $X_T$ -- задані точки, $t \in [t_0, T]$.
\end{problem}

\begin{solution}
	Знайдемо шукане керування за формулою \[ u(t) = B^*(t) \Theta^*(T, t) \Phi^{-1}(T, t_0) (x_T - \Theta(T, t_0) x_0). \]

	У цій задачі $\Theta(t, s) = e^{\cos (s) - \cos (t)}$, знайдене з системи $\dot \Theta = A \Theta$, $\Phi(T, t_0) = e^{-2 \cos (T)} \int_0^T e^{2 \cos (s)} \diff s$, підставляючи знаходимо \[ u(t) = \frac{e^{\cos (t) + \cos (T)} \cdot (x_T - e^{1 - \cos (T)} x_0)}{\int_0^T e^{2 \cos (s)} \diff s}. \]
\end{solution}

\begin{problem}
    За допомогою грамміана керованості розв'язати таку задачу оптимального керування: мінімізувати критерій якості
    \[ \mathcal{J}(u) = \int_0^T u^2(s) dx \]
    за умов, що
    \[ \dfrac{d^2x(t)}{dt^2} - 5\dfrac{dx(t)}{dt} + 6x(t) = u(t), \]
    \[ x(0) = x_0, x'(0) = y_0, x(T) = x'(T) = 0.\]
    Тут $x$ -- стан системми, $u(t)$ -- скалярне керування, $t \in [0, T]$.
\end{problem}

\begin{solution}
    Почнемо з того що зведемо рівняння другого порядку до системи рівнянь заміною $x_1 = x$, $x_2 = \dot x_1$, тоді маємо систему
    \[ \begin{pmatrix} \dot x_1 \\ \dot x_2 \end{pmatrix} (t) = \begin{pmatrix} 0 & 1 \\ -6 & 5 \end{pmatrix} \begin{pmatrix} x_1 \\ x_2 \end{pmatrix} (t) + \begin{pmatrix} 0 \\ 1 \end{pmatrix} u(t). \]
    
    Знайдемо власні числа матриці $A - \lambda E$: $\det(A - \lambda E) = \begin{vmatrix} -\lambda & 1 \\ -6 & 5-\lambda \end{vmatrix} = \lambda^2 - 5\lambda + 6 = (\lambda - 2) (\lambda - 3) = 0$, звідки $\lambda_1 = 2$, $\lambda_2 = 3$. Знайдемо власні вектори, вони будуть $\begin{pmatrix} 1 \\ 2 \end{pmatrix}$ і $\begin{pmatrix} 1 \\ 3 \end{pmatrix}$ відповідно. Звідси знаходимо загальний розв'язок
    \[ \begin{pmatrix} x_1 \\ x_2 \end{pmatrix} (t) = c_1 \begin{pmatrix} e^{2t} \\ 2e^{2t} \end{pmatrix} + c_2 \begin{pmatrix} e^{3t} \\ 3e^{3t} \end{pmatrix}. \]
    
    З рівняння
    \[c_1 \begin{pmatrix} e^{2s} \\ 2e^{2s} \end{pmatrix} + c_2 \begin{pmatrix} e^{3s} \\ 3e^{3s} \end{pmatrix} = \begin{pmatrix} 1 \\ 0 \end{pmatrix} \]
    знаходимо $c_1 = 3e^{-2s}$, $c_2 = -2e^{-3s}$, а з рівняння
    \[c_1 \begin{pmatrix} e^{2s} \\ 2e^{2s} \end{pmatrix} + c_2 \begin{pmatrix} e^{3s} \\ 3e^{3s} \end{pmatrix} = \begin{pmatrix} 0 \\ 1 \end{pmatrix} \]
    знаходимо $c_1 = -e^{-2s}$, $c_2 = e^{-3s}$, тобто
    \[ \Theta(t, s) = \begin{pmatrix} 3e^{2(t-s)} - 2e^{3(t-s)} & -e^{2(t-s)} + e^{3(t-s)} \\ 6e^{2(t-s)} - 6e^{3(t-s)} & -2e^{2(t-s)} + 3e^{3(t-s)} \end{pmatrix}. \]
    
    Знайдемо грамміан за формулою \[\Phi(T, 0) = \int_0^T \Theta(T, s) B(s) B^* (s) \Theta^*(T, s) ds. \]
    
    \[ \Theta(T, s) B(s) = \begin{pmatrix} -e^{2(T - s)} + e^{3(T - s)} \\ -2e^{2(T - s)} + 3e^{3(T - s)} \end{pmatrix}. \]
    \[ B^* (s) \Theta^*(T, s)  =  (\Theta(T, s) B(s))^\star = \begin{pmatrix} -e^{2(T - s)} + e^{3(T - s)} & -2e^{2(T - s)} + 3e^{3(T - s)} \end{pmatrix}. \]
    
    \begin{align*} 
        \Phi(T, 0) &= \int_0^T \begin{pmatrix} -e^{2(T - s)} + e^{3(T - s)} \\ -2e^{2(T - s)} + 3e^{3(T - s)} \end{pmatrix} \begin{pmatrix} -e^{2(T - s)} + e^{3(T - s)} & -2e^{2(T - s)} + 3e^{3(T - s)} \end{pmatrix} ds = \\
        &= \int_0^T \begin{pmatrix} e^{4(T - s)} - 2 e^{5(T - s)} + e^{6(T - s)} & 2 e^{4(T - s)} - 5 e^{5(T - s)} + 3 e^{6(T - s)} \\ 2e^{4(T - s)} - 5 e^{5(T - s)} + 3 e^{6(T - s)} & 4 e^{4(T - s)} - 12 e^{5(T - s)} + 9  e^{6(T - s)} \end{pmatrix} ds = \\
        &= \begin{pmatrix} \dfrac{e^{4T} - 1}{4} - \dfrac{2(e^{5T} - 1)}{5} + \dfrac{e^{6T} - 1}{6} & \dfrac{e^{4T} - 1}{2} - (e^{5T} - 1) + \dfrac{e^{6T} - 1}{2} \\ \\ \dfrac{e^{4T} - 1}{2} - (e^{5T} - 1) + \dfrac{e^{6T} - 1}{2} & (e^{4T} - 1) - \dfrac{12(e^{5T} - 1)}{5} + \dfrac{3(e^{6T} - 1)}{2} \end{pmatrix}
    \end{align*}
    
    Чесно кажучи вже обчислення визначника грамміану є надто складною обчислювальною задачею, не бачу сенсу її робити вручну.
\end{solution}

\begin{problem}
	Записати систему диференціальних рівнянь для знаходження першої матриці керованості (грамміана керованості) і сформулювати критерій керованості на інтервалі $[0, T]$ у випадку, якщо система керування має вигляд:
	\begin{enumerate}
		\item \[ 
		\left\{
			\begin{aligned}
				\frac{\diff x_1(t)}{\diff t} = t x_1 (t) + x_2 (t) + u_1 (t), \\
				\frac{\diff x_2(t)}{\diff t} = - x_1 (t) + 2 x_2 (t) + t^2 u_2 (t).
			\end{aligned}
		\right.
		\]

		Тут $x = (x_1, x_2)^*$ -- вектор стану, $u = (u_1, u_2)^*$ -- вектор керування, $t \in [0, T]$.

		\item \[ \frac{\diff^2 x(t)}{\diff t^2} + \sin(t) \cdot x(t) = u(t). \]

		Тут $x$ -- стан системи, $u(t)$ -- скалярне керування, $t \in [0, T]$.
	\end{enumerate}
\end{problem}

\begin{solution}
	\begin{enumerate}
		\item $A = \begin{pmatrix} t & 1 \\ -1 & 2 \end{pmatrix}$, $B = \begin{pmatrix} 1 & 0 \\ 0 & t^2 \end{pmatrix}$, \[
		\left\{
			\begin{aligned}
				\dot \phi_{11} &= 2 t \phi_{11} + 2 \phi_{12} + 1, \\
				\dot \phi_{12} &= - \phi_{11} + (t + 2) \phi_{12} + \phi_{22}, \\
				\dot \phi_{21} &= \ldots 
			\end{aligned}
		\right.
		\]

		\item Введемо нову змінну $x_2 = \dot x$, тоді $A = \begin{pmatrix} 0 & 1 \\ -\sin(t) & 0 \end{pmatrix}$, $B = \begin{pmatrix} 0 \\ 1 \end{pmatrix}$, \[
		\left\{
			\begin{aligned}
				\dot \phi_{11} &= 2 \phi_{12}, \\
				\dot \phi_{12} &= (1 - \sin(t)) \phi_{11}, \\
				\ldots 
			\end{aligned}
		\right.
		\]
	\end{enumerate}
\end{solution}

\begin{problem}
 	Знайти диференціальне рівняння грамміана керованості для системи керування \[ \left\{ \begin{aligned}
 		\frac{\diff x_1 (t)}{\diff t} &= \cos(t) \cdot x_1(t)-\sin(t) \cdot x_2(t) + u_1(t) - 2 u_2(t), \\
 		\frac{\diff x_2 (t)}{\diff t} &= \sin(t) \cdot x_1(t)+\cos(t) \cdot x_2(t) - 3 u_1(t) + 4 u_2(t).
 	\end{aligned} \right. \]
\end{problem}

\begin{solution}
 	% 3.6
\end{solution}


\begin{problem}
    Дослідити системи на керованість. використовуючи другий критерій керованості:
    \begin{enumerate}
        \item \[\ddot x + a \dot x + b x = u; \]
        \item \[ \left\{ \begin{aligned} \dot x_1 &= 2x_1 + x_2 + au \\ \dot x_2 &= x_1 + 4 x_2 + u \end{aligned} \right. \]
        \item \[ \left\{ \begin{aligned} \dot x_1 &= 2x_1 + x_2 + u_1 \\ \dot x_2 &= x_1 + 3 x_3 + u_2 \\ \dot x_3 &= x_2 + x_3 + u_2  \end{aligned} \right. \]
    \end{enumerate}
\end{problem}

\begin{solution}
    \begin{enumerate}
        \item Почнемо з того що зведемо рівняння другого порядку до системи рівнянь заміною $x_1 = x$, $x_2 = \dot x_1$, тоді маємо систему
        \[ \left\{ \begin{aligned} \dot x_1 &= x_2 \\ \dot x_2 &= - a x_2 - b x_1 + u  \end{aligned} \right. \]
        Тоді
        \[ A = \begin{pmatrix} 0 & 1 \\ -b & -a \end{pmatrix} \qquad B = \begin{pmatrix} 0 \\ 1 \end{pmatrix}. \]
        \[ D = \begin{pmatrix} B & AB \end{pmatrix} = \begin{pmatrix} 0 & 1 \\ 1 & - a \end{pmatrix}. \]
        Її ранг дорівнює 2 якщо за будь-яких $a$ і $b$, тобто система завжди цілком керована.
        \item 
        \[ A = \begin{pmatrix} 2 & 1 \\ 1 & 4 \end{pmatrix} \qquad B = \begin{pmatrix} a \\ 1 \end{pmatrix}. \]
        \[ D = \begin{pmatrix} B & AB \end{pmatrix} = \begin{pmatrix} a & 2 a + 1 \\ 1 & a + 4 \end{pmatrix}. \]
        Її визначник $a^2 + 4a - 2a - 1 = a^2 + 2a - 1 = 0$ якщо $a = -1 \pm \sqrt 2$, тоді система не є цілком керованою, а інакше є.
        \item 
        \[ A = \begin{pmatrix} 2 & 1 & 0 \\ 1 & 0 & 3 \\ 0 & 1 & 1 \end{pmatrix} \qquad B = \begin{pmatrix} 1 & 0 \\ 0 & 1 \\ 0 & 1  \end{pmatrix}. \]
        \[ D = \begin{pmatrix} B & AB & A^2B \end{pmatrix} = \begin{pmatrix} 1 & 0 & 2 & 1 & 5 & 5 \\ 0 & 1 & 1 & 3 & 2 & 7 \\ 0 & 1 & 0 & 2 & 1 & 5 \end{pmatrix} .\]
        Її ранг дорівнює 3, тобто система цілком керована.
    \end{enumerate}    
\end{solution} \setcounter{section}{2}

\section{Домашнє завдання за 9/21}

\setcounter{problem}{7}

\begin{problem}
    Перевести систему 
    \[ \dfrac{dx}{dt} = 2tx + u, t \in [0, T], \]
    з точки $x(0) = x_0$ в точку $x(T) = y_0$ за допомогою керування з класу:
    \begin{enumerate}
        \item постійних функцій $u(t) = c$, $c$ -- константа; 
        \item кусково-постійних функцій 
        \[
        u(t) = \begin{cases}
            c_1 & t \in [0, t_1), \\
            c_2 & t \in (t_1, T].
        \end{cases}
        \]
        Тут $c_1$, $c_2$ -- константи, $c_1 \ne c_2$, $0 < t_1 < T$;
        \item програмних керувань $u(t) = ct$, $c$ -- константа;
        \item керувань з оберненим зв'язок $u(x) = cx$, $c$ -- константа.
    \end{enumerate}
\end{problem}

\begin{solution}
Будемо просто підставляти керування у диференційне рівняння і розв'язувати його:
\begin{enumerate}
\item Зводимо до канонічного вигляду лінійного рівняння:
\[ \dfrac{dx}{dt} - 2t \cdot x(t) = c. \]
Домножаємо на множник що інтегрує:
\begin{align*}
    \exp\{-t^2\} \cdot \dfrac{dx}{dt} - 2 t \cdot \exp\{-t^2\} \cdot x(t) &= c \cdot \exp\{-t^2\} \\
    \\
    \exp\{-t^2\} \cdot \dfrac{dx}{dt} + x(t) \cdot \dfrac {d \exp\{-t^2\}} {dt} &= c \cdot \exp\{-t^2\}.
\end{align*}
Згортаємо похідну добутку:
\[ \dfrac {d (\exp\{-t^2\} \cdot x(t))} {dt} = c \cdot \exp\{-t^2\}. \]
Інтегруємо:
\begin{align*}
    \left.(\exp\{-t^2\} \cdot x(t))\right|_0^T &= \Int_0^T c \cdot \exp\{-t^2\} dt \\
    \\
    \exp\{-T^2\} \cdot y_0 - x_0 &= c \cdot \dfrac {\sqrt \pi} 2 \cdot \erf (T),
\end{align*}
і виражаємо звідси $c$:
\[ c = 2 \cdot \dfrac{\exp\{-T^2\} \cdot y_0 - x_0} {\sqrt \pi \cdot \erf (T)}, \]
де $\erf$ позначає функцію помилок, тобто $\erf(x) = \dfrac 2 {\sqrt \pi} \cdot \Int_0^x \exp\{-t^2\} dt$.\\

Зауважимо, що задача має розв'язок завжди.
\item Нескладно зрозуміти, що нас задовольнить довільне керування вигляду
\[ c_2 = 2 \cdot \dfrac{\exp\{-T^2\} \cdot y_0 - \exp\{-t_1^2\} \cdot x_1} {\sqrt \pi \cdot (\erf (T) - \erf (t_1))},\] де \[ x_1 = \dfrac{2x_0 + c_1\sqrt \pi \cdot \erf (T)}{2\cdot \exp\{-t_1^2\}}, \]
тобто  ми просто дозволили $c_1$ бути довільною сталою, обчислили $x(t_1)$, а потім розв'язали задачу переведення системи з точки $(t_1, x_1)$ у точку $(T, y_0)$ як у першому пункті, з мінімальними поправками на межі інтегрування. \\

Зокрема, якщо $c_1 = 0$, то $x_1 = \dfrac {x_0} {\exp\{-t_1^2\}}$, тому $c_2 = 2 \cdot \dfrac{\exp\{-T^2\} \cdot y_0 - x_0} {\sqrt \pi \cdot (\erf (T) - \erf (t_1))}$.\\

Зауважимо, що задача має розв'язок завжди.
\item 
\begin{align*}
    \dfrac{dx}{2x(t) + c} &= t dt \\
    \\
    \Int_0^T \dfrac{dx}{2x(t) + c} &= \Int_0^T t dt \\
    \\
    \left.\left(\dfrac 12 \ln(2x(t) + c)\right)\right|_0^T &= \dfrac {T^2} 2 \\
    \\
    \ln(2y_0 + c) - \ln(2x_0 + c) &= T^2 \\
    \\
    \ln\left(\dfrac{2y_0 + c}{2x_0 + c}\right) &= T^2 \\
    \\
    \dfrac{2y_0 + c}{2x_0 + c} &= \exp\{T^2\} \\
    \\
    2y_0 + c &= (2x_0 + c)\cdot \exp\{T^2\} \\
    \\
    2(y_0 - x_0 \cdot \exp\{T^2\}) &= c \cdot (\exp\{T^2\} - 1) 
\end{align*}
звідки
\[c = 2\cdot \dfrac{y_0 - x_0 \cdot \exp\{T^2\}}{\exp\{T^2\} - 1}. \]
Зауважимо, що задача має розв'язок завжди.
\item 
\begin{align*}
    \dfrac{dx}{dt} &= 2t\cdot x(t) + c\cdot x(t) \\
    \\
    \dfrac{dx}{x(t)} &= (2t + c) dt \\
    \\
    \Int_0^T \dfrac{dx}{x(t)} &= \Int_0^T (2t + c) dt \\
    \\
    (\ln(x(t))|_0^T &= T^2 + cT \\
    \\
    \ln(y_0) - \ln(x_0) &= T^2 + cT \\ 
    \\
    \ln\left(y_0/x_0\right) &= T^2 + cT 
\end{align*}
звідки
\[ c = \dfrac{\ln\left(y_0/x_0\right) - T^2}{T}. \]
Зауважимо, що задача має розв'язок тільки якщо $\signum(x_0) = \signum(y_0)$.
\end{enumerate}
\end{solution} 

\begin{problem}
\begin{enumerate}
    \item Знайти грамміан керованості для системи керування \[\dfrac{dx(t)}{dt} = tx(t) + u(t)\] і дослідити її на керованість, використовуючи перший критерій керованості.
    \item За допомогою грамміана керованості розв'язати таку задачу оптимального керування: \[\mathcal{J}(u) = \Int_0^T u^2(s) ds \to \min\]
    за умов, що \[\dfrac{dx(t)}{dt} = tx(t) + u(t), x(0) = x_0, x(T) = x_T. \]
    Тут $x$ -- стан системи. $u(t)$ -- скалярне керування, $x_0$, $x_T$ -- задані точки, $t\in[0,T]$.
\end{enumerate}

\end{problem}

\begin{solution}
\begin{enumerate}
    \item Одразу помітимо, що $A(t) = (t)$, $B(t) = (1)$. Далі, з рівняння $\dfrac{d\Theta(t,s)}{dt} = A(t) \cdot \Theta(t,s)$ знаходимо $\Theta(t,s) = \exp\{t^2 / 2 - s^2 / 2\}$. Залишилося всього нічого, знайти власне грамміан:
    \[ \Phi(T,0) = \Int_0^T \Theta(T,s)B(s)B^\star(s),\Theta^\star(T,s) ds = \Int_0^T (\exp\{T^2 - s^2\}) ds = \begin{pmatrix}\dfrac{\sqrt \pi}{2} \cdot \exp\{T^2\} \cdot \erf(T) \end{pmatrix}, \] і $\det\Phi(T,0)\ne0$, тобто система цілком керована на $[0, T]$.
    \item Пригадаємо наступний результат: розв'язком вищезгаданої задачі про оптимальне керування є функція
    \begin{align*}
        u(t) &= B^\star(t) \Theta^\star(T,t)\Phi^{-1}(T,0)(x_T-\Theta(T,0) x_0) = \\
        \\
        &= \exp\{T^2 / 2 - t^2 / 2\} \begin{pmatrix}\dfrac2{\sqrt \pi} \cdot \exp\{-T^2\} \cdot \dfrac1{\erf(T)} \end{pmatrix} (x_T - \exp\{T^2 / 2\} x_0) = \\
        \\
        &= \dfrac{2}{\sqrt{\pi}\cdot \erf(T)} \cdot \left(x_T \cdot \exp\left\{-\dfrac{T^2+t^2}2\right\} - x_0 \cdot \exp\left\{- \dfrac{t^2}2\right\}\right).
    \end{align*} 
\end{enumerate}
\end{solution} 

\setcounter{section}{2}

\setcounter{problem}{7}

\begin{problem}
Знайти опорну функцію множини досяжності для системи керування:
\begin{equation*}
    \left\{
    \begin{aligned}
    \dfrac{dx_1}{dt} &= x_1 - x_2 + 2u_1, \\
    \dfrac{dx_2}{dt} &= -4x_1 + x_2 + u_2,
    \end{aligned}
    \right.
\end{equation*}
де $x(0) = (x_{01}, x_{02}) \in \mathcal{M}_0$, $u(t) = (u_1(t), u_2(t)) \in\mathcal{U}$, $t\ge0$,
\begin{align*}
    \mathcal{M}_0 &= \{(x_{01},x_{02}): x_{01}^2 + x_{02}^2 \le 4\}, \\
    \mathcal{U} &= \{(u_1, u_2): u_1^2 + u_2^2 \le 1\}.
\end{align*}
\end{problem}

\begin{solution}
    Одразу помітимо, що $C=\begin{pmatrix}2&0\\0&1\end{pmatrix}$.\\

    $\Theta(t,s)$ знайдемо розв'язавши однорідну систему:
    \begin{equation*}
        \left\{
        \begin{aligned}
        \dfrac{dx_1}{dt} &= x_1 - x_2, \\
        \dfrac{dx_2}{dt} &= -4x_1 + x_2,
        \end{aligned}
        \right.
    \end{equation*}
    
    Її визначник $\begin{vmatrix} 1 - \lambda & - 1 \\ - 4 & 1 - \lambda \end{vmatrix} = (1 - \lambda)^2 - 4 = (\lambda + 1) (\lambda - 3) = 0$, звідки $\lambda_1 = -1$, $\lambda_2 = 3$. \\
    
    Підставляючи знайдені числа у систему, знаходимо власні вектори: $\begin{pmatrix} 1 \\ 2 \end{pmatrix}$ та $\begin{pmatrix} 1 \\ -2 \end{pmatrix}$ відповідно. \\

    Отже загальний розв'язок має вигляд \[\begin{pmatrix} x_1 \\ x_2 \end{pmatrix}(t) = c_1 \begin{pmatrix} e^{-t} \\ 2e^{-t} \end{pmatrix} + c_2 \begin{pmatrix} e^{3t} \\ -2e^{3t} \end{pmatrix}\]
    
    Розв'язуючи рівняння
    \[ c_1 \begin{pmatrix} e^{-s} \\ 2e^{-s} \end{pmatrix} + c_2 \begin{pmatrix} e^{3s} \\ -2e^{3s} \end{pmatrix} = \begin{pmatrix} 1 \\ 0 \end{pmatrix} \]
    і
    \[ c_1 \begin{pmatrix} e^{-s} \\ 2e^{-s} \end{pmatrix} + c_2 \begin{pmatrix} e^{3s} \\ -2e^{3s} \end{pmatrix} = \begin{pmatrix} 0 \\ 1 \end{pmatrix}, \]
    знаходимо фундаментальну матрицю системи, нормовану за моментом $s$, а саме 
    \[ \Theta(t,s) = \begin{pmatrix} \dfrac{e^{s-t} + e^{3(t-s)}}{2} & \dfrac{e^{s-t} - e^{3(t-s)}}{4} \\ e^{s-t} - e^{3(t-s)} & \dfrac{e^{s-t} + e^{3(t-s)}}{2} \end{pmatrix} \]
    
    
    Далі знаходимо $c(\mathcal{M}_0, \psi) = c(\mathcal{K}_2(0), \psi) = 2\|\psi\|$, та $c(\mathcal{U}, \psi) = c(\mathcal{K}_1(0), \psi) = \|\psi\|$, вже достатньо відомі нам опорні функції. \\
    
    Нарешті, можемо зібрати це все докупи: 
    \begin{align*}
        c(\mathcal{X}(t, \mathcal{M}_0), \psi) &= c(\mathcal{M}_0, \Theta^\star(t, 0) \psi) + \Int_{0}^t c(\mathcal{U}(s), C^\star(s) \Theta^\star(t, s)\psi) ds = \\
        \\
        &= 2 \|\Theta^\star(t, 0) \psi\| + \Int_{0}^t \left\|C^\star(s) \Theta^\star(t, s)\psi\right\| ds = \\
        \\
        &= 2 \left\|\begin{pmatrix} \dfrac{e^{-t} + e^{3t}}{2} & e^{3t} - e^{-t} \\ \dfrac{e^{3t} - e^{-t}}{4} & \dfrac{e^{-t} + e^{3t}}{2} \end{pmatrix} \begin{pmatrix} \psi_1 \\ \psi_2 \end{pmatrix}\right\| + \\
        \\
        &+ \Int_{0}^t \left\|\begin{pmatrix} e^{s-t} + e^{3(t-s)} & 2(e^{3(t-s)} - e^{s-t}) \\ \dfrac{e^{3(t-s)} - e^{s-t}}{4} & \dfrac{e^{s-t} + e^{3(t-s)}}{2} \end{pmatrix} \begin{pmatrix} \psi_1 \\ \psi_2 \end{pmatrix}\right\| ds = \\
        \\
        &= 2 \left\| \begin{pmatrix} \dfrac{e^{-t} + e^{3t}}{2} \cdot \psi_1 + (e^{3t} - e^{-t}) \cdot \psi_2 \\ \dfrac{e^{3t} - e^{-t}}{4}\cdot\psi_1 + \dfrac{e^{-t} + e^{3t}}{2}\cdot\psi_2 \end{pmatrix} \right\| + \\
    \end{align*}
    
\end{solution}  \newpage

\input{aw-04.tex} % OK, incomplete, missing 4.7, 4.12, 4.13

\subsection{Домашнє завдання}

\begin{problem}
    % 4.7
\end{problem}

\begin{solution}
    % 4.7
\end{solution}

\begin{problem}
    Записати диференціальне рівняння для знаходження грамміана спостережуваності системи
    \begin{equation*}
        \left\{
            \begin{aligned}
                \dfrac{dx_1(t)}{dt} &= x_1(t) + x_2(t), \\
                \dfrac{dx_2(t)}{dt} &= - t^2x_2(t), \\
                y(t) &= \sin (t) \cdot x_1(t) + \cos (t) \cdot x_2(t).
            \end{aligned}   
        \right.
    \end{equation*}
    Тут $x = (x_1, x_2)^\star$ -- вектор фазових координат, $y$ -- скалярне спостереження.
\end{problem}

\begin{solution}
    Почнемо з того, що $A = \begin{pmatrix} 1 & 1 \\ 0 & -t^2 \end{pmatrix}$, $H = \begin{pmatrix} \sin(t) & \cos(t) \end{pmatrix}$, $A^\star = \begin{pmatrix} 1 & 0 \\ 1 & -t^2 \end{pmatrix}$, $H^\star = \begin{pmatrix} \sin(t) \\ \cos(t) \end{pmatrix}$.\\
    
    Тоді диференціальне рівняння для знаходження грамміана спостережуваності набуває вигляду
    \[ \dfrac{\diff\NN(t, t_0)}{dt} = -\begin{pmatrix} 1 & 0 \\ 1 & -t^2 \end{pmatrix} \NN(t, t_0) - \NN(t, t_0) \begin{pmatrix} 1 & 1 \\ 0 & -t^2 \end{pmatrix} + \begin{pmatrix} \sin(t) \\ \cos(t) \end{pmatrix} \begin{pmatrix} \sin(t) & \cos(t) \end{pmatrix}. \]
    Або, що те саме,
    \begin{multline*}
        \begin{pmatrix} \dot n_{11} & \dot n_{12} \\ \dot n_{12} & \dot n_{22} \end{pmatrix} (t, t_0) = -\begin{pmatrix} 1 & 0 \\ 1 & -t^2 \end{pmatrix} \begin{pmatrix} n_{11} & n_{12} \\ n_{12} & n_{22} \end{pmatrix} (t, t_0) - \\ - \begin{pmatrix} n_{11} & n_{12} \\ n_{12} & n_{22} \end{pmatrix}(t, t_0) \begin{pmatrix} 1 & 1 \\ 0 & -t^2 \end{pmatrix} + \begin{pmatrix} \sin^2(t) & \sin(t)\cdot \cos(t) \\ \sin(t) \cdot \cos(t) & \cos^2(t) \end{pmatrix}.
    \end{multline*} 
    
    \begin{multline*}
        \begin{pmatrix} \dot n_{11} & \dot n_{12} \\ \dot n_{12} & \dot n_{22} \end{pmatrix} (t, t_0) = - \begin{pmatrix} n_{11} & n_{12} \\ n_{11} - t^2n_{12} & n_{12} - t^2n_{22} \end{pmatrix} (t, t_0) - \\ 
        - \begin{pmatrix} n_{11} & n_{11} - t^2n_{12} \\ n_{12} & n_{12} - t^2n_{22} \end{pmatrix}(t, t_0) + \begin{pmatrix} \sin^2(t) & \sin(t)\cdot \cos(t) \\ \sin(t) \cdot \cos(t) & \cos^2(t) \end{pmatrix}.
    \end{multline*} 
    
    \begin{equation*}
        \left\{
            \begin{aligned}
                \dot n_{11} (t, t_0) &= - 2n_{11} (t, t_0) + \sin^2(t) \\
                \dot n_{12} (t, t_0) &= - n_{11} (t, t_0) + (t^2 - 1) n_{12} (t, t_0) + \sin(t)\cdot \cos(t) \\
                \dot n_{22} (t, t_0) &= - 2n_{12} (t, t_0) + 2t^2 n_{22} (t, t_0) + \cos^2(t)
            \end{aligned}
        \right.
    \end{equation*}
\end{solution}

\begin{problem}
    Чи буде система цілком спостережуваною?
    \begin{enumerate}
        \item \[\ddot x = a^2 x, \quad y(t) = p\dot x(t); \]
        \item \begin{equation*}
            \left\{
                \begin{aligned}
                    \dot x_1 &= 2x_1 + \alpha x_2, \\
                    \dot x_2 &= - \alpha x_1 - \alpha x_2, \\
                    y(t) &= x_1 + \beta x_2.
                \end{aligned}
            \right.
        \end{equation*}
        \item \begin{equation*}
            \left\{
                \begin{aligned}
                    \dot x_1 &= x_2 - 2 x_3, \\
                    \dot x_2 &= x_1 - x_3, \\
                    \dot x_3 &= - 2 x_3, \\
                    y(t) &= - x_1 + x_2 - x_3.
                \end{aligned}
            \right.
        \end{equation*}
    \end{enumerate}
\end{problem}

\begin{solution}
    \begin{enumerate}
        \item Почнемо з того, що $A = \begin{pmatrix} 0 & 1 \\ a^2 & 0 \end{pmatrix}$, $H = \begin{pmatrix} 0 & p \end{pmatrix}$. Матриці стаціонарні, тому застосуємо другий критерій спостережуваності:
        \[ R = \begin{pmatrix} H^\star & A^\star H^\star \end{pmatrix} = \begin{pmatrix} 0 & a^2 \\ p & 0 \end{pmatrix}. \]
        Як бачимо, ранг 2, тобто система є цілком спостережуваною, якщо тільки $a \ne 0$ і $p \ne 0$.
        \item Почнемо з того, що $A = \begin{pmatrix} 2 & \alpha \\ -\alpha & -\alpha \end{pmatrix}$, $H = \begin{pmatrix} 1 & \beta \end{pmatrix}$. Матриці стаціонарні, тому застосуємо другий критерій спостережуваності:
        \[ R = \begin{pmatrix} H^\star & A^\star H^\star \end{pmatrix} = \begin{pmatrix} 1 & 2 - \alpha\beta \\ \beta & \alpha - \alpha\beta \end{pmatrix}. \]
        $\det R = \alpha - 2\beta - \alpha\beta + \alpha\beta^2 \ne 0$ (тобто система є спостережуваною), якщо тільки $\alpha \ne \dfrac{2\beta}{1-\beta+\beta^2}$.
        \item Почнемо з того, що $A = \begin{pmatrix} 0 & 1 & -2 \\ 1 & 0 & - 1 \\ 0 & 0 & -2 \end{pmatrix}$, $H = \begin{pmatrix} -1 & 1 & -1 \end{pmatrix}$. Матриці стаціонарні, тому застосуємо другий критерій спостережуваності:
        \[ R = \begin{pmatrix} H^\star & A^\star H^\star & (A^\star)^2 H^\star \end{pmatrix} = \begin{pmatrix} -1 & 1 & -1 \\ 1 & -1 & 1 \\ -1 & 3 & -7 \end{pmatrix}.\]
        Як бачимо, ранг 2 а не 3, тому система не є цілком спостережуваною.
    \end{enumerate}
\end{solution}

\begin{problem}
    Для яких параметрів $a$, $b$ система
    \begin{equation*}
        \left\{
            \begin{aligned}
                \dfrac{dx_1(t)}{dt} &= ax_1(t), \\
                \dfrac{dx_2(t)}{dt} &= bx_2(t), \\
                y(t) &= x_1(t) + x_2(t)
            \end{aligned}
        \right.
    \end{equation*}
    є цілком спостережуваною? Тут $x = (x_1, x_2)^\star$ -- вектор фазових координат, $y$ -- скалярне спостереження.
\end{problem}

\begin{solution}
    Почнемо з того, що $A = \begin{pmatrix} a & 0 \\ 0 & b \end{pmatrix}$, $H = \begin{pmatrix} 1 & 1 \end{pmatrix}$. Матриці стаціонарні, тому застосуємо другий критерій спостережуваності:
        \[ R = \begin{pmatrix} H^\star & A^\star H^\star \end{pmatrix} = \begin{pmatrix} 1 & a \\ 1 & b \end{pmatrix}. \]
        $\det R = b - a$, тобто система є цілком керованою якщо тільки $a \ne b$.
\end{solution}

\begin{problem}
    Побудувати спостерігач у загальному вигляді для такої системи:
    \begin{enumerate}
        \item \begin{equation*}
            \left\{
                \begin{aligned}
                    \dot x_1 &= x_1 + t^2 x_2, \\
                    \dot x_2 &= 2x_1 - 3x_2, \\
                    y(t) &= bx_1(t) + x_2(t).
                \end{aligned}
            \right.
        \end{equation*}
        \item \begin{equation*}
            \left\{
                \begin{aligned}
                    & \dfrac{d^2x}{dt^2} + k_1 \dfrac{dx}{dt} + k_2 x = 0, \\
                    & y(t) = x(t) + \beta \dfrac{dx(t)}{dt}.
                \end{aligned}
            \right.
        \end{equation*}
    \end{enumerate}
\end{problem}

\begin{solution}
    \begin{enumerate}
        \item Почнемо з того, що $A = \begin{pmatrix} 1 & t^2 \\ 2 & -3 \end{pmatrix}$, $H = \begin{pmatrix} b & 1 \end{pmatrix}$. Далі пишемо
        \[ \begin{pmatrix} \hat x_1 \\ \hat x_2 \end{pmatrix}^\prime (t) = \begin{pmatrix} 1 & t^2 \\ 2 & -3 \end{pmatrix} \begin{pmatrix} \hat x_1 \\ \hat x_2 \end{pmatrix} (t) + \begin{pmatrix} k_1 \\ k_2 \end{pmatrix} (t) \left( y(t) - \begin{pmatrix} b & 1 \end{pmatrix} \begin{pmatrix} \hat x_1 \\ \hat x_2 \end{pmatrix} (t) \right) .\]
        
        \begin{equation*}
            \left\{
                \begin{aligned}
                    \hat x_1^\prime (t) &= \hat x_1 (t) + t^2 \hat x_2 (t) + k_1(t) (y(t) - b \hat x_1(t) - \hat x_2(t)) \\
                    \hat x_2^\prime (t) &= 2 \hat x_1 (t) - 3 \hat x_2 (t) + k_2(t) (y(t) - b \hat x_1(t) - \hat x_2(t))
                \end{aligned}
            \right.
        \end{equation*}
        
        \item Введемо заміну $x_1 = x$, $x_2 = \dot x$, тоді маємо $A = \begin{pmatrix} 0 & 1 \\ - k_2 & - k_1 \end{pmatrix}$, $H = \begin{pmatrix} 1 & b \end{pmatrix}$. Далі пишемо
        \[ \begin{pmatrix} \hat x_1 \\ \hat x_2 \end{pmatrix}^\prime (t) = \begin{pmatrix} 0 & 1 \\ - k_2 & - k_1 \end{pmatrix} \begin{pmatrix} \hat x_1 \\ \hat x_2 \end{pmatrix} (t) + \begin{pmatrix} K_1 \\ K_2 \end{pmatrix} (t) \left( y(t) - \begin{pmatrix} 1 & b \end{pmatrix} \begin{pmatrix} \hat x_1 \\ \hat x_2 \end{pmatrix} (t) \right) .\]
        
        \begin{equation*}
            \left\{
                \begin{aligned}
                    \hat x_1^\prime (t) &= \hat x_2 (t) + K_1 (t) (y(t) - \hat x_1(t) - b \hat x_2 (t)) \\
                    \hat x_2^\prime (t) &= - k_2 \hat x_1(t) - k_1 \hat x_2 (t) + K_2 (t) (y(t) - \hat x_1(t) - b \hat x_2 (t))
                \end{aligned}
            \right.
        \end{equation*}
    \end{enumerate}
\end{solution}

\begin{problem}
    % 4.12
\end{problem}

\begin{solution}
    % 4.12
\end{solution}

\begin{problem}
    % 4.13
\end{problem}

\begin{solution}
    % 4.13
\end{solution}
 \newpage

\section{Задача фільтрації. Множинний підхід}

\subsection*{Аудиторне заняття}

\begin{problem}
	Задана динамічна система \[ \left\{ \begin{aligned}
		\frac{\diff x(t)}{\diff t} &= t x(t) + v(t), \\
		y(t) &= p x(t) + w(t),
	\end{aligned} \right. \]
	де $x(t) \in \RR^1$ -- вектор стану, $v(t) \in \RR^1$, $w(t) \in \RR^1$ -- невідомі шуми, $x_0 \in \RR^1$ -- невідома початкова умова, $y(t) \in \RR^1$ -- відомі спостереження. Побудувати інформаційну множину такої системи в момент $\tau \in [0, T]$ за умови, що \[ \int_0^\tau (v^2(s) + w^2(s)) \diff s + x^2(0) \le 1, \quad \tau \in [0, T]. \]
\end{problem}

\begin{solution}
	Загальна постановка задачі фільтрації має вигляд \[ \dot x = A x + v, \quad y = G x + w, \quad \int ( \langle Mv, v\rangle + \langle Nw, w\rangle )  + \langle p_0 x, x \rangle \le \mu^2. \] У нашій задачі $A = \begin{pmatrix} t \end{pmatrix}$, $G = \begin{pmatrix} p \end{pmatrix}$, $M = \begin{pmatrix} 1 \end{pmatrix}$, $N = \begin{pmatrix} 1 \end{pmatrix}$, $p_0 = 1$, $\mu = 1$. \\

	Знайдемо фільтр (спостерігач) цієї задачі у вигляді \[ \dot{\hat{x}} = A \hat x + K (y - G \hat x), \] де $K = R G^* N$, де у свою чергу $\dot R = A T + T A^* - R G^* N G R$, $R(t_0) = p_0^{-1}$. \\
\end{solution}

\begin{problem}
	% 5.2
\end{problem}

\begin{solution}
	% 5.2
\end{solution}

\begin{problem}
	% 5.3
\end{problem}

\begin{solution}
	% 5.3
\end{solution}
 \subsection{Домашнє завдання}

\begin{problem}
    Задана динамічна система
    \[ \left\{ \begin{aligned}
        \dot x &= x + v + t^2, \\
        y &= -x + w,
    \end{aligned} \right. \]
    де $x(t) \in \RR^1$ -- вектор стану, $v(t)\in\RR^1$, $w(t)\in\RR^1$ -- невідомі шуми, $y(t)\in\RR^1$ -- відомі спостереження. Побудувати інформаційну множину такої системи в момент $\tau\in[0, T]$ за умови. що 
    \[ \int_0^\tau (v^2(s) + w^2(s)) \diff s + (x(0) - 1)^2 \le 1, \quad \tau \in [0, T]. \]
\end{problem}

\begin{solution}
	% 5.4
\end{solution}

\begin{problem}
    Побудувати оцінку стану системи
    \[ \left\{ \begin{aligned}
        \dot x &= tx + v, \\
        y &= px + w,
    \end{aligned} \right. \]
    у формі фільтра, де $y(t)\in\RR^1$ -- відомі спостереження. Тут $x(t) \in \RR^1$ -- вектор стану, $v(t)\in\RR^1$, $w(t)\in\RR^1$ -- невідомі шуми, $x_0$ -- невідома початкова умова, при цьому
    \[ \int_0^\tau (v^2(s) + w^2(s)) \diff s + (x(0) - 1)^2 \le 4, \]
    $\tau \in [0, T]$. Знайти похибку оцінювання.
\end{problem}

\begin{solution}
	Загальна постановка задачі фільтрації має вигляд \[ \dot x (t)= A (t) \cdot x (t)+ v (t), \quad y (t)= G (t) \cdot x(t) + w(t),\] \[\int_{t_0}^t ( \langle M(t) \cdot v(t), v(t)\rangle + \langle N(t) \cdot w(t), w(t)\rangle )  + \langle p_0 x(t_0), x(t_0) \rangle \le \mu^2. \] У нашій задачі $A (t)= \begin{pmatrix} t \end{pmatrix}$, $G (t)= \begin{pmatrix} p \end{pmatrix}$, $M (t)= \begin{pmatrix} 1 \end{pmatrix}$, $N (t)= \begin{pmatrix} 1 \end{pmatrix}$, $p_0 = 1$, $\mu = 2$. \\

    Знайдемо фільтр (спостерігач) цієї задачі у вигляді \[ \dot{\hat{x}} (t) = A (t) \cdot \hat x (t) + K (t) \cdot (y (t) - G (t) \cdot \hat x (t)), \quad \hat x(0) = 1, \] де $K (t)= R (t) \cdot G^* (t) \cdot N(t)$, де у свою чергу \[\dot R (t)= A (t) \cdot R (t)+ R (t) \cdot A^* (t)- R (t) \cdot G^* (t) \cdot N (t) \cdot G (t) \cdot R(t), \quad R(t_0) = p_0^{-1}. \]

    Підставляючи відомі функції знаходимо \[\dot R (t)= 2 t \cdot R(t) - p^2 \cdot R^2(t), \quad R(t_0) = 1. \]

    Це рівняння Бернуллі, його розв'язок \[ R(t) = \frac{2 e^{t^2}}{2 e^{t_0^2} + p^2 \sqrt{\pi} (\erfi(t) - \erfi(t_0))}. \]

    Далі, \[ K(t) = \frac{2 p e^{t^2}}{2 e^{t_0^2} + p^2 \sqrt{\pi} (\erfi(t) - \erfi(t_0))}, \] і \[ \dot{\hat{x}} (t) = t \cdot \hat x (t) + \frac{2 p e^{t^2} \cdot (y (t) - p \cdot \hat x (t))}{2 e^{t_0^2} + p^2 \sqrt{\pi} (\erfi(t) - \erfi(t_0))}, \quad \hat x(0) = 1. \]

    Похибка $e(\tau)$ оцінювання задовольняє оцінці \[ |e(\tau)| \le \sqrt{\mu^2-k(\tau)} \cdot \sqrt{\lambda_* (R(\tau))} = \frac{ \sqrt{4 - k(\tau)} \cdot e^{2\tau}}{2e^{2\tau}-e^{2t_0}},\] де \[ \dot k (s) = \langle N(s) (y(s) - G(s) \cdot \hat x(s)), y(s) - G(s) \cdot \hat x(s)\rangle, \quad k(t_0) = 0, \] тобто \[ \dot k (s) = \langle (y(s) - p \hat x(s)), y(s) - p \hat x(s)\rangle = |y(s) - p \hat x(s)|^2, \quad k(t_0) = 0. \]
\end{solution}

\begin{problem}
	% 5.6
\end{problem}

\begin{solution}
	% 5.6
\end{solution}

\end{document}