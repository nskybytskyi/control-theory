% cd ..\..\Users\NikitaSkybytskyi\Desktop\control-theory

% cls && pdflatex ControlTheoryAlgorithms.tex && cls && pdflatex ControlTheoryAlgorithms.tex && start ControlTheoryAlgorithms.pdf

\documentclass[a5paper, 10pt]{article}
\usepackage[T2A,T1]{fontenc}
\usepackage[utf8]{inputenc}
\usepackage[english, ukrainian]{babel}
\usepackage{amsmath, amssymb}
\usepackage[top = 2 cm, left = 1 cm, right = 1 cm, bottom = 2 cm]{geometry} 

\usepackage{fancyhdr}
\pagestyle{fancy}
\lhead{Алгоритми теорії керування}
\rhead{Нікіта Скибицький}
\cfoot{\thepage}

\usepackage{amsthm}
\newtheorem{definition}{Визначення}
\theoremstyle{definition}
\newtheorem*{problem*}{\normalfont{\textit{Задача}}}
\newtheorem{problem}{\normalfont{\textit{Задача}}}[section]
\newtheorem{algorithm}{\tt Алгоритм}[section]
\newtheorem*{solution}{Розв'язок}

\allowdisplaybreaks
\setlength\parindent{0pt}
\numberwithin{equation}{section}

\usepackage{graphicx}

\newcommand{\JJ}{\mathcal{J}}
\newcommand{\KK}{\mathcal{K}}
\newcommand{\MM}{\mathcal{M}}
\newcommand{\UU}{\mathcal{U}}
\newcommand{\XX}{\mathcal{X}}
\newcommand{\BB}{\mathcal{B}}
\newcommand{\NN}{\mathcal{N}}
\newcommand{\HH}{\mathcal{H}}
\newcommand{\EE}{\mathcal{E}}
\newcommand{\RR}{\mathbb{R}}
\newcommand{\Max}{\displaystyle\max\limits}

\newcommand*\diff{\mathop{}\!\mathrm{d}}

\renewcommand{\phi}{\varphi}
\renewcommand{\SS}{\mathcal{S}}
\renewcommand{\epsilon}{\varepsilon}

\DeclareMathOperator{\erf}{erf}
\DeclareMathOperator{\erfi}{erfi}
\DeclareMathOperator{\signum}{sgn}
\DeclareMathOperator*{\argmin}{arg\,min}

\begin{document}

\section{Системи керування. Постановка задачі оптимального керування}

\subsection{Алгоритми}

\begin{problem*}
	Задана лінійна система керування \[ \dot{x} (t) = A (t) \cdot x(t) + B(t) \cdot u(t),\] де $x \in \RR^n$ -- вектор фазових координат, $A \in \RR^{n\times n}$, $u \in \RR^m$ -- \textbf{відоме} керування, $B \in \RR^{n \times m}$, з початковими умовами \[ x(t_0) = x_0, \] де $t_0 \in \RR^1$, $x_0 \in \RR^n$. Необхідно:
	\begin{enumerate}
		\item Визначити клас керування: програмне чи з оберненим зв'язком.
		\item Знайти траєкторію системи що відповідає заданому керуванню.
		\item Звести задане керування до програмного.
		\item Перевірити траєкторію на неперервну диференційованість.
		\item Порівняти задане керування з іншим керуванням відносно заданого критерію якості \[\JJ = \int_{t_0}^T f \diff t + \Phi(T) \to \min,\] де $\JJ = \JJ(u)\in\RR^1$, $f=f(x,u,t)\in\RR^1$, $\Phi(T)=\Phi(x(T))\in\RR^1$, $T \in \RR^1$.
		\item Знайти фундаментальну матрицю системи, нормовану за моментом $s$, де $s \in \RR^1$.
		\item Побудувати спряжену систему.
	\end{enumerate}
\end{problem*}

\begin{algorithm} \tt
	Розглянемо всі пункти задачі вище.
	\begin{enumerate}
		\item Якщо $u$ не залежить від $x$ то керування програмне, інакше з о\-бер\-не\-ним зв'язком.
		\item Розв'язується система з підставленим $u$.
		\item Замінюється $x$ у визначенні $u$ на знайдену у попередньому пункті \allowbreak тра\-єк\-то\-рі\-ю.
		\item Задача математичного аналізу.
		\item Обчислюється значення критерію якості на обох керуваннях і по\-рів\-ню\-є\-ться.
		\item \begin{enumerate}
			\item Пошук матриці -- задача диференційних рівнянь, або знаходимо з системи $\dot \Theta(t,s) = A(t) \cdot \Theta(t,s)$.
			\item Нормування за моментом $s$ полягає у підборі констант як функ\-цій від $s$ так, щоб $\Theta(s,s)=E$.
		\end{enumerate} 
	\end{enumerate}
\end{algorithm}

\begin{problem*}
	Звести задачу Лагранжа/Больца вигляду \[\JJ = \int_{t_0}^T f \diff t + \Phi(T) \to \inf \] за умов \[ \dot{x} (t) = A (t) \cdot x(t) + B(t) \cdot u(t) \] до задачі Майєра.
\end{problem*}

\begin{algorithm} \tt
	\begin{enumerate}
		\item Вводиться змінна \[x_{n+1} (t) \overset{\text{def}}{=} \int_{t_0}^t f \diff t.\]
		\item Тоді \[ \JJ = x_{n+1} (T) + \Phi(T) \to \inf. \]
		\item До системи додається умова \[ \dot x_{n+1} = f. \]
	\end{enumerate}
\end{algorithm}


\section{Елементи багатозначного аналізу. Множина досяжності}

\subsection{Алгоритми}

\begin{problem*}
	Знайти
	\begin{enumerate}
		\item $A+B$;
		\item $\lambda A$;
		\item $\alpha(A,B)$;
		\item $MA$,
	\end{enumerate}
	де множини $A \subset\RR^m$, $B\subset\RR^m$, скаляр $\lambda\in\RR^1$, матриця $M\in\RR^{n\times m}$.
\end{problem*}

\begin{algorithm} \tt
	Розглянемо всі пункти задачі вище.
	\begin{enumerate}
		\item Знаходимо за визначенням, \[A+B=\{a+b|a\in A,b\in B\}.\]
		\item Знаходимо за визначенням, \[\lambda A =\{\lambda a|a\in A\}.\] 
		\item \begin{enumerate}
			\item Знаходимо $\beta(A,B)$ і $\beta(B,A)$ за визначенням, \[ \beta(A,B) = \max_{a\in A}\rho(a,B), \]
			де \[\rho(a,B) = \min_{b\in B} \rho(a,b).\]
			\item Знаходимо $\alpha(A,B)$ за визначенням, \[\alpha(A,B)=\max\{\beta(A,B),\beta(B,A).\]
		\end{enumerate} 
		\item Знаходимо за визначенням, \[MA=\{Ma|a\in A\}.\]
	\end{enumerate}
\end{algorithm}

\begin{problem*}
	Знайти опорну функцію множини $A \subset \RR^n$.
\end{problem*}

\begin{algorithm} \tt
	\begin{enumerate}
		\item \textbf{Намагаємося} знайти за визначенням, \[ c(A,\psi) = \max_{a\in A} \langle a, \psi \rangle. \]
		\item Якщо не вийшло, то намагаємося знайти за геометричною властивістю: $c(A,\psi)$ -- (орієнтована) відстань від початку координат до опорної \allowbreak пло\-щи\-ни множини $A$, для якої напрямок-вектор $\psi$ є вектором нормалі.
	\end{enumerate}
\end{algorithm}

\begin{problem*}
	Знайти інтеграл Аумана $\JJ = \int F \diff x$, де $F = F(x)\subset\RR^n$.
\end{problem*}

\begin{algorithm} \tt
	\begin{enumerate}
		\item Знаходимо опорну функцію від інтегралу: \[ c(\JJ, \psi) = \int c (F, \psi) \diff x.\]
		\item Знаходимо $\JJ$ як опуклий компакт з відомою опорною функцією $c(\JJ, \psi)$.
	\end{enumerate}
\end{algorithm}

\begin{problem*}
	Знайти множину досяжності системи $\dot x = A x + B u$, де $x(t_0) \in \mathcal{M}_0$, $u \in \mathcal{U}$.
\end{problem*}

\begin{algorithm} \tt
	\begin{enumerate}
		\item Знаходимо фундаментальну матрицю $\Theta(t,s)$ системи нор\-мо\-ва\-ну за моментом $s$.
		\item Знаходимо інтеграл Аумана \[\int_{t_0}^t \Theta(t, s) B(s) \UU(s) \diff s.\]
		\item Використовуємо теорему про вигляд множини досяжності лінійної сис\-те\-ми керування: \[ \XX(t, \MM_0) = \Theta(t, t_0) \MM_0 + \int_{t_0}^t \Theta(t, s) B(s) \UU(s) \diff s. \]
	\end{enumerate}
\end{algorithm}

\begin{problem*}
	Знайти опорну функцію множини досяжності системи $\dot x = A x + B u$, де $x(t_0) \in \mathcal{M}_0$, $u \in \mathcal{U}$.
\end{problem*}

\begin{algorithm} \tt
	\begin{enumerate}
		\item Знаходимо фундаментальну матрицю $\Theta(t,s)$ системи нор\-мо\-ва\-ну за моментом $s$.
		\item Знаходимо опорну функцію $c(\MM_0, \Theta^*(t, t_0) \psi)$.
		\item Знаходимо опорну функцію $c(\UU(s), B^*(s) \Theta^*(t, s) \psi)$.
		\item Використовуємо теорему про вигляд опорної функції множини до\-сяж\-но\-с\-ті лінійної системи керування: \[ c(\XX(t, \MM_0), \psi) = c(\MM_0, \Theta^*(t, t_0) \psi) + \int_{t_0}^t c(\UU(s), B^*(s) \Theta^*(t, s) \psi) \diff s. \]
	\end{enumerate}
\end{algorithm}



\section{Задача про переведення системи з точки в точку. Критерії керованості лінійної системи керування}

\subsection{Алгоритми}

\begin{problem*}
	Перевести систему $\dot x = A x + B u$ з точки $x_0$ в точку $x_T \in \RR^1$ за допомогою керування з класу $K$ (керування, залежні від вектору параметрів $c$).
\end{problem*}

\begin{algorithm} \tt
	\begin{enumerate}
		\item Знаходимо траєкторію системи при заданому керуванні (залежну від параметра $c$). 
		\item Знаходимо з отриманого алгебраїчного рівняння параметр $c$.
	\end{enumerate}
\end{algorithm}

\begin{problem*}
	\begin{enumerate}
		\item Знайти грамміан керованості системи $\dot x = A x + B u$ за визначенням.
		\item Записати систему диференційних рівнянь для знаходження грамміана керованості.
		\item Використовуючи грамміан керованості, знайти інтервал повної керованості системи.
		\item Для цього інтервалу записати керування яке певеродить систему з точки $x_0$ в точку $x_T$/розв'язати задачу оптимального керування.
	\end{enumerate}
\end{problem*}

\begin{algorithm} \tt
	Розглянемо всі пункти задачі вище.
	\begin{enumerate}
		\item \begin{enumerate}
			\item Знаходимо $\Theta(T,s)$.
			\item Використовуємо формулу \[\Phi(T, t_0) = \int_{t_0}^T \Theta(T, s) B(s) B^*(s) \Theta^*(T, s) \diff s.\]
		\end{enumerate}
		\item Записуємо систему \[ \dot \Phi(t, t_0) = A(t) \cdot \Phi(t,t_0)+\Phi(t,t_0)\cdot A^*(t)+B(t)\cdot B^*(t), \Phi(t_0,t_0) = 0. \]
		\item Це інтервал на якому $\Phi(t,t_0) \ne 0$.
		\item Використовуємо формулу \[ u (t) = B^*(t) \cdot \Theta(T,t)\cdot\Phi^{-1}(T,t_0)(x_T-\Theta(T,t_0)\cdot x_0). \]
	\end{enumerate}
\end{algorithm}

\begin{problem*}
	Дослідити стаціонарну систему $\dot x = A x + B u$ на керованість використовуючи другий критерій керованості.
\end{problem*}

\begin{algorithm} \tt
	\begin{enumerate}
		\item Знаходимо $D = \left(B \vdots AB \vdots A^nB \vdots\ldots\vdots A^{n-1}B\right)$.
		\item Якщо $rang D = n$ то стаціонарна системи цілком керована, інакше ні.
	\end{enumerate}
\end{algorithm}


\input{algo-04.tex}

\section{Задача фільтрації. Множинний підхід}

\subsection{Алгоритми}

\begin{problem*}
	Задана динамічна система $\dot x = A x + v$, $y = G x + w$, де $v(t) \in \RR^1$, $w(t) \in \RR^1$ -- невідомі шуми, $x_0 \in \RR^1$ -- невідома початкова умова, $y(t) \in \RR^1$ -- відомі спостереження. 

	\begin{enumerate}
		\item Побудувати інформаційну множину такої системи в момент $\tau \in [0, T]$ за умови, що \[ \int_0^\tau (Mv^2(s) + Nw^2(s)) \diff s + p_0x^2(0) \le \mu^2. \]
		\item Знайти похибку оцінювання.
	\end{enumerate}
\end{problem*}

\begin{algorithm} \tt
	\begin{enumerate}
		\item \begin{enumerate}
			\item Знайдемо $R(t)$ з рівняння Бернуллі \[\dot R (t)= A (t) \cdot R (t)+ R (t) \cdot A^* (t)- R (t) \cdot G^* (t) \cdot N (t) \cdot G (t) \cdot R(t), \quad R(t_0) = p_0^{-1}. \]
			\item Знайдемо $K(t)$ за формулою \[K (t)= R (t) \cdot G^* (t) \cdot N(t).\]
			\item Знайдемо фільтр (спостерігач) за формулою \[ \dot{\hat{x}} (t) = A (t) \cdot \hat x (t) + K (t) \cdot (y (t) - G (t) \cdot \hat x (t)). \]
			\item Знайдемо $k(s)$ з системи \[ \dot k (s) = \langle N(s) (y(s) - G(s) \cdot \hat x(s)), y(s) - G(s) \cdot \hat x(s)\rangle, \quad k(t_0) = 0. \]
			\item Знайдемо $\XX(\tau)$ за формулою \[ \XX(\tau) = \EE (\hat x(\tau), (\mu^2 - k(\tau)) \cdot R(\tau)). \]
		\end{enumerate}
		\item Похибка $e(\tau)$ оцінювання задовольняє оцінці \[ |e(\tau)| \le \sqrt{\mu^2-k(\tau)} \cdot \sqrt{\lambda_* (R(\tau))}.\]
	\end{enumerate}
\end{algorithm}


\input{algo-06.tex}

\input{algo-07.tex}

\input{algo-08.tex}

\section{Дискретний варіант методу динамічного програмування}

\subsection{Алгоритми}

\begin{problem*}
	Розглядається задача оптимального керування \[ \JJ (u, x) = \sum_{k=0}^{N-1} g_k(x(k),u(k))+\Phi(x(N))\to\min \] при умовах \[ x(k + 1) = f_k(x(k),u(k)), \quad k=0,1,\ldots,N-1, \] \[ x(k)\in\XX_k, \quad k=0,1,\ldots,N, \] \[ u(k)\in\UU_k, \quad k=0,1,\ldots,N-1. \] Знайти оптимальне керування, оптимальну траєкторію, функцію Белмана і оптимальне значення критерію якості.
\end{problem*}

\begin{algorithm} \tt
	\begin{enumerate}
		\item $\BB_N(z) = \Phi(z)$.
		\item Для $s=\overline{N-1..0}$ записуємо і розв'язуємо дискретне рівняння Белмана: \[\BB_s(z) = \min_{u\in \UU_s} (g_s(z,u)+\BB_{s+1}(f_s(z,u))) \] для всіх $z \in \XX_s$, запам'ятовуючи $\{u_*(s)\}$.
		\item Знаходимо $x_*(0)$ як \[ x_*(0) = \argmin_{z\in \XX_0} \BB_0(z).\]
		\item Знаходимо $\JJ_*$ як $\JJ_* = \BB_0(x_*(0))$.
		\item Для $s=\overline{0..N-1}$ відновлюємо $x_*(s+1)$ за відомим керуванням: \[x_*(s+1) = f_s(x_*(s),u_*(s)).\] 
	\end{enumerate}
\end{algorithm}

\input{algo-10.tex}

\end{document}