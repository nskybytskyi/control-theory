\setcounter{section}{6}

\section{Принцип максимуму Понтрягіна для задачі з вільними правим кінцем}

\subsection{Лекція}

\subsubsection{Постановка задачі і формулювання принципу максимуму}

Розглянемо задачу Больца з вільним правим кінцем
\begin{equation}
    \label{eq:6.1}
    \mathcal{J}(u, x) = \Int_{t_0}^T f_0 (x(s), u(s), s) \dif s + \Phi(x(T))
\end{equation}
за умов
\begin{equation}
    \label{eq:6.2}
    \dfrac{\dif x(t)}{\dif t} = f(x(t), u(t), t), t \in [t_0, T],
\end{equation}
\begin{equation}
    \label{eq:6.3}
    x(t_0) = x_0.
\end{equation}
Тут $x = (x_1, x_2, \ldots, x_n)^*$ -- фазові координати, $u = (u_1, u_2, \ldots, u_m)^*$ -- керування. Керування $u(\cdot)$ є кусково-неперервним. Також $u(t) \in \mathcal{U}, t \in [t_0, T]$, де $\mathcal{U} \subseteq \RR^m$, не залежить від часу. \\

$f_0(x, u, t) \to \RR$, $f(x, u, t) \to \RR^n$ є неперервними за сукупністю змінних, разом зі своїми градієнтами за $x$, $\Phi(x)$ -- неперервно диференційовна, $(x, u, t) \in \RR^n \times \mathcal{U} \times [t_0, T]$, $x_0 \in \RR^n$. \\

За цих умов справджується теорема про існування та єдиність (кусково-гладкого) розв'язку задачі Коші для системи \ref{eq:6.2} для довільного керування. \\

Моменти часу $t_0$ і $T$ фіксовані, а обмеження на фазові координати відсутні.

\subsection{Аудиторне заняття}

\begin{problem}

\end{problem}

\begin{problem}

\end{problem}

\begin{problem}

\end{problem}

\begin{problem}

\end{problem}

\begin{problem}

\end{problem}

\begin{problem}

\end{problem}

\subsection{Домашнє завдання}

\begin{problem}

\end{problem}

\begin{problem}

\end{problem}

\begin{problem}

\end{problem}

\begin{problem}

\end{problem}

\begin{problem}

\end{problem}

\begin{problem}

\end{problem}